% ============================================
% CHAPITRE V - RÉALISATION ET IMPLÉMENTATION
% ============================================
\chapter{Réalisation et Implémentation}
\thispagestyle{fancy}

\section{Environnement de Développement}

\lettrine[lines=3, lhang=0.15, loversize=0.1, findent=3pt]{\textcolor{primaryGreen}{C}}{e chapitre} présente la réalisation concrète du système AI Sentinel, de la configuration de l'environnement de développement jusqu'au déploiement des différents modules. Nous détaillerons les choix technologiques, les extraits de code significatifs et les défis rencontrés lors de l'implémentation.

\subsection{Outils et Technologies}

Le tableau suivant récapitule l'ensemble des technologies utilisées dans le projet AI Sentinel :

\begin{table}[H]
\centering
\caption{Stack technologique complète}
\label{tab:stack}
\rowcolors{2}{mintGreen!30}{white}
\begin{tabular}{l l l p{5cm}}
\toprule
\rowcolor{primaryGreen}
\textcolor{white}{\textbf{Catégorie}} & \textcolor{white}{\textbf{Technologie}} & \textcolor{white}{\textbf{Version}} & \textcolor{white}{\textbf{Rôle}} \\
\midrule
\multirow{5}{*}{Backend} 
& Python & 3.10+ & Langage principal \\
& FastAPI & 0.100+ & Framework API REST \\
& TensorFlow & 2.x & Deep Learning (MobileNetV2, CAM) \\
& Ultralytics & 8.x & YOLO pour détection temps réel \\
& OpenCV & 4.x & Traitement d'images et vidéo \\
\midrule
\multirow{5}{*}{Frontend}
& React & 19 & Framework UI \\
& TypeScript & 5.x & Typage statique \\
& Vite & 5.x & Build tool moderne \\
& TailwindCSS & 3.x & Framework CSS utility-first \\
& Leaflet & 1.9 & Cartographie interactive \\
\midrule
\multirow{3}{*}{DevOps}
& Git/GitHub & - & Gestion de version \\
& npm & 10.x & Gestionnaire de paquets frontend \\
& pip & latest & Gestionnaire de paquets Python \\
\bottomrule
\end{tabular}
\end{table}

\subsection{Configuration du Projet}

L'architecture du projet suit une organisation claire séparant le frontend du backend.

\begin{techbox}{Structure du Projet}
\begin{verbatim}
WildFireDetection/
├── backend/
│   ├── main.py                    # Point d'entrée FastAPI
│   ├── requirements.txt           # Dépendances Python
│   ├── .env                       # Variables d'environnement
│   ├── services/
│   │   ├── yolo_service.py
│   │   ├── classification_service.py
│   │   ├── firms_service.py
│   │   ├── sentinel_service.py
│   │   ├── prediction_service.py
│   │   ├── monitoring_service.py
│   │   ├── email_service.py
│   │   └── telegram_service.py
│   └── models/
│       ├── mobilenetv2_fire_detector.h5
│       ├── best.pt
│       └── cam_model.h5
├── frontend/
│   ├── src/
│   │   ├── components/
│   │   ├── pages/
│   │   ├── services/
│   │   └── App.tsx
│   ├── package.json
│   ├── vite.config.ts
│   └── tailwind.config.js
└── README.md
\end{verbatim}
\end{techbox}

\subsubsection{Variables d'Environnement}

\begin{techbox}{Configuration .env}
\begin{verbatim}
# NASA FIRMS
NASA_FIRMS_MAP_KEY=your_map_key_here

# Sentinel Hub
SENTINEL_CLIENT_ID=your_client_id
SENTINEL_CLIENT_SECRET=your_client_secret

# Telegram
TELEGRAM_BOT_TOKEN=123456789:ABCdefGHI...
TELEGRAM_CHAT_ID=-1001234567890

# Email
SMTP_SERVER=smtp.gmail.com
SMTP_PORT=587
SMTP_EMAIL=your_email@gmail.com
SMTP_PASSWORD=your_app_password
ALERT_RECIPIENT=recipient@example.com
\end{verbatim}
\end{techbox}

% ============================================
\newpage
\section{Implémentation du Backend}
% ============================================

Cette section présente les extraits de code les plus significatifs de l'implémentation backend.

\subsection{Configuration FastAPI}

Le point d'entrée de l'application configure FastAPI avec le middleware CORS nécessaire pour la communication avec le frontend.

\begin{techbox}{main.py - Configuration de l'Application}
\begin{verbatim}
from fastapi import FastAPI, UploadFile, File
from fastapi.middleware.cors import CORSMiddleware
from fastapi.responses import StreamingResponse, JSONResponse
import uvicorn

# Initialisation de l'application
app = FastAPI(
    title="AI Sentinel - WildFire Detection API",
    description="API pour la détection d'incendies de forêt",
    version="1.0.0"
)

# Configuration CORS pour le frontend React
app.add_middleware(
    CORSMiddleware,
    allow_origins=["http://localhost:5173", "http://localhost:3000"],
    allow_credentials=True,
    allow_methods=["*"],
    allow_headers=["*"],
)

# Import des services
from services.yolo_service import YoloService
from services.classification_service import ClassificationService
from services.firms_service import FirmsService
from services.prediction_service import PredictionService

# Initialisation des services (singleton)
yolo_service = YoloService()
classification_service = ClassificationService()
firms_service = FirmsService()
prediction_service = PredictionService()
\end{verbatim}
\end{techbox}

\subsection{Implémentation du Service YOLO}

Le service YOLO gère la détection d'objets en temps réel sur les flux vidéo.

\begin{techbox}{yolo\_service.py - Service de Détection}
\begin{verbatim}
from ultralytics import YOLO
import cv2
import numpy as np

class YoloService:
    def __init__(self, model_path: str = "models/best.pt"):
        """Initialise le service avec le modèle YOLO."""
        self.model = YOLO(model_path)
        self.class_names = {0: "Smoke", 1: "Fire"}
        self.confidence_threshold = 0.5
        
    def detect(self, frame: np.ndarray) -> tuple:
        """Effectue la détection sur une frame."""
        # Inférence YOLO
        results = self.model(frame, verbose=False)[0]
        
        detections = []
        annotated_frame = frame.copy()
        
        for box in results.boxes:
            # Extraction des données
            x1, y1, x2, y2 = map(int, box.xyxy[0])
            confidence = float(box.conf[0])
            class_id = int(box.cls[0])
            
            if confidence >= self.confidence_threshold:
                class_name = self.class_names.get(class_id, "Unknown")
                
                # Couleur selon la classe
                color = (0, 0, 255) if class_name == "Fire" else (128, 128, 128)
                
                # Dessin du rectangle et label
                cv2.rectangle(annotated_frame, (x1, y1), (x2, y2), color, 2)
                label = f"{class_name}: {confidence:.2f}"
                cv2.putText(annotated_frame, label, (x1, y1 - 10),
                           cv2.FONT_HERSHEY_SIMPLEX, 0.5, color, 2)
                
                detections.append({
                    "class": class_name,
                    "confidence": confidence,
                    "bbox": [x1, y1, x2, y2]
                })
        
        return annotated_frame, detections
    
    def has_fire(self, detections: list) -> bool:
        """Vérifie si un feu a été détecté."""
        return any(d["class"] == "Fire" for d in detections)
\end{verbatim}
\end{techbox}

\subsection{Implémentation du Service NASA FIRMS}

\begin{techbox}{firms\_service.py - Récupération des Hotspots}
\begin{verbatim}
import requests
import csv
from io import StringIO

class FirmsService:
    BASE_URL = "https://firms.modaps.eosdis.nasa.gov/api/area/csv"
    
    # Régions prédéfinies du Maroc
    REGIONS = {
        "north": {"lat": 34.0, "lng": -6.0, "name": "North Morocco"},
        "rif": {"lat": 34.5, "lng": -5.0, "name": "Rif Mountains"},
        "oriental": {"lat": 33.5, "lng": -3.0, "name": "Oriental"},
        "casablanca": {"lat": 33.0, "lng": -8.0, "name": "Casablanca"},
        "middle_atlas": {"lat": 32.5, "lng": -6.0, "name": "Middle Atlas"},
        "marrakech": {"lat": 31.0, "lng": -8.5, "name": "Marrakech"},
        "high_atlas": {"lat": 30.5, "lng": -8.0, "name": "High Atlas"},
        "souss": {"lat": 29.5, "lng": -10.0, "name": "Souss-Massa"}
    }
    
    def __init__(self, api_key: str):
        self.api_key = api_key
    
    def fetch_hotspots(self, region: str, days: int = 1) -> list:
        """Récupère les hotspots pour une région."""
        if region not in self.REGIONS:
            return []
        
        coords = self.REGIONS[region]
        # Bounding box autour du point central
        bbox = f"{coords['lng']-2},{coords['lat']-2},{coords['lng']+2},{coords['lat']+2}"
        
        url = f"{self.BASE_URL}/{self.api_key}/VIIRS_SNPP_NRT/{bbox}/{days}"
        
        try:
            response = requests.get(url, timeout=30)
            response.raise_for_status()
            
            hotspots = self._parse_csv(response.text)
            return self._enrich_with_weather(hotspots)
            
        except requests.RequestException as e:
            print(f"Error fetching FIRMS data: {e}")
            return []
    
    def _parse_csv(self, csv_data: str) -> list:
        """Parse les données CSV de FIRMS."""
        hotspots = []
        reader = csv.DictReader(StringIO(csv_data))
        
        for row in reader:
            hotspots.append({
                "latitude": float(row.get("latitude", 0)),
                "longitude": float(row.get("longitude", 0)),
                "brightness": float(row.get("bright_ti4", 0)),
                "confidence": row.get("confidence", "nominal"),
                "frp": float(row.get("frp", 0)),
                "daynight": row.get("daynight", "D"),
                "acq_date": row.get("acq_date", ""),
                "acq_time": row.get("acq_time", "")
            })
        
        return hotspots
\end{verbatim}
\end{techbox}

\subsection{Implémentation du Service Sentinel Hub}

\begin{techbox}{sentinel\_service.py - API Sentinel Hub}
\begin{verbatim}
import requests
from datetime import datetime, timedelta

class SentinelService:
    AUTH_URL = "https://services.sentinel-hub.com/oauth/token"
    PROCESS_URL = "https://services.sentinel-hub.com/api/v1/process"
    
    # Evalscript pour détection fire (SWIR enhanced)
    FIRE_EVALSCRIPT = """
    //VERSION=3
    function setup() {
        return { input: ["B04", "B08", "B12"], output: { bands: 3 } };
    }
    function evaluatePixel(sample) {
        let fire_index = (sample.B12 - sample.B08) / (sample.B12 + sample.B08);
        if (sample.B12 > 0.3 && fire_index > 0.3) {
            return [1, 0, 0]; // Rouge = feu probable
        }
        return [3.5*sample.B04, 3.5*sample.B08*0.5, 3.5*sample.B04*0.5];
    }
    """
    
    def __init__(self, client_id: str, client_secret: str):
        self.client_id = client_id
        self.client_secret = client_secret
        self.token = None
        self.token_expiry = None
    
    def authenticate(self):
        """Obtient un token OAuth2."""
        response = requests.post(
            self.AUTH_URL,
            data={
                "grant_type": "client_credentials",
                "client_id": self.client_id,
                "client_secret": self.client_secret
            }
        )
        response.raise_for_status()
        data = response.json()
        self.token = data["access_token"]
        self.token_expiry = datetime.now() + timedelta(seconds=data["expires_in"] - 60)
    
    def get_image(self, lat: float, lng: float, size: int = 512) -> bytes:
        """Récupère une image satellite pour les coordonnées données."""
        if not self.token or datetime.now() > self.token_expiry:
            self.authenticate()
        
        # Bounding box (environ 10km)
        delta = 0.05
        bbox = [lng - delta, lat - delta, lng + delta, lat + delta]
        
        # Plage de dates (derniers 30 jours)
        to_date = datetime.now()
        from_date = to_date - timedelta(days=30)
        
        payload = {
            "input": {
                "bounds": {"bbox": bbox, "properties": {"crs": "http://www.opengis.net/def/crs/EPSG/0/4326"}},
                "data": [{"type": "sentinel-2-l2a", "dataFilter": {
                    "timeRange": {"from": from_date.isoformat() + "Z", "to": to_date.isoformat() + "Z"},
                    "maxCloudCoverage": 30
                }}]
            },
            "output": {"width": size, "height": size, "responses": [{"format": {"type": "image/png"}}]},
            "evalscript": self.FIRE_EVALSCRIPT
        }
        
        response = requests.post(
            self.PROCESS_URL,
            headers={"Authorization": f"Bearer {self.token}", "Content-Type": "application/json"},
            json=payload
        )
        response.raise_for_status()
        return response.content
\end{verbatim}
\end{techbox}

\subsection{Endpoint Flux Vidéo Temps Réel}

\begin{techbox}{main.py - Streaming MJPEG}
\begin{verbatim}
import cv2
from fastapi.responses import StreamingResponse

# Variable globale pour le cooldown des alertes
last_alert_time = 0
ALERT_COOLDOWN = 30  # secondes

def generate_frames():
    """Générateur de frames pour le flux MJPEG."""
    global last_alert_time
    
    cap = cv2.VideoCapture(0)  # Webcam
    cap.set(cv2.CAP_PROP_FRAME_WIDTH, 640)
    cap.set(cv2.CAP_PROP_FRAME_HEIGHT, 480)
    
    while True:
        ret, frame = cap.read()
        if not ret:
            break
        
        # Détection YOLO
        annotated_frame, detections = yolo_service.detect(frame)
        
        # Vérification alerte
        current_time = time.time()
        if yolo_service.has_fire(detections):
            if current_time - last_alert_time > ALERT_COOLDOWN:
                # Envoyer alerte (async en production)
                send_telegram_alert(detections)
                last_alert_time = current_time
        
        # Encodage JPEG
        _, buffer = cv2.imencode('.jpg', annotated_frame)
        frame_bytes = buffer.tobytes()
        
        yield (b'--frame\r\n'
               b'Content-Type: image/jpeg\r\n\r\n' + frame_bytes + b'\r\n')
    
    cap.release()

@app.get("/video_feed")
async def video_feed():
    """Endpoint pour le flux vidéo en temps réel."""
    return StreamingResponse(
        generate_frames(),
        media_type="multipart/x-mixed-replace; boundary=frame"
    )
\end{verbatim}
\end{techbox}

\subsection{Service de Prédiction}

\begin{techbox}{prediction\_service.py - Algorithme de Propagation}
\begin{verbatim}
class PredictionService:
    BASE_RADIUS = 3.0  # km
    
    def calculate_spread(self, brightness: float, confidence: str, 
                         wind_speed: float, wind_direction: float) -> dict:
        """Calcule le rayon de propagation estimé."""
        
        # Facteur luminosité
        if brightness > 350:
            brightness_factor = 3.0
        elif brightness > 320:
            brightness_factor = 1.5
        else:
            brightness_factor = 0.0
        
        # Facteur confiance
        confidence_factors = {"high": 1.0, "nominal": 0.0, "low": -1.0}
        confidence_factor = confidence_factors.get(confidence.lower(), 0.0)
        
        # Facteur vent
        wind_factor = (wind_speed / 30) * 2.0
        
        # Calcul du rayon
        radius = self.BASE_RADIUS + brightness_factor + confidence_factor + wind_factor
        
        # Bornage entre 1 et 15 km
        radius = max(1.0, min(15.0, radius))
        
        # Surface en km²
        import math
        area = math.pi * radius ** 2
        
        # Niveau de risque
        if radius > 10:
            risk_level = "EXTREME"
        elif radius > 7:
            risk_level = "HIGH"
        elif radius > 4:
            risk_level = "MODERATE"
        else:
            risk_level = "LOW"
        
        return {
            "predicted_radius_km": round(radius, 2),
            "predicted_area_km2": round(area, 2),
            "risk_level": risk_level,
            "factors": {
                "base": self.BASE_RADIUS,
                "brightness": brightness_factor,
                "confidence": confidence_factor,
                "wind": round(wind_factor, 2)
            }
        }
\end{verbatim}
\end{techbox}

% ============================================
\newpage
\section{Entraînement des Modèles IA}
% ============================================

\subsection{Préparation du Dataset}

\subsubsection{Dataset Classification (MobileNetV2)}

Le modèle de classification a été entraîné sur un dataset de plus de 43 000 images provenant de sources publiques.

\begin{table}[H]
\centering
\caption{Répartition du dataset de classification}
\label{tab:dataset}
\rowcolors{2}{mintGreen!30}{white}
\begin{tabular}{l c c c}
\toprule
\rowcolor{primaryGreen}
\textcolor{white}{\textbf{Classe}} & \textcolor{white}{\textbf{Training}} & \textcolor{white}{\textbf{Validation}} & \textcolor{white}{\textbf{Test}} \\
\midrule
Smoke & 10,500 & 2,625 & 3,500 \\
Fire & 10,800 & 2,700 & 3,600 \\
Non-Fire & 11,200 & 2,800 & 3,700 \\
\midrule
\textbf{Total} & \textbf{32,500} & \textbf{8,125} & \textbf{10,800} \\
\bottomrule
\end{tabular}
\end{table}

\subsection{Processus d'Entraînement MobileNetV2}

\begin{techbox}{Notebook d'Entraînement - Extraits}
\begin{verbatim}
import tensorflow as tf
from tensorflow.keras.applications import MobileNetV2
from tensorflow.keras.layers import GlobalAveragePooling2D, Dense, Dropout
from tensorflow.keras.models import Model
from tensorflow.keras.optimizers import Adam

# Chargement du modèle de base
base_model = MobileNetV2(
    weights='imagenet',
    include_top=False,
    input_shape=(224, 224, 3)
)

# Construction de la tête de classification
x = base_model.output
x = GlobalAveragePooling2D()(x)
x = Dropout(0.3)(x)
x = Dense(128, activation='relu')(x)
predictions = Dense(3, activation='softmax')(x)

model = Model(inputs=base_model.input, outputs=predictions)

# PHASE 1 : Feature Extraction (backbone gelé)
for layer in base_model.layers:
    layer.trainable = False

model.compile(
    optimizer=Adam(learning_rate=1e-4),
    loss='categorical_crossentropy',
    metrics=['accuracy']
)

history_phase1 = model.fit(
    train_generator,
    epochs=10,
    validation_data=val_generator,
    callbacks=[early_stopping, reduce_lr]
)

# PHASE 2 : Fine-tuning (100 dernières couches)
for layer in base_model.layers[-100:]:
    layer.trainable = True

model.compile(
    optimizer=Adam(learning_rate=1e-5),
    loss='categorical_crossentropy',
    metrics=['accuracy']
)

history_phase2 = model.fit(
    train_generator,
    epochs=5,
    validation_data=val_generator
)

# Sauvegarde
model.save('mobilenetv2_fire_detector.h5')
\end{verbatim}
\end{techbox}

\subsection{Courbes d'Apprentissage}

\begin{center}
\begin{tikzpicture}
\begin{axis}[
    width=14cm,
    height=7cm,
    xlabel={Époques},
    ylabel={Accuracy (\%)},
    xmin=0, xmax=15,
    ymin=75, ymax=100,
    legend pos=south east,
    grid=major,
    grid style={dashed, gray!30},
    title={\textbf{Courbes d'Apprentissage MobileNetV2}}
]

% Training accuracy
\addplot[color=primaryGreen, line width=2pt, mark=*] coordinates {
    (1, 82) (2, 87) (3, 90) (4, 92) (5, 93)
    (6, 94) (7, 95) (8, 96) (9, 96.5) (10, 97)
    (11, 97.2) (12, 97.5) (13, 97.7) (14, 97.8) (15, 98)
};
\addlegendentry{Training Accuracy}

% Validation accuracy
\addplot[color=accentTeal, line width=2pt, mark=square*] coordinates {
    (1, 80) (2, 85) (3, 88) (4, 90) (5, 91)
    (6, 92) (7, 93) (8, 94) (9, 95) (10, 95.5)
    (11, 96) (12, 96.5) (13, 96.8) (14, 97) (15, 97.2)
};
\addlegendentry{Validation Accuracy}

% Ligne verticale pour Phase 2
\draw[dashed, moroccanRed, line width=1pt] (axis cs:10,75) -- (axis cs:10,100);
\node[font=\footnotesize, text=moroccanRed] at (axis cs:10.5,78) {Fine-tuning};

\end{axis}
\end{tikzpicture}
\end{center}

\subsection{Matrice de Confusion}

\begin{center}
\begin{tikzpicture}
    % Cadre
    \draw[primaryGreen, line width=2pt, rounded corners=10pt] (-0.5, -4) rectangle (8, 4);
    
    % Titre
    \node[font=\bfseries\large, text=darkGreen] at (3.75, 3.5) {Matrice de Confusion - Test Set};
    
    % En-têtes colonnes
    \node[font=\bfseries] at (2, 2.5) {Smoke};
    \node[font=\bfseries] at (4, 2.5) {Fire};
    \node[font=\bfseries] at (6, 2.5) {Non-Fire};
    
    % En-têtes lignes
    \node[font=\bfseries, rotate=90] at (-0.2, 1) {Smoke};
    \node[font=\bfseries, rotate=90] at (-0.2, -0.5) {Fire};
    \node[font=\bfseries, rotate=90] at (-0.2, -2) {Non-Fire};
    
    % Cellules diagonales (corrects)
    \fill[primaryGreen!40] (1.2, 0.2) rectangle (2.8, 1.8);
    \fill[primaryGreen!40] (3.2, -1.3) rectangle (4.8, 0.3);
    \fill[primaryGreen!40] (5.2, -2.8) rectangle (6.8, -1.2);
    
    % Valeurs
    \node[font=\bfseries\large] at (2, 1) {3,420};
    \node[font=\large] at (4, 1) {45};
    \node[font=\large] at (6, 1) {35};
    
    \node[font=\large] at (2, -0.5) {52};
    \node[font=\bfseries\large] at (4, -0.5) {3,510};
    \node[font=\large] at (6, -0.5) {38};
    
    \node[font=\large] at (2, -2) {28};
    \node[font=\large] at (4, -2) {35};
    \node[font=\bfseries\large] at (6, -2) {3,637};
    
    % Précisions par classe
    \node[font=\small, text=primaryGreen] at (7.5, 1) {97.7\%};
    \node[font=\small, text=primaryGreen] at (7.5, -0.5) {97.5\%};
    \node[font=\small, text=primaryGreen] at (7.5, -2) {98.3\%};
    
    % Précision globale
    \node[font=\bfseries, text=darkGreen] at (3.75, -3.5) {Précision Globale : 97.8\%};
    
\end{tikzpicture}
\end{center}

% ============================================
\newpage
\section{Implémentation du Frontend}
% ============================================

\subsection{Configuration React + Vite}

\begin{techbox}{vite.config.ts}
\begin{verbatim}
import { defineConfig } from 'vite'
import react from '@vitejs/plugin-react'

export default defineConfig({
  plugins: [react()],
  server: {
    port: 5173,
    proxy: {
      '/api': {
        target: 'http://localhost:8000',
        changeOrigin: true
      }
    }
  }
})
\end{verbatim}
\end{techbox}

\subsection{Composant Flux Vidéo}

\begin{techbox}{RealTimeDetection.tsx}
\begin{verbatim}
import React, { useState } from 'react';
import { motion } from 'framer-motion';

const RealTimeDetection: React.FC = () => {
  const [isStreaming, setIsStreaming] = useState(false);
  
  return (
    <div className="min-h-screen bg-gradient-to-br from-green-900 to-emerald-800 p-8">
      <motion.div 
        initial={{ opacity: 0, y: 20 }}
        animate={{ opacity: 1, y: 0 }}
        className="max-w-6xl mx-auto"
      >
        <h1 className="text-4xl font-bold text-white mb-8">
          🔥 Real-Time Fire Detection
        </h1>
        
        <div className="grid grid-cols-1 lg:grid-cols-3 gap-6">
          {/* Video Stream */}
          <div className="lg:col-span-2 bg-black/30 backdrop-blur-lg rounded-2xl p-4">
            {isStreaming ? (
              <img
                src="http://localhost:8000/video_feed"
                alt="Live Detection Stream"
                className="w-full rounded-xl"
              />
            ) : (
              <div className="aspect-video bg-gray-800 rounded-xl flex items-center justify-center">
                <p className="text-gray-400">Click Start to begin detection</p>
              </div>
            )}
          </div>
          
          {/* Controls */}
          <div className="space-y-4">
            <button
              onClick={() => setIsStreaming(!isStreaming)}
              className={`w-full py-4 rounded-xl font-bold text-white transition-all ${
                isStreaming 
                  ? 'bg-red-500 hover:bg-red-600' 
                  : 'bg-green-500 hover:bg-green-600'
              }`}
            >
              {isStreaming ? '⏹ Stop Detection' : '▶ Start Detection'}
            </button>
            
            <div className="bg-white/10 backdrop-blur-lg rounded-xl p-4">
              <h3 className="text-lg font-bold text-white mb-2">Statistics</h3>
              <div className="grid grid-cols-2 gap-4">
                <div className="bg-red-500/20 rounded-lg p-3 text-center">
                  <span className="text-2xl">🔥</span>
                  <p className="text-white font-bold">Fire</p>
                </div>
                <div className="bg-gray-500/20 rounded-lg p-3 text-center">
                  <span className="text-2xl">💨</span>
                  <p className="text-white font-bold">Smoke</p>
                </div>
              </div>
            </div>
          </div>
        </div>
      </motion.div>
    </div>
  );
};

export default RealTimeDetection;
\end{verbatim}
\end{techbox}

\subsection{Composant Carte Interactive}

\begin{techbox}{HotspotMap.tsx - Carte Leaflet}
\begin{verbatim}
import React from 'react';
import { MapContainer, TileLayer, CircleMarker, Popup } from 'react-leaflet';
import 'leaflet/dist/leaflet.css';

interface Hotspot {
  latitude: number;
  longitude: number;
  brightness: number;
  confidence: string;
  spreadRadius: number;
}

const getColorByConfidence = (confidence: string): string => {
  switch (confidence.toLowerCase()) {
    case 'high': return '#DC2626';    // Rouge
    case 'nominal': return '#F59E0B'; // Orange
    case 'low': return '#FCD34D';     // Jaune
    default: return '#6B7280';
  }
};

const HotspotMap: React.FC<{ hotspots: Hotspot[] }> = ({ hotspots }) => {
  return (
    <MapContainer 
      center={[31.7917, -7.0926]} 
      zoom={6} 
      className="h-[600px] w-full rounded-xl"
    >
      <TileLayer
        url="https://{s}.tile.openstreetmap.org/{z}/{x}/{y}.png"
        attribution='&copy; OpenStreetMap contributors'
      />
      
      {hotspots.map((spot, index) => (
        <React.Fragment key={index}>
          {/* Cercle de propagation */}
          <CircleMarker
            center={[spot.latitude, spot.longitude]}
            radius={spot.spreadRadius * 5}
            pathOptions={{
              color: getColorByConfidence(spot.confidence),
              fillColor: getColorByConfidence(spot.confidence),
              fillOpacity: 0.2
            }}
          />
          
          {/* Marqueur central */}
          <CircleMarker
            center={[spot.latitude, spot.longitude]}
            radius={8}
            pathOptions={{
              color: getColorByConfidence(spot.confidence),
              fillColor: getColorByConfidence(spot.confidence),
              fillOpacity: 0.8
            }}
          >
            <Popup>
              <div className="text-sm">
                <p><strong>Brightness:</strong> {spot.brightness}K</p>
                <p><strong>Confidence:</strong> {spot.confidence}</p>
                <p><strong>Spread:</strong> {spot.spreadRadius.toFixed(2)} km</p>
                <a 
                  href={`https://maps.google.com/?q=${spot.latitude},${spot.longitude}`}
                  target="_blank"
                  className="text-blue-500 underline"
                >
                  View on Google Maps
                </a>
              </div>
            </Popup>
          </CircleMarker>
        </React.Fragment>
      ))}
    </MapContainer>
  );
};

export default HotspotMap;
\end{verbatim}
\end{techbox}

% ============================================
\newpage
\section{Intégration des Services Externes}
% ============================================

\subsection{API NASA FIRMS}

\begin{infobox}{Configuration NASA FIRMS}
\begin{itemize}[leftmargin=0.5cm, itemsep=5pt]
    \item \textbf{URL de base :} \texttt{https://firms.modaps.eosdis.nasa.gov/api/area/csv}
    \item \textbf{Authentification :} MAP\_KEY (gratuit après inscription)
    \item \textbf{Source de données :} VIIRS\_SNPP\_NRT (Near Real-Time)
    \item \textbf{Format :} CSV avec colonnes latitude, longitude, brightness, confidence, frp
    \item \textbf{Latence :} ~3 heures après passage satellite
\end{itemize}
\end{infobox}

\subsection{Sentinel Hub}

\begin{infobox}{Configuration Sentinel Hub}
\begin{itemize}[leftmargin=0.5cm, itemsep=5pt]
    \item \textbf{Authentification :} OAuth2 Client Credentials
    \item \textbf{API :} Process API v1
    \item \textbf{Type de données :} Sentinel-2 L2A (atmosphériquement corrigé)
    \item \textbf{Evalscripts :} True Color et Fire Detection (SWIR enhanced)
    \item \textbf{Résolution :} 10m/pixel (bandes visibles)
\end{itemize}
\end{infobox}

\subsection{Services de Notification}

\begin{techbox}{telegram\_service.py}
\begin{verbatim}
import requests

class TelegramService:
    API_URL = "https://api.telegram.org/bot{token}"
    
    def __init__(self, token: str, chat_id: str):
        self.token = token
        self.chat_id = chat_id
        self.base_url = self.API_URL.format(token=token)
    
    def send_alert(self, message: str, image_path: str = None):
        """Envoie une alerte Telegram avec image optionnelle."""
        if image_path:
            with open(image_path, 'rb') as photo:
                response = requests.post(
                    f"{self.base_url}/sendPhoto",
                    data={"chat_id": self.chat_id, "caption": message, "parse_mode": "Markdown"},
                    files={"photo": photo}
                )
        else:
            response = requests.post(
                f"{self.base_url}/sendMessage",
                data={"chat_id": self.chat_id, "text": message, "parse_mode": "Markdown"}
            )
        return response.json()
\end{verbatim}
\end{techbox}

\vspace{1cm}

% Transition
\begin{center}
\begin{tikzpicture}
    \node[
        fill=primaryGreen!10,
        draw=primaryGreen,
        line width=1.5pt,
        rounded corners=12pt,
        inner sep=20pt,
        text width=13cm,
        align=center
    ] {
        \textcolor{primaryGreen}{\fontsize{24}{28}\selectfont\faArrowCircleRight}\\[15pt]
        \large\textbf{Chapitre Suivant}\\[10pt]
        \normalsize Le prochain chapitre présente les résultats obtenus,\\
        les démonstrations du système et l'évaluation des performances.\\[10pt]
        \textit{\textcolor{textGray}{Chapitre VI --- Résultats et Évaluation}}
    };
\end{tikzpicture}
\end{center}

\end{document}
