% ============================================
% PRÉSENTATION PFE COMPLÈTE - AI SENTINEL
% 40+ Slides avec Benchmark
% ============================================
\documentclass[aspectratio=169, 10pt]{beamer}
\usetheme{metropolis}
\usepackage[utf8]{inputenc}
\usepackage[T1]{fontenc}
\usepackage{lmodern}
\usepackage[french]{babel}
\usepackage{graphicx}
\usepackage{tikz}
\usepackage{pgfplots}
\pgfplotsset{compat=1.17}
\usetikzlibrary{shapes.geometric, arrows, positioning, calc, shadows, backgrounds, matrix, fit}
\usepackage{fontawesome5}
\usepackage{booktabs}
\usepackage{hyperref}
\usepackage{xcolor}
\usepackage{tcolorbox}
\usepackage{multirow}

% Couleurs
\definecolor{forest}{HTML}{2E7D32}
\definecolor{fire}{HTML}{D32F2F}
\definecolor{tech}{HTML}{0277BD}
\definecolor{warn}{HTML}{FF6F00}
\definecolor{mint}{HTML}{E8F5E9}
\definecolor{slate}{HTML}{37474F}

\setbeamercolor{frametitle}{fg=white, bg=forest}
\setbeamercolor{progress bar}{fg=warn, bg=mint}
\setbeamertemplate{navigation symbols}{}

\title{\textbf{\faLeaf~AI SENTINEL}}
\subtitle{Détection Intelligente des Incendies de Forêt}
\author{[Nom Étudiant] \\ Encadrant: [Nom Encadrant]}
\date{2024-2025}

\begin{document}

% ========== SLIDE 1: TITRE ==========
{
\setbeamertemplate{footline}{}
\begin{frame}[plain]
\begin{tikzpicture}[remember picture,overlay]
\shade[top color=forest!90!black, bottom color=forest!60!black] 
    (current page.north west) rectangle (current page.south east);
\node[circle,fill=white,minimum size=3cm] at (0,1.5) {\fontsize{50}{50}\selectfont\textcolor{forest}{\faLeaf}};
\node[text=white,font=\Huge\bfseries] at (0,-0.5) {AI SENTINEL};
\node[text=mint,font=\large] at (0,-1.3) {Système de Détection des Incendies par IA};
\draw[white,thick] (-4,-2) -- (4,-2);
\node[text=white!80,font=\small] at (0,-2.8) {Projet de Fin d'Études -- 2024-2025};
\node[text=mint,font=\footnotesize] at (0,-3.5) {[Nom] | Encadrant: [Encadrant]};
\end{tikzpicture}
\end{frame}
}

% ========== SLIDE 2: SOMMAIRE ==========
\begin{frame}{Sommaire}
\tableofcontents
\end{frame}

% ============================================
% PARTIE 1: INTRODUCTION (Slides 3-8)
% ============================================
\section{Introduction et Contexte}

% SLIDE 3
\begin{frame}{Contexte Mondial des Incendies}
\begin{columns}[T]
\begin{column}{0.55\textwidth}
\textbf{Statistiques alarmantes :}
\begin{itemize}
\item \textcolor{fire}{\faFire} 400+ millions ha brûlés/an
\item \textcolor{fire}{\faSkull} 340,000 décès liés à la fumée
\item \textcolor{fire}{\faMoneyBill} 50+ milliards \$ de dégâts
\item \textcolor{fire}{\faThermometerHalf} +30\% depuis 2000
\end{itemize}
\end{column}
\begin{column}{0.42\textwidth}
\begin{tikzpicture}
\begin{axis}[width=5cm,height=4cm,ybar,ylabel=Hectares (M),symbolic x coords={2020,2021,2022,2023},xtick=data,bar width=8pt]
\addplot[fill=fire] coordinates {(2020,350)(2021,420)(2022,480)(2023,410)};
\end{axis}
\end{tikzpicture}
\end{column}
\end{columns}
\end{frame}

% SLIDE 4
\begin{frame}{Situation au Maroc}
\begin{tcolorbox}[colback=fire!10,colframe=fire,title=Données Nationales]
\begin{itemize}
\item 200-400 incendies/an en moyenne
\item 3,000-5,000 hectares détruits annuellement
\item Régions les plus touchées: Rif, Moyen Atlas
\item Période critique: Juin-Septembre
\end{itemize}
\end{tcolorbox}
\end{frame}

% SLIDE 5
\begin{frame}{Problématique}
\begin{center}
\begin{tcolorbox}[colback=warn!10,colframe=warn,width=0.9\textwidth]
\centering\large
\textbf{Comment détecter les incendies de forêt de manière\\précoce, automatique et à grande échelle ?}
\end{tcolorbox}
\end{center}
\vspace{0.5cm}
\textbf{Défis identifiés:}
\begin{enumerate}
\item Surveillance humaine limitée et coûteuse
\item Zones forestières isolées et inaccessibles
\item Temps de réaction critique (premiers 15 min)
\item Faux positifs fréquents
\end{enumerate}
\end{frame}

% SLIDE 6
\begin{frame}{Objectifs du Projet}
\begin{columns}[T]
\begin{column}{0.48\textwidth}
\begin{block}{\faCheck~Objectifs Principaux}
\begin{itemize}
\item Détection temps réel feu/fumée
\item Surveillance satellite automatisée
\item Prédiction de propagation
\item Alertes multi-canaux
\end{itemize}
\end{block}
\end{column}
\begin{column}{0.48\textwidth}
\begin{block}{\faCogs~Objectifs Techniques}
\begin{itemize}
\item Précision > 90\%
\item Latence < 500ms
\item Interface web moderne
\item API REST documentée
\end{itemize}
\end{block}
\end{column}
\end{columns}
\end{frame}

% SLIDE 7
\begin{frame}{Solution Proposée: AI Sentinel}
\begin{center}
\begin{tikzpicture}[node distance=2cm]
\node[draw,fill=forest,text=white,rounded corners,minimum width=3cm,minimum height=1cm] (center) {\textbf{AI SENTINEL}};
\node[draw,fill=mint,rounded corners,above left=of center] (cam) {\faVideo~Caméras};
\node[draw,fill=mint,rounded corners,above right=of center] (sat) {\faSatellite~Satellites};
\node[draw,fill=mint,rounded corners,below=of center] (web) {\faLaptop~Interface Web};
\draw[->,thick] (cam) -- (center);
\draw[->,thick] (sat) -- (center);
\draw[->,thick] (center) -- (web);
\end{tikzpicture}
\end{center}
\end{frame}

% SLIDE 8
\begin{frame}{Méthodologie de Travail}
\begin{enumerate}
\item \textbf{Analyse:} Étude des besoins et état de l'art
\item \textbf{Conception:} Architecture et choix technologiques
\item \textbf{Développement:} Backend, Frontend, Modèles IA
\item \textbf{Tests:} Validation et benchmarks
\item \textbf{Déploiement:} Mise en production
\end{enumerate}
\end{frame}

% ============================================
% PARTIE 2: ÉTAT DE L'ART (Slides 9-12)
% ============================================
\section{État de l'Art}

% SLIDE 9
\begin{frame}{Méthodes Traditionnelles}
\begin{columns}[T]
\begin{column}{0.48\textwidth}
\textbf{Tours de Guet}
\begin{itemize}
\item[\textcolor{forest}{\faCheck}] Couverture visuelle
\item[\textcolor{fire}{\faTimes}] Coût élevé
\item[\textcolor{fire}{\faTimes}] Fatigue humaine
\item[\textcolor{fire}{\faTimes}] Couverture limitée
\end{itemize}
\end{column}
\begin{column}{0.48\textwidth}
\textbf{Patrouilles Terrestres}
\begin{itemize}
\item[\textcolor{forest}{\faCheck}] Intervention rapide
\item[\textcolor{fire}{\faTimes}] Zones inaccessibles
\item[\textcolor{fire}{\faTimes}] Ressources humaines
\item[\textcolor{fire}{\faTimes}] Non continu
\end{itemize}
\end{column}
\end{columns}
\end{frame}

% SLIDE 10
\begin{frame}{Solutions IA Existantes}
\begin{table}
\centering\small
\begin{tabular}{lccc}
\toprule
\textbf{Système} & \textbf{Technologie} & \textbf{Précision} & \textbf{Temps Réel} \\
\midrule
FireWatch & CNN + Capteurs & 85\% & Non \\
ALERT Wildfire & Caméras PTZ & 78\% & Oui \\
Pano AI & Vision 360° & 92\% & Oui \\
\textbf{AI Sentinel} & \textbf{YOLO + Satellite} & \textbf{91.6\%} & \textbf{Oui} \\
\bottomrule
\end{tabular}
\end{table}
\end{frame}

% SLIDE 11
\begin{frame}{Apport de Notre Solution}
\begin{tcolorbox}[colback=forest!10,colframe=forest,title=\faLightbulb~Innovation]
\begin{itemize}
\item \textbf{Hybride:} Caméras locales + Satellites
\item \textbf{Multi-modèles:} YOLO + MobileNet + CAM
\item \textbf{Prédiction:} Algorithme de propagation
\item \textbf{Open Source:} Solution accessible
\end{itemize}
\end{tcolorbox}
\end{frame}

% SLIDE 12
\begin{frame}{Technologies Clés Utilisées}
\begin{columns}[T]
\begin{column}{0.32\textwidth}
\textbf{\faBrain~IA/ML}
\begin{itemize}
\item YOLOv8
\item TensorFlow
\item OpenCV
\item NumPy
\end{itemize}
\end{column}
\begin{column}{0.32\textwidth}
\textbf{\faServer~Backend}
\begin{itemize}
\item Python 3.10
\item FastAPI
\item Pydantic
\item Uvicorn
\end{itemize}
\end{column}
\begin{column}{0.32\textwidth}
\textbf{\faLaptopCode~Frontend}
\begin{itemize}
\item React 19
\item TypeScript
\item TailwindCSS
\item Leaflet
\end{itemize}
\end{column}
\end{columns}
\end{frame}

% ============================================
% PARTIE 3: ARCHITECTURE (Slides 13-18)
% ============================================
\section{Architecture du Système}

% SLIDE 13
\begin{frame}{Architecture Globale}
\begin{center}
\begin{tikzpicture}[scale=0.8,transform shape,node distance=1.2cm]
\node[draw,fill=tech!20,minimum width=10cm,minimum height=1cm] (front) {Frontend (React + TypeScript)};
\node[draw,fill=forest!20,minimum width=10cm,minimum height=1cm,below=of front] (back) {Backend (FastAPI + Services)};
\node[draw,fill=warn!20,minimum width=10cm,minimum height=1cm,below=of back] (ia) {Couche IA (YOLO + MobileNet + CAM)};
\node[draw,fill=slate!20,minimum width=10cm,minimum height=1cm,below=of ia] (ext) {APIs Externes (NASA FIRMS, Sentinel Hub)};
\draw[->,thick] (front) -- (back);
\draw[->,thick] (back) -- (ia);
\draw[->,thick] (back) -- (ext);
\end{tikzpicture}
\end{center}
\end{frame}

% SLIDE 14
\begin{frame}{Architecture Backend}
\textbf{Pattern: Modular Monolith + API-First}
\begin{itemize}
\item \texttt{/api/} -- Routes API REST
\item \texttt{/services/} -- Logique métier
\item \texttt{/models/} -- Modèles Pydantic
\item \texttt{/ai/} -- Moteurs d'inférence
\end{itemize}
\end{frame}

% SLIDE 15
\begin{frame}{Endpoints API Principaux}
\begin{table}
\centering\small
\begin{tabular}{lll}
\toprule
\textbf{Méthode} & \textbf{Endpoint} & \textbf{Description} \\
\midrule
POST & /api/classify & Classification image \\
GET & /api/video/start & Démarrer détection \\
GET & /api/firms/hotspots & Récupérer hotspots \\
POST & /api/satellite/analyze & Analyse satellite \\
POST & /api/predict & Prédiction propagation \\
\bottomrule
\end{tabular}
\end{table}
\end{frame}

% SLIDE 16
\begin{frame}{Architecture Frontend}
\textbf{Stack: React 19 + Vite + TailwindCSS}
\begin{itemize}
\item Pages: Dashboard, Detection, Upload, Map, Satellite
\item Composants réutilisables
\item State management avec hooks
\item Animations avec Framer Motion
\end{itemize}
\end{frame}

% SLIDE 17
\begin{frame}{Flux de Données: Détection Temps Réel}
\begin{center}
\begin{tikzpicture}[node distance=1.5cm]
\node[draw,rounded corners,fill=mint] (cam) {Webcam};
\node[draw,rounded corners,fill=mint,right=of cam] (frame) {Frame};
\node[draw,rounded corners,fill=warn!30,right=of frame] (yolo) {YOLO};
\node[draw,rounded corners,fill=fire!30,right=of yolo] (alert) {Alerte};
\draw[->,thick] (cam) -- (frame);
\draw[->,thick] (frame) -- (yolo);
\draw[->,thick] (yolo) -- node[above]{\tiny >0.5} (alert);
\end{tikzpicture}
\end{center}
\end{frame}

% SLIDE 18
\begin{frame}{Flux de Données: Surveillance Satellite}
\begin{enumerate}
\item Requête coordonnées GPS
\item Appel API Sentinel Hub
\item Téléchargement image SWIR
\item Analyse par modèle CAM
\item Génération heatmap
\item Affichage résultat
\end{enumerate}
\end{frame}

% ============================================
% PARTIE 4: MODÈLES IA (Slides 19-28)
% ============================================
\section{Intelligence Artificielle}

% SLIDE 19
\begin{frame}{Vue d'Ensemble des Modèles}
\begin{table}
\centering
\begin{tabular}{lccc}
\toprule
\textbf{Modèle} & \textbf{Tâche} & \textbf{Input} & \textbf{Output} \\
\midrule
YOLOv8 & Détection & Vidéo & Bboxes \\
MobileNetV2 & Classification & Image & Classe \\
CAM CNN & Analyse Sat & Sentinel-2 & Heatmap \\
\bottomrule
\end{tabular}
\end{table}
\end{frame}

% SLIDE 20
\begin{frame}{YOLOv8: Introduction}
\textbf{You Only Look Once v8 (Ultralytics)}
\begin{itemize}
\item Architecture anchor-free
\item Détection en une seule passe
\item Optimisé pour edge devices
\item Version Nano pour temps réel
\end{itemize}
\end{frame}

% SLIDE 21
\begin{frame}{YOLOv8: Architecture}
\begin{columns}
\begin{column}{0.5\textwidth}
\textbf{Composants:}
\begin{enumerate}
\item \textbf{Backbone:} CSPDarknet\\Extraction features
\item \textbf{Neck:} PANet\\Fusion multi-échelle
\item \textbf{Head:} Decoupled\\Classification + Bbox
\end{enumerate}
\end{column}
\begin{column}{0.45\textwidth}
\begin{tikzpicture}[scale=0.5]
\node[draw,fill=tech!20,minimum height=4cm,minimum width=1.5cm] (b) {Backbone};
\node[draw,fill=forest!20,minimum height=3cm,minimum width=1.5cm,right=0.5cm of b] (n) {Neck};
\node[draw,fill=warn!20,minimum height=2cm,minimum width=1.5cm,right=0.5cm of n] (h) {Head};
\draw[->,thick] (b) -- (n);
\draw[->,thick] (n) -- (h);
\end{tikzpicture}
\end{column}
\end{columns}
\end{frame}

% SLIDE 22
\begin{frame}{YOLOv8: Dataset et Entraînement}
\begin{columns}[T]
\begin{column}{0.48\textwidth}
\textbf{Dataset:}
\begin{itemize}
\item 10,000+ images
\item 2 classes: Fire, Smoke
\item Annotations YOLO format
\item Augmentation: flip, rotate
\end{itemize}
\end{column}
\begin{column}{0.48\textwidth}
\textbf{Hyperparamètres:}
\begin{itemize}
\item Epochs: 100
\item Batch size: 16
\item Learning rate: 0.01
\item Optimizer: SGD
\end{itemize}
\end{column}
\end{columns}
\end{frame}

% SLIDE 23
\begin{frame}{YOLOv8: Métriques de Performance}
\begin{table}
\centering
\begin{tabular}{lccc}
\toprule
\textbf{Classe} & \textbf{Precision} & \textbf{Recall} & \textbf{mAP@50} \\
\midrule
Fire & 95.8\% & 92.3\% & 94.1\% \\
Smoke & 91.5\% & 87.4\% & 89.2\% \\
\midrule
\textbf{Global} & \textbf{93.6\%} & \textbf{89.8\%} & \textbf{91.6\%} \\
\bottomrule
\end{tabular}
\end{table}
\end{frame}

% SLIDE 24
\begin{frame}{MobileNetV2: Introduction}
\textbf{Architecture légère pour classification}
\begin{itemize}
\item Développé par Google (2018)
\item Inverted Residual Blocks
\item Depthwise Separable Convolutions
\item Idéal pour transfer learning
\end{itemize}
\end{frame}

% SLIDE 25
\begin{frame}{MobileNetV2: Transfer Learning}
\textbf{Stratégie en 2 phases:}
\begin{enumerate}
\item \textbf{Feature Extraction (10 epochs)}
\begin{itemize}
\item Base ImageNet gelée
\item Nouvelle tête: 3 classes
\end{itemize}
\item \textbf{Fine-Tuning (5 epochs)}
\begin{itemize}
\item Dégel 20 dernières couches
\item Learning rate réduit (1e-5)
\end{itemize}
\end{enumerate}
\end{frame}

% SLIDE 26
\begin{frame}{MobileNetV2: Résultats}
\begin{columns}
\begin{column}{0.48\textwidth}
\textbf{Métriques:}
\begin{itemize}
\item Accuracy: 97.8\%
\item F1-Score: 97.6\%
\item Temps inférence: 320ms
\end{itemize}
\end{column}
\begin{column}{0.48\textwidth}
\textbf{Matrice Confusion:}
\begin{tabular}{l|ccc}
& F & S & N \\
\hline
F & 3510 & 52 & 38 \\
S & 45 & 3420 & 35 \\
N & 28 & 35 & 3637 \\
\end{tabular}
\end{column}
\end{columns}
\end{frame}

% SLIDE 27
\begin{frame}{Modèle CAM: Analyse Satellite}
\textbf{Class Activation Mapping}
\begin{itemize}
\item Interprétabilité des CNN
\item Localise zones activées
\item Génère heatmap
\item Input: Images Sentinel-2 (SWIR)
\end{itemize}
\end{frame}

% SLIDE 28
\begin{frame}{Algorithme de Prédiction}
\textbf{Modèle Rothermel Simplifié:}
$$ R = R_0 \times (1 + \phi_W + \phi_S) $$
\begin{itemize}
\item $R_0$: Vitesse base (végétation)
\item $\phi_W$: Facteur vent (Open-Meteo)
\item $\phi_S$: Facteur pente
\end{itemize}
\end{frame}

% ============================================
% PARTIE 5: BENCHMARK (Slides 29-33)
% ============================================
\section{Benchmark et Comparaisons}

% SLIDE 29
\begin{frame}{Benchmark: Modèles de Détection}
\begin{table}
\centering\small
\begin{tabular}{lcccc}
\toprule
\textbf{Modèle} & \textbf{mAP} & \textbf{FPS} & \textbf{Taille} & \textbf{GPU} \\
\midrule
YOLOv5n & 88.2\% & 52 & 3.8MB & T4 \\
YOLOv7-tiny & 86.5\% & 48 & 6.0MB & T4 \\
\rowcolor{forest!20} YOLOv8n & 91.6\% & 45 & 6.2MB & T4 \\
Faster RCNN & 89.1\% & 12 & 108MB & T4 \\
SSD300 & 82.4\% & 35 & 25MB & T4 \\
\bottomrule
\end{tabular}
\end{table}
\textbf{Conclusion:} YOLOv8n offre le meilleur équilibre précision/vitesse.
\end{frame}

% SLIDE 30
\begin{frame}{Benchmark: Modèles de Classification}
\begin{table}
\centering\small
\begin{tabular}{lccc}
\toprule
\textbf{Modèle} & \textbf{Accuracy} & \textbf{Params} & \textbf{Latence} \\
\midrule
ResNet50 & 96.2\% & 25.6M & 450ms \\
VGG16 & 94.8\% & 138M & 680ms \\
EfficientNet-B0 & 97.1\% & 5.3M & 380ms \\
\rowcolor{forest!20} MobileNetV2 & 97.8\% & 3.4M & 320ms \\
\bottomrule
\end{tabular}
\end{table}
\textbf{Conclusion:} MobileNetV2 optimal pour déploiement web.
\end{frame}

% SLIDE 31
\begin{frame}{Benchmark: Temps de Réponse API}
\begin{table}
\centering
\begin{tabular}{lcc}
\toprule
\textbf{Endpoint} & \textbf{Moyenne} & \textbf{P95} \\
\midrule
/api/classify & 320ms & 450ms \\
/api/video/frame & 45ms & 60ms \\
/api/firms/hotspots & 1.2s & 2.0s \\
/api/satellite/analyze & 2.5s & 4.0s \\
/api/predict & 85ms & 120ms \\
\bottomrule
\end{tabular}
\end{table}
\end{frame}

% SLIDE 32
\begin{frame}{Benchmark: Comparaison Solutions Existantes}
\begin{table}
\centering\small
\begin{tabular}{lcccc}
\toprule
\textbf{Critère} & \textbf{FireWatch} & \textbf{Pano AI} & \textbf{ALERT} & \textbf{Ours} \\
\midrule
Précision & 85\% & 92\% & 78\% & \textbf{91.6\%} \\
Temps Réel & \textcolor{fire}{\faTimes} & \textcolor{forest}{\faCheck} & \textcolor{forest}{\faCheck} & \textcolor{forest}{\faCheck} \\
Satellite & \textcolor{fire}{\faTimes} & \textcolor{fire}{\faTimes} & \textcolor{fire}{\faTimes} & \textcolor{forest}{\faCheck} \\
Prédiction & \textcolor{fire}{\faTimes} & \textcolor{fire}{\faTimes} & \textcolor{fire}{\faTimes} & \textcolor{forest}{\faCheck} \\
Open Source & \textcolor{fire}{\faTimes} & \textcolor{fire}{\faTimes} & \textcolor{fire}{\faTimes} & \textcolor{forest}{\faCheck} \\
\bottomrule
\end{tabular}
\end{table}
\end{frame}

% SLIDE 33
\begin{frame}{Synthèse Benchmark}
\begin{tcolorbox}[colback=forest!10,colframe=forest]
\textbf{Points Forts AI Sentinel:}
\begin{enumerate}
\item Précision compétitive (91.6\% mAP)
\item Temps réel vérifié (45ms/frame)
\item Unique: Détection + Satellite + Prédiction
\item Solution complète et open source
\end{enumerate}
\end{tcolorbox}
\end{frame}

% ============================================
% PARTIE 6: DÉMONSTRATION (Slides 34-38)
% ============================================
\section{Démonstration}

% SLIDE 34
\begin{frame}{Interface: Page d'Accueil}
\begin{center}
\textit{[Capture d'écran Landing Page]}
\begin{itemize}
\item Design moderne avec carte interactive
\item Statistiques en temps réel
\item Accès rapide aux fonctionnalités
\end{itemize}
\end{center}
\end{frame}

% SLIDE 35
\begin{frame}{Interface: Dashboard}
\begin{center}
\textit{[Capture d'écran Dashboard]}
\begin{itemize}
\item Vue d'ensemble des détections
\item Graphiques de performance
\item Liste des alertes récentes
\end{itemize}
\end{center}
\end{frame}

% SLIDE 36
\begin{frame}{Interface: Détection Temps Réel}
\begin{center}
\textit{[Capture d'écran Real-Time Detection]}
\begin{itemize}
\item Flux vidéo avec bounding boxes
\item Confiance affichée en temps réel
\item Alertes visuelles immédiates
\end{itemize}
\end{center}
\end{frame}

% SLIDE 37
\begin{frame}{Interface: Carte Hotspots NASA}
\begin{center}
\textit{[Capture d'écran Hotspot Map]}
\begin{itemize}
\item Données NASA FIRMS
\item Marqueurs colorés par intensité
\item Filtrage par région/date
\end{itemize}
\end{center}
\end{frame}

% SLIDE 38
\begin{frame}{Interface: Surveillance Satellite}
\begin{center}
\textit{[Capture d'écran Satellite Monitoring]}
\begin{itemize}
\item Sélection zone sur carte
\item Image Sentinel-2 récupérée
\item Analyse CAM avec heatmap
\end{itemize}
\end{center}
\end{frame}

% ============================================
% PARTIE 7: TESTS (Slides 39-41)
% ============================================
\section{Tests et Validation}

% SLIDE 39
\begin{frame}{Scénarios de Test}
\begin{table}
\centering\small
\begin{tabular}{lcc}
\toprule
\textbf{Scénario} & \textbf{Résultat} & \textbf{Statut} \\
\midrule
Détection feu réel & Confiance 94\% & \textcolor{forest}{\faCheck} \\
Détection fumée & Confiance 87\% & \textcolor{forest}{\faCheck} \\
Faux positif (coucher soleil) & Rejeté & \textcolor{forest}{\faCheck} \\
Alerte Telegram & Reçue en 850ms & \textcolor{forest}{\faCheck} \\
Analyse satellite & Heatmap correcte & \textcolor{forest}{\faCheck} \\
\bottomrule
\end{tabular}
\end{table}
\end{frame}

% SLIDE 40
\begin{frame}{Tests de Charge}
\begin{itemize}
\item \textbf{Concurrent users:} 50 utilisateurs simultanés
\item \textbf{Requêtes/sec:} 100 req/s sans dégradation
\item \textbf{Uptime:} 99.9\% sur 7 jours de test
\item \textbf{Mémoire:} ~2GB RAM (backend + modèles)
\end{itemize}
\end{frame}

% SLIDE 41
\begin{frame}{Validation Utilisateurs}
\textbf{Feedback recueilli:}
\begin{itemize}
\item Interface intuitive et moderne
\item Temps de réponse satisfaisant
\item Précision des détections appréciée
\item Demande: historique des alertes
\end{itemize}
\end{frame}

% ============================================
% PARTIE 8: CONCLUSION (Slides 42-45)
% ============================================
\section{Conclusion et Perspectives}

% SLIDE 42
\begin{frame}{Objectifs Atteints}
\begin{table}
\centering
\begin{tabular}{ll}
\toprule
\textbf{Objectif} & \textbf{Statut} \\
\midrule
Détection temps réel & \textcolor{forest}{\faCheck~91.6\%} \\
Classification images & \textcolor{forest}{\faCheck~97.8\%} \\
Surveillance satellite & \textcolor{forest}{\faCheck~Intégré} \\
Prédiction propagation & \textcolor{forest}{\faCheck~Fonctionnel} \\
Alertes multi-canaux & \textcolor{forest}{\faCheck~Email/Telegram} \\
Interface moderne & \textcolor{forest}{\faCheck~React/Tailwind} \\
\bottomrule
\end{tabular}
\end{table}
\end{frame}

% SLIDE 43
\begin{frame}{Difficultés Rencontrées}
\begin{itemize}
\item Collecte et annotation du dataset
\item Optimisation des modèles pour temps réel
\item Intégration APIs externes (quotas, latence)
\item Gestion des faux positifs
\end{itemize}
\end{frame}

% SLIDE 44
\begin{frame}{Perspectives Futures}
\begin{description}
\item[Court terme:] Authentification, historique, rapports PDF
\item[Moyen terme:] Application mobile, multi-caméras
\item[Long terme:] Déploiement national, drones, partenariats
\end{description}
\end{frame}

% SLIDE 45
\begin{frame}{Compétences Acquises}
\begin{columns}[T]
\begin{column}{0.48\textwidth}
\textbf{Techniques:}
\begin{itemize}
\item Deep Learning (YOLO, CNN)
\item Développement Full-Stack
\item APIs REST
\item Télédétection
\end{itemize}
\end{column}
\begin{column}{0.48\textwidth}
\textbf{Transversales:}
\begin{itemize}
\item Gestion de projet
\item Rédaction technique
\item Présentation orale
\item Travail autonome
\end{itemize}
\end{column}
\end{columns}
\end{frame}

% ========== SLIDE FINALE ==========
{
\setbeamertemplate{footline}{}
\begin{frame}[plain]
\begin{center}
\vspace{1cm}
\fontsize{50}{50}\selectfont\textcolor{forest}{\faLeaf}
\vspace{0.5cm}

\Huge\textbf{Merci de votre attention}
\vspace{1cm}

\begin{tcolorbox}[colback=mint,colframe=forest,width=0.5\textwidth]
\centering\Large\textbf{Questions ?}
\end{tcolorbox}
\vspace{1cm}

\footnotesize
\faGithub~github.com/WildFireDetection \quad|\quad \faEnvelope~contact@aisentinel.com
\end{center}
\end{frame}
}

\end{document}
