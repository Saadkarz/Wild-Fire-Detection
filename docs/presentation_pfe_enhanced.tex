% ============================================
% PRÉSENTATION PFE AMÉLIORÉE - AI SENTINEL
% ============================================
\documentclass[aspectratio=169, 10pt]{beamer}

% ============================================
% PACKAGES ET CONFIGURATION
% ============================================
\usetheme{metropolis}
\usepackage[utf8]{inputenc}
\usepackage[T1]{fontenc}
\usepackage{lmodern}
\usepackage[french]{babel}
\usepackage{graphicx}
\usepackage{tikz}
\usepackage{pgfplots}
\pgfplotsset{compat=1.17}
\usetikzlibrary{shapes.geometric, arrows, positioning, calc, shadows, decorations.pathreplacing, backgrounds, shapes.arrows, matrix, fit}
\usepackage{fontawesome5}
\usepackage{booktabs}
\usepackage{colortbl}
\usepackage{hyperref}
\usepackage{xcolor}
\usepackage{listings}
\usepackage{animate}
\usepackage{tcolorbox}

% ============================================
% PALETTE DE COULEURS AVANCÉE
% ============================================
% Verts - Nature & Croissance
\definecolor{forestGreen}{HTML}{2E7D32}
\definecolor{emerald}{HTML}{00695C}
\definecolor{lime}{HTML}{CDDC39}
\definecolor{mint}{HTML}{E0F2F1}

% Rouges/Oranges - Feu & Alerte
\definecolor{fireRed}{HTML}{D32F2F}
\definecolor{ember}{HTML}{FF6F00}
\definecolor{warning}{HTML}{FBC02D}

% Bleus/Gris - Tech & Structure
\definecolor{techBlue}{HTML}{0277BD}
\definecolor{slate}{HTML}{37474F}
\definecolor{silver}{HTML}{CFD8DC}

% Configuration Metropolis
\setbeamercolor{background canvas}{bg=white}
\setbeamercolor{normal text}{fg=slate}
\setbeamercolor{alerted text}{fg=fireRed}
\setbeamercolor{example text}{fg=forestGreen}
\setbeamercolor{frametitle}{fg=white, bg=forestGreen}
\setbeamercolor{title separator}{fg=ember}
\setbeamercolor{progress bar}{fg=ember, bg=mint}

% ============================================
% STYLES TIKZ PERSONNALISÉS
% ============================================
\tikzset{
    block/.style={
        draw=forestGreen, 
        fill=mint, 
        text=slate, 
        rectangle, 
        rounded corners=4pt, 
        minimum height=2em, 
        minimum width=4em, 
        font=\small,
        align=center,
        drop shadow
    },
    decision/.style={
        draw=ember, 
        fill=warning!20, 
        diamond, 
        aspect=2, 
        font=\small,
        align=center
    },
    cloud/.style={
        draw=techBlue, 
        fill=techBlue!10, 
        ellipse, 
        minimum height=2em, 
        font=\small,
        align=center
    },
    line/.style={
        draw=slate, 
        -latex', 
        thick
    }
}

% ============================================
% METADATA
% ============================================
\title{\textbf{\faLeaf~AI SENTINEL}}
\subtitle{Système Intelligent de Détection et Prévention des Incendies de Forêt}
\author{\textbf{Présenté par :} [Nom de l'Étudiant]}
\institute{
    \textbf{Université / École}\\
    Département d'Informatique
}
\date{Année Universitaire 2024-2025}

% ============================================
% DOCUMENT
% ============================================
\begin{document}

% --------------------------------------------
% PAGE DE TITRE
% --------------------------------------------
{
\setbeamertemplate{footline}{}
\begin{frame}[plain]
    \begin{tikzpicture}[remember picture, overlay]
        % Fond avec dégradé sophistiqué
        \shade[top color=forestGreen!90!black, bottom color=emerald!80!black] 
            (current page.north west) rectangle (current page.south east);
        
        % Éléments graphiques de fond
        \begin{scope}[opacity=0.1]
            \fill[white] (-2, -5) -- (0, -2) -- (2, -5) -- cycle;
            \fill[white] (-6, -6) -- (-4, -3) -- (-2, -6) -- cycle;
            \fill[white] (2, -5.5) -- (4, -2.5) -- (6, -5.5) -- cycle;
            \node at (6, 2) {\fontsize{120}{120}\selectfont\faSatellite};
            \node at (-6, 2) {\fontsize{100}{100}\selectfont\faBrain};
        \end{scope}
        
        % Logo Central
        \node[
            circle, 
            fill=white, 
            minimum size=3.5cm,
            drop shadow={opacity=0.5, shadow xshift=2pt, shadow yshift=-2pt}
        ] at (0, 1.5) {
            \fontsize{60}{60}\selectfont\textcolor{forestGreen}{\faLeaf}
        };
        
        % Titre et informations
        \node[text=white, font=\bfseries\Huge] at (0, -1) {AI SENTINEL};
        \node[text=warning, font=\large] at (0, -1.8) {Intelligence Artificielle pour la Protection des Forêts};
        
        \draw[white, line width=1pt] (-4, -2.5) -- (4, -2.5);
        
        \node[text=mint, font=\small] at (0, -3.2) {
            Projet de Fin d'Études \textbullet~ 2024-2025
        };
        
        \node[text=white!80, font=\footnotesize] at (0, -4) {
             [Nom de l'Étudiant] \quad|\quad Encadrant : [Nom de l'Encadrant]
        };
    \end{tikzpicture}
\end{frame}
}

% --------------------------------------------
% PLAN
% --------------------------------------------
\begin{frame}{Sommaire}
    \setbeamertemplate{section in toc}[sections numbered]
    \tableofcontents[hideallsubsections]
\end{frame}

% ============================================
% 1. INTRODUCTION & CONTEXTE
% ============================================
\section{Introduction et Contexte}

\begin{frame}{Problématique : L'Urgence Incendie}
    \begin{columns}[T]
        \begin{column}{0.6\textwidth}
            \begin{alertblock}{\faExclamationTriangle~Le Constat}
                Les méthodes traditionnelles de détection (tours de guet, patrouilles) montrent leurs limites face à l'augmentation des incendies.
            \end{alertblock}
            
            \vspace{0.4cm}
            
            \textbf{Défis majeurs :}
            \begin{itemize}
                \item[\textcolor{fireRed}{\faClock}] \textbf{Latence :} Détection souvent trop tardive
                \item[\textcolor{fireRed}{\faEyeSlash}] \textbf{Couverture :} Zones inaccessibles non surveillées
                \item[\textcolor{fireRed}{\faTimesCircle}] \textbf{Faux Positifs :} Erreurs humaines fréquentes
                \item[\textcolor{fireRed}{\faMoneyBillWave}] \textbf{Coût :} Surveillance 24/7 onéreuse
            \end{itemize}
        \end{column}
        \begin{column}{0.38\textwidth}
            % Graphique d'impact
             \begin{tikzpicture}
                \begin{axis}[
                    width=5.5cm, height=5cm,
                    ybar, bar width=10pt,
                    ylabel={Hectares détruits (k)},
                    title={\small Impact Annuel},
                    symbolic x coords={2020, 2021, 2022, 2023},
                    xtick=data,
                    nodes near coords,
                    nodes near coords style={font=\tiny},
                    every axis plot/.append style={fill=fireRed}
                ]
                \addplot coordinates {(2020,35) (2021,42) (2022,55) (2023,48)};
                \end{axis}
            \end{tikzpicture}
        \end{column}
    \end{columns}
\end{frame}

\begin{frame}{Notre Solution : Approche Hybride}
    \begin{center}
    \begin{tikzpicture}[node distance=2.5cm, auto]
        % Nodes
        \node [block, fill=forestGreen, text=white] (center) {AI SENTINEL};
        
        \node [block, above left=of center] (cam) {\faVideo\\Caméras\\Locales};
        \node [block, above right=of center] (sat) {\faSatellite\\Satellites\\Globaux};
        \node [block, below=of center, text width=4cm] (platform) {\faLaptopCode\\Plateforme Unifiée\\(Web/Mobile)};
        
        % Arrows
        \path [line, dashed] (cam) -- node [midway, left, font=\tiny] {Temps Réel} (center);
        \path [line, dashed] (sat) -- node [midway, right, font=\tiny] {Large Échelle} (center);
        \path [line, ultra thick] (center) -- (platform);
        
        % AI Models
        \node [cloud, left=0.5cm of center] (yolo) {YOLOv8};
        \node [cloud, right=0.5cm of center] (cam_model) {CAM Model};
        
        \draw [line, dotted] (yolo) -- (center);
        \draw [line, dotted] (cam_model) -- (center);
    \end{tikzpicture}
    \end{center}
\end{frame}

% ============================================
% 2. ARCHITECTURE TECHNIQUE
% ============================================
\section{Architecture Technique}

\begin{frame}[fragile]{Architecture Globale Microservices}
    \begin{center}
    \resizebox{0.95\textwidth}{!}{
    \begin{tikzpicture}[node distance=1.5cm]
        % Frontend Layer
        \node[font=\bfseries\color{forestGreen}] at (-5, 1) {FRONTEND};
        \node[block, fill=white, draw=techBlue] (react) at (-5, 0) {\faReact\\React App};
        
        % API Gateway / Backend
        \node[font=\bfseries\color{forestGreen}] at (0, 2) {BACKEND (FastAPI)};
        \node[block, fill=mint, minimum width=8cm, minimum height=3cm] (backend) at (0, 0) {};
        
        \node[block, fill=white] (api) at (-2, 0.5) {API Router};
        \node[block, fill=white] (auth) at (-2, -0.5) {Auth Service};
        \node[block, fill=white] (ia_service) at (2, 0.5) {IA Service};
        \node[block, fill=white] (notif) at (2, -0.5) {Notification};
        
        % AI Engines
        \node[font=\bfseries\color{forestGreen}] at (0, -2.5) {MOTEURS IA};
        \node[block, fill=warning!20] (yolo) at (-2, -3.5) {\textbf{YOLOv8}\\Dét. Objet};
        \node[block, fill=warning!20] (mobilenet) at (0, -3.5) {\textbf{MobileNet}\\Classif.};
        \node[block, fill=warning!20] (sat_mod) at (2, -3.5) {\textbf{CAM}\\Satellite};
        
        % External
        \node[cloud] (nasa) at (5, 1) {NASA FIRMS};
        \node[cloud] (sentinel) at (5, -1) {Sentinel Hub};
        
        % Connections
        \draw[<->, thick] (react) -- (backend);
        \draw[->] (ia_service) -- (yolo);
        \draw[->] (ia_service) -- (mobilenet);
        \draw[->] (ia_service) -- (sat_mod);
        \draw[<->, dashed] (backend) -- (nasa);
        \draw[<->, dashed] (backend) -- (sentinel);
        
    \end{tikzpicture}
    }
    \end{center}
\end{frame}

% ============================================
% 3. INTELLIGENCE ARTIFICIELLE (DÉTAILLÉ)
% ============================================
\section{Intelligence Artificielle : Au Cœur du Système}

\begin{frame}{Vue d'Ensemble des Modèles}
    \begin{columns}[T]
        \begin{column}{0.32\textwidth}
            \begin{tcolorbox}[title=\faVideo~Détection Vidéo, colback=white, colframe=forestGreen]
                \textbf{Modèle :} YOLOv8 Nano\\
                \textbf{Rôle :} Détection Objets\\
                \textbf{Cible :} Feu, Fumée\\
                \textbf{Vitesse :} Temps Réel
            \end{tcolorbox}
        \end{column}
        \begin{column}{0.32\textwidth}
            \begin{tcolorbox}[title=\faImage~Classification, colback=white, colframe=techBlue]
                \textbf{Modèle :} MobileNetV2\\
                \textbf{Rôle :} Vérification\\
                \textbf{Classes :} Fire, Smoke, Safe\\
                \textbf{Précision :} Haute
            \end{tcolorbox}
        \end{column}
        \begin{column}{0.32\textwidth}
            \begin{tcolorbox}[title=\faSatellite~Analyse Sat, colback=white, colframe=ember]
                \textbf{Modèle :} CAM CNN\\
                \textbf{Rôle :} Zones à risque\\
                \textbf{Input :} Images Sentinel-2\\
                \textbf{Sortie :} Heatmap
            \end{tcolorbox}
        \end{column}
    \end{columns}
\end{frame}

% --- YOLO SECTION ---
\begin{frame}{1. YOLOv8 : Architecture Profonde}
    \begin{columns}
        \begin{column}{0.5\textwidth}
            \textbf{Pourquoi YOLOv8 ?}
            \begin{itemize}
                \item Architecture \textit{Anchor-Free} (plus rapide, plus précis pour petits objets).
                \item \textbf{Backbone :} CSPDarknet53 modifié (extraction caractéristiques).
                \item \textbf{Neck :} PANet (fusion multi-échelles).
                \item \textbf{Head :} Découplée (Classification + Bbox regression séparées).
            \end{itemize}
        \end{column}
        \begin{column}{0.5\textwidth}
            \begin{tikzpicture}[scale=0.6, transform shape]
                % Backbone
                \node[draw, fill=blue!10, minimum width=2cm, minimum height=6cm, label=above:Backbone] (back) {};
                % Neck
                \node[draw, fill=green!10, minimum width=1.5cm, minimum height=4cm, right=1cm of back, label=above:Neck] (neck) {};
                % Head
                \node[draw, fill=red!10, minimum width=1.5cm, minimum height=2cm, right=1cm of neck, label=above:Head] (head) {};
                
                \draw[->, thick] (back) -- (neck);
                \draw[->, thick] (neck) -- (head);
                \node[right=0.5cm of head] {Output};
            \end{tikzpicture}
            
            \vspace{0.2cm}
            \begin{exampleblock}{Métriques Entraînement}
                \begin{itemize}
                    \item \textbf{mAP@50 :} 91.6\%
                    \item \textbf{FPS :} 45 (sur GPU T4)
                    \item \textbf{Taille :} 6.2 MB (Nano)
                \end{itemize}
            \end{exampleblock}
        \end{column}
    \end{columns}
\end{frame}

% --- MOBILENET SECTION ---
\begin{frame}{2. MobileNetV2 : Transfer Learning}
    \textbf{Stratégie d'Entraînement :}
    \begin{enumerate}
        \item \textbf{Base :} Modèle pré-entraîné sur ImageNet (poids gelés).
        \item \textbf{Top Layer :} Remplacement de la tête de classification (1000 classes $\to$ 3 classes).
        \item \textbf{Fine-Tuning :} Dégel des 20 dernières couches pour spécialisation "Feu".
    \end{enumerate}

    \vspace{0.4cm}
    
    \begin{center}
    \begin{tikzpicture}[scale=0.8]
        \node[block, fill=white] (input) {Image Input};
        \node[block, fill=blue!10, right=0.5cm of input, text width=2.5cm] (conv) {Conv Layers\\(Frozen)};
        \node[block, fill=green!10, right=0.5cm of conv, text width=2.5cm] (fine) {To Layers\\(Trainable)};
        \node[block, fill=red!10, right=0.5cm of fine] (out) {Softmax\\(3 Classes)};
        
        \draw[->] (input) -- (conv);
        \draw[->] (conv) -- (fine);
        \draw[->] (fine) -- (out);
    \end{tikzpicture}
    \end{center}
    
    \vspace{0.2cm}
    \centering \textbf{Précision Finale :} 97.8\%
\end{frame}

% --- CAM SECTION ---
\begin{frame}{3. Modèle CAM et Imagerie Satellite}
    \begin{columns}[T]
        \begin{column}{0.55\textwidth}
            \textbf{Class Activation Mapping (CAM)}
            \begin{itemize}
                \item Permet l'interprétabilité des modèles CNN.
                \item Identifie \textit{quelles} parties de l'image ont déclenché la détection.
                \item Génère une \textbf{Heatmap} superposable.
            \end{itemize}
            
            \vspace{0.3cm}
            
            \textbf{Utilisation Sentinel-2 :}
            \begin{itemize}
                \item Bandes SWIR (B11, B12) très sensibles à la chaleur.
                \item Fusion avec bandes RGB pour visualisation.
            \end{itemize}
        \end{column}
        \begin{column}{0.4\textwidth}
             % Simulation visuelle Heatmap
            \begin{tikzpicture}
                \fill[gray!20] (0,0) rectangle (4,3);
                \node at (2,1.5) {Image Satellite};
                
                % Heatmap simulation
                \fill[red, opacity=0.3] (2,2) circle (0.8cm);
                \fill[yellow, opacity=0.4] (2,2) circle (0.4cm);
                \node[font=\footnotesize, text=red] at (2, 2.9) {Zone Activée};
            \end{tikzpicture}
        \end{column}
    \end{columns}
\end{frame}

% --- PREDICTION SECTION ---
\begin{frame}{4. Algorithme de Prédiction de Propagation}
    \textbf{Modèle Mathématique Simplifié (Rothermel adapté)}
    
    $$ R = R_0 \times (1 + \phi_W + \phi_S) $$
    
    Où :
    \begin{itemize}
        \item $R$ : Vitesse de propagation estimée.
        \item $R_0$ : Vitesse de base (type de végétation).
        \item $\phi_W$ : \textbf{Facteur Vent} (Vitesse, Direction via Open-Meteo API).
        \item $\phi_S$ : \textbf{Facteur Pente} (Terrain, Topographie).
    \end{itemize}
    
    \vspace{0.3cm}
    
    \begin{block}{\faCode~Implémentation}
        Code Python personnalisé intégrant les données météo en temps réel pour calculer le rayon de danger futur (t+1h, t+3h).
    \end{block}
\end{frame}

% ============================================
% 4. DÉMONSTRATION ET RÉSULTATS
% ============================================
\section{Démonstration et Résultats}

\begin{frame}[fragile]{Pipeline de Détection Temps Réel}
    \begin{center}
    \begin{tikzpicture}[node distance=1.5cm, auto]
        \node [block] (cam) {Webcam Stream};
        \node [cloud, right=of cam] (frame) {Frame Extraction};
        \node [decision, right=of frame] (detect) {YOLO Detect?};
        \node [block, below=of detect, fill=fireRed!20] (alert) {ALERTE};
        \node [block, right=of alert, fill=mint] (notif) {Telegram/Mail};
        
        \path [line] (cam) -- (frame);
        \path [line] (frame) -- (detect);
        \path [line] (detect) -- node [near start] {Oui (>0.5)} (alert);
        \path [line] (detect) -- node [near start] {Non} +(0,1.5) -| (frame);
        \path [line] (alert) -- (notif);
        
        \node [right=0.5cm of detect, font=\tiny, text width=2cm] {Traitement: 45ms\\Anti-Spam: 30s};
    \end{tikzpicture}
    \end{center}
\end{frame}

\begin{frame}{Interfaces Utilisateur (React + Leaflet)}
    \begin{columns}[T]
        \begin{column}{0.48\textwidth}
            \textbf{Dashboard de Supervision}
            \begin{itemize}
                \item Carte interactive des Hotspots (NASA).
                \item Statistiques en temps réel.
                \item Liste des dernières alertes.
            \end{itemize}
        \end{column}
        \begin{column}{0.48\textwidth}
            % Placeholder visuel pour screenshot
            \begin{tikzpicture}
                \draw[fill=slate!10, rounded corners] (0,0) rectangle (5,3);
                \node at (2.5, 1.5) {\faDesktop~Interface Web};
                \draw[fill=white] (0.2, 0.2) rectangle (4.8, 2.8);
                \fill[forestGreen] (0.5, 0.5) circle (2pt);
                \fill[fireRed] (3.5, 2.0) circle (2pt);
            \end{tikzpicture}
        \end{column}
    \end{columns}
\end{frame}


% ============================================
% 5. CONCLUSION
% ============================================
\section{Conclusion et Perspectives}

\begin{frame}{Bilan du Projet}
    \begin{table}
        \centering
        \begin{tabular}{l l}
            \toprule
            \textbf{Objectif} & \textbf{Statut} \\
            \midrule
            Détection Feu/Fumée & \textcolor{forestGreen}{\faCheck~Accompli (91.6\%)} \\
            Surveillance Satellite & \textcolor{forestGreen}{\faCheck~Intégré (Sentinel/FIRMS)} \\
            Temps Réel & \textcolor{forestGreen}{\faCheck~Optimisé (45ms)} \\
            Interface Utilisateur & \textcolor{forestGreen}{\faCheck~Moderne & Responsive} \\
            Système d'Alerte & \textcolor{forestGreen}{\faCheck~Multi-canal} \\
            \bottomrule
        \end{tabular}
    \end{table}
    
    \vspace{0.2cm}
    \textbf{Valeur Ajoutée :} Une solution \textbf{tout-en-un} accessible et performante.
\end{frame}

\begin{frame}{Perspectives Futures}
    \begin{description}
        \item[\faMobileAlt~\textbf{Application Mobile}] Version native pour les gardes forestiers sur le terrain.
        \item[\faPlane~\textbf{Drones Autonomes}] Intégration de flottes de drones pour vérification automatique des alertes satellites.
        \item[\faDatabase~\textbf{Big Data}] Analyse historique pour prédire les saisons à haut risque (Machine Learning prédictif).
    \end{description}
\end{frame}

% --------------------------------------------
% SLIDE DE FIN
% --------------------------------------------
{
\setbeamertemplate{footline}{}
\begin{frame}[plain]
    \centering
    \vspace{1cm}
    
    \fontsize{40}{50}\selectfont \textcolor{forestGreen}{\faLeaf}
    
    \vspace{0.5cm}
    
    \textcolor{slate}{\Huge \textbf{Merci de votre attention}}
    
    \vspace{1cm}
    
    \begin{tcolorbox}[colback=mint, colframe=forestGreen, width=0.6\textwidth, align=center]
        \large \textbf{Questions ?}
    \end{tcolorbox}
    
    \vspace{1cm}
    
    \footnotesize
    \faGithub~ github.com/username/WildFireDetection \quad|\quad \faEnvelope~ contact@aisentinel.com
\end{frame}
}

\end{document}
