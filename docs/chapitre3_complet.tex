% ============================================
% CHAPITRE III - ANALYSE ET SPÉCIFICATION DES BESOINS
% Version complète - À inclure dans le document principal
% ============================================
% 
% Ce fichier contient le Chapitre 3 complet.
% Pour l'utiliser, incluez-le dans votre document principal avec :
% % ============================================
% CHAPITRE III - ANALYSE ET SPÉCIFICATION DES BESOINS
% Version complète - À inclure dans le document principal
% ============================================
% 
% Ce fichier contient le Chapitre 3 complet.
% Pour l'utiliser, incluez-le dans votre document principal avec :
% % ============================================
% CHAPITRE III - ANALYSE ET SPÉCIFICATION DES BESOINS
% Version complète - À inclure dans le document principal
% ============================================
% 
% Ce fichier contient le Chapitre 3 complet.
% Pour l'utiliser, incluez-le dans votre document principal avec :
% % ============================================
% CHAPITRE III - ANALYSE ET SPÉCIFICATION DES BESOINS
% Version complète - À inclure dans le document principal
% ============================================
% 
% Ce fichier contient le Chapitre 3 complet.
% Pour l'utiliser, incluez-le dans votre document principal avec :
% \input{chapitre3_complet.tex}
%
% Ou copiez son contenu à la place du chapitre correspondant.
%
% Structure du chapitre :
% - 3.1 Présentation du Projet (Vision, Parties Prenantes)
% - 3.2 Analyse des Besoins Fonctionnels (8 modules)
% - 3.3 Analyse des Besoins Non Fonctionnels (Performance, Fiabilité, etc.)
% - 3.4 Méthodologie de Développement (Agile/Scrum)
% - 3.5 Modélisation UML (Use Cases, Séquences, Classes, etc.)
%
% Les fichiers sources sont divisés en 4 parties pour faciliter la maintenance :
% - chapitre3_partie1.tex : Sections 3.1 et 3.2 (partiellement)
% - chapitre3_partie2.tex : Suite de 3.2 (Satellite, Prévention, Prédiction, Notifications)
% - chapitre3_partie3.tex : Sections 3.3 et 3.4
% - chapitre3_partie4.tex : Section 3.5 (UML)
% ============================================

\input{chapitre3_partie1.tex}
\input{chapitre3_partie2.tex}
\input{chapitre3_partie3.tex}
\input{chapitre3_partie4.tex}

%
% Ou copiez son contenu à la place du chapitre correspondant.
%
% Structure du chapitre :
% - 3.1 Présentation du Projet (Vision, Parties Prenantes)
% - 3.2 Analyse des Besoins Fonctionnels (8 modules)
% - 3.3 Analyse des Besoins Non Fonctionnels (Performance, Fiabilité, etc.)
% - 3.4 Méthodologie de Développement (Agile/Scrum)
% - 3.5 Modélisation UML (Use Cases, Séquences, Classes, etc.)
%
% Les fichiers sources sont divisés en 4 parties pour faciliter la maintenance :
% - chapitre3_partie1.tex : Sections 3.1 et 3.2 (partiellement)
% - chapitre3_partie2.tex : Suite de 3.2 (Satellite, Prévention, Prédiction, Notifications)
% - chapitre3_partie3.tex : Sections 3.3 et 3.4
% - chapitre3_partie4.tex : Section 3.5 (UML)
% ============================================

% ============================================
% CHAPITRE III - ANALYSE ET SPÉCIFICATION DES BESOINS
% ============================================
\chapter{Analyse et Spécification des Besoins}
\thispagestyle{fancy}

\section{Présentation du Projet}

\lettrine[lines=3, lhang=0.15, loversize=0.1, findent=3pt]{\textcolor{primaryGreen}{L}}{a phase} d'analyse et de spécification des besoins constitue une étape fondamentale dans tout projet de développement logiciel. Elle permet de définir avec précision ce que le système doit accomplir, comment il doit se comporter, et quelles contraintes il doit respecter. Dans le cadre du projet \textbf{AI Sentinel}, cette phase revêt une importance particulière compte tenu de la criticité du domaine d'application : la détection précoce des incendies de forêt, où chaque minute gagnée peut sauver des vies et préserver des hectares de forêt.

Ce chapitre présente une analyse exhaustive des besoins fonctionnels et non fonctionnels du système, accompagnée d'une modélisation UML détaillée permettant de visualiser les interactions entre les différents acteurs et composants du système.

\subsection{Vision Globale}

Le projet \textbf{AI Sentinel} s'inscrit dans une vision ambitieuse : créer un écosystème technologique complet capable de détecter, surveiller, prédire et alerter en temps réel sur les risques d'incendies de forêt. Cette vision se concrétise à travers une plateforme web full-stack qui intègre les technologies les plus avancées en matière d'intelligence artificielle et de traitement d'images.

\begin{greenbox}[\faEye\ Vision du Projet AI Sentinel]
Notre ambition va au-delà de la simple détection : nous souhaitons fournir aux autorités et gestionnaires forestiers un \textbf{outil d'aide à la décision} complet qui leur permette d'anticiper les risques, d'optimiser leurs ressources, et d'intervenir de manière proactive plutôt que réactive.

La plateforme AI Sentinel se positionne comme un \textbf{hub centralisé} intégrant :
\begin{itemize}[leftmargin=1cm, itemsep=5pt]
    \item[\textcolor{primaryGreen}{\faVideo}] L'analyse vidéo en temps réel par intelligence artificielle
    \item[\textcolor{primaryGreen}{\faSatellite}] La surveillance satellite à couverture globale
    \item[\textcolor{primaryGreen}{\faChartLine}] Les algorithmes prédictifs de propagation des feux
    \item[\textcolor{primaryGreen}{\faBell}] Un système d'alertes multi-canaux réactif
    \item[\textcolor{primaryGreen}{\faCloudSun}] L'intégration des données météorologiques
\end{itemize}
\end{greenbox}

\vspace{0.5cm}

L'objectif principal est de réduire significativement le \textbf{temps de détection} des incendies, facteur critique dans la limitation des dégâts. Les études montrent qu'une intervention dans les 15 premières minutes suivant le départ d'un feu permet de contenir 90\% des incendies avant qu'ils ne deviennent incontrôlables.

\begin{center}
\begin{tikzpicture}[
    block/.style={rectangle, rounded corners=10pt, draw=primaryGreen, line width=2pt, fill=mintGreen, text width=3cm, minimum height=2cm, align=center, font=\small},
    arrow/.style={->, >=stealth, line width=2pt, color=leafGreen}
]
    % Blocs
    \node[block] (detect) at (0,0) {\textcolor{darkGreen}{\faSearch}\\[5pt]\textbf{Détection}\\Rapide};
    \node[block] (analyse) at (4,0) {\textcolor{darkGreen}{\faBrain}\\[5pt]\textbf{Analyse}\\IA};
    \node[block] (predict) at (8,0) {\textcolor{darkGreen}{\faChartArea}\\[5pt]\textbf{Prédiction}\\Propagation};
    \node[block] (alert) at (12,0) {\textcolor{darkGreen}{\faBell}\\[5pt]\textbf{Alerte}\\Instantanée};
    
    % Flèches
    \draw[arrow] (detect) -- (analyse);
    \draw[arrow] (analyse) -- (predict);
    \draw[arrow] (predict) -- (alert);
    
    % Temps
    \node[font=\footnotesize, text=textGray] at (2, -1.5) {< 500ms};
    \node[font=\footnotesize, text=textGray] at (6, -1.5) {< 1s};
    \node[font=\footnotesize, text=textGray] at (10, -1.5) {< 2s};
    
    % Titre
    \node[font=\bfseries\large, text=darkGreen] at (6, 2) {Chaîne de Traitement AI Sentinel};
\end{tikzpicture}
\end{center}

\subsection{Parties Prenantes}

L'identification et la compréhension des parties prenantes sont essentielles pour garantir que le système réponde aux besoins réels de ses utilisateurs. Le projet AI Sentinel implique plusieurs catégories d'acteurs, chacune ayant des attentes et des besoins spécifiques.

\subsubsection{Utilisateurs Finaux}

Les utilisateurs finaux représentent le cœur de notre cible. Ils constituent les bénéficiaires directs du système et leurs besoins orientent l'ensemble des choix de conception.

\begin{infobox}{Profils Utilisateurs Identifiés}

\textbf{\faUserShield\ Autorités de Protection Civile}

Les services de protection civile et les pompiers forestiers représentent les utilisateurs principaux du système. Leur mission consiste à surveiller les zones à risque, coordonner les interventions et gérer les ressources humaines et matérielles. Ils requièrent :
\begin{itemize}[leftmargin=0.5cm, itemsep=3pt]
    \item Des alertes instantanées et fiables
    \item Une localisation précise des foyers d'incendie
    \item Des informations sur la propagation probable
    \item Un accès rapide aux données météorologiques
\end{itemize}

\vspace{0.3cm}

\textbf{\faTree\ Gestionnaires Forestiers}

Les agents des Eaux et Forêts et les gestionnaires d'espaces naturels utilisent le système pour la surveillance quotidienne de leurs territoires. Leurs besoins incluent :
\begin{itemize}[leftmargin=0.5cm, itemsep=3pt]
    \item La surveillance continue des zones forestières
    \item L'identification des zones à haut risque
    \item L'historique des incidents pour l'analyse
    \item La planification préventive des interventions
\end{itemize}

\vspace{0.3cm}

\textbf{\faCity\ Collectivités Territoriales}

Les communes et régions concernées par les zones forestières peuvent utiliser le système pour :
\begin{itemize}[leftmargin=0.5cm, itemsep=3pt]
    \item La coordination avec les services d'urgence
    \item L'information et l'alerte des populations
    \item La gestion des évacuations si nécessaire
\end{itemize}
\end{infobox}

\subsubsection{Administrateurs Système}

Les administrateurs techniques sont responsables du bon fonctionnement, de la maintenance et de l'évolution du système. Leur rôle est crucial pour garantir la disponibilité et les performances de la plateforme.

\begin{techbox}{Responsabilités des Administrateurs}
\begin{itemize}[leftmargin=0.5cm, itemsep=8pt]
    \item[\textcolor{accentTeal}{\faServer}] \textbf{Gestion de l'Infrastructure :} Déploiement, monitoring et maintenance des serveurs backend et frontend
    
    \item[\textcolor{accentTeal}{\faCogs}] \textbf{Configuration des Services :} Paramétrage des API externes (NASA FIRMS, Sentinel Hub), gestion des clés API et des quotas
    
    \item[\textcolor{accentTeal}{\faBrain}] \textbf{Gestion des Modèles IA :} Mise à jour et réentraînement des modèles de détection (YOLOv8, MobileNetV2)
    
    \item[\textcolor{accentTeal}{\faBell}] \textbf{Configuration des Alertes :} Paramétrage des canaux de notification (Telegram, Email) et gestion des destinataires
    
    \item[\textcolor{accentTeal}{\faChartBar}] \textbf{Monitoring des Performances :} Suivi des métriques de performance et optimisation continue
\end{itemize}
\end{techbox}

\subsubsection{Équipe de Développement}

L'équipe de développement assure la conception, l'implémentation et l'évolution continue du système. Elle est composée de profils complémentaires travaillant en synergie.

\begin{table}[H]
\centering
\caption{Composition de l'équipe de développement}
\label{tab:equipe}
\rowcolors{2}{mintGreen!30}{white}
\begin{tabular}{l l p{7cm}}
\toprule
\rowcolor{primaryGreen}
\textcolor{white}{\textbf{Rôle}} & \textcolor{white}{\textbf{Compétences}} & \textcolor{white}{\textbf{Responsabilités}} \\
\midrule
Développeur Backend & Python, FastAPI, IA & API REST, intégration modèles IA, services externes \\
Développeur Frontend & React, TypeScript & Interface utilisateur, cartographie, UX/UI \\
Ingénieur IA/ML & TensorFlow, PyTorch & Entraînement modèles, optimisation, évaluation \\
DevOps & Docker, CI/CD & Déploiement, automatisation, monitoring \\
\bottomrule
\end{tabular}
\end{table}

\vspace{0.5cm}

\begin{center}
\begin{tikzpicture}[
    actor/.style={circle, draw=primaryGreen, line width=2pt, fill=mintGreen, minimum size=2cm, align=center, font=\small},
    system/.style={rectangle, rounded corners=15pt, draw=darkGreen, line width=3pt, fill=leafGreen!20, minimum width=6cm, minimum height=3cm, align=center}
]
    % Système central
    \node[system] (sys) at (0,0) {\textbf{\large AI Sentinel}\\[5pt]\footnotesize Système de Détection\\des Incendies de Forêt};
    
    % Acteurs
    \node[actor] (user) at (-5, 2) {\faUser\\Utilisateur};
    \node[actor] (admin) at (5, 2) {\faUserCog\\Admin};
    \node[actor] (dev) at (-5, -2) {\faCode\\Développeur};
    \node[actor] (ext) at (5, -2) {\faCloud\\Services\\Externes};
    
    % Connexions
    \draw[->, >=stealth, line width=1.5pt, primaryGreen] (user) -- (sys);
    \draw[->, >=stealth, line width=1.5pt, primaryGreen] (admin) -- (sys);
    \draw[->, >=stealth, line width=1.5pt, primaryGreen] (dev) -- (sys);
    \draw[<->, >=stealth, line width=1.5pt, accentTeal] (ext) -- (sys);
    
\end{tikzpicture}
\end{center}

% ============================================
\newpage
\section{Analyse des Besoins Fonctionnels}
% ============================================

L'analyse des besoins fonctionnels définit ce que le système doit \textbf{faire}. Chaque besoin fonctionnel décrit une fonctionnalité spécifique que le système doit offrir à ses utilisateurs. Dans le cadre d'AI Sentinel, nous avons organisé ces besoins en \textbf{huit modules} distincts, chacun répondant à un aspect particulier de la détection et de la prévention des incendies.

\begin{greenbox}[\faListOl\ Vue d'Ensemble des Modules Fonctionnels]
\begin{center}
\begin{tikzpicture}[
    module/.style={rectangle, rounded corners=8pt, draw=primaryGreen, line width=1.5pt, fill=mintGreen, text width=3.8cm, minimum height=1.8cm, align=center, font=\small}
]
    % Ligne 1
    \node[module] (m1) at (0,0) {\textcolor{darkGreen}{\faVideo}\\[3pt]\textbf{Détection}\\Temps Réel};
    \node[module] (m2) at (4.5,0) {\textcolor{darkGreen}{\faImage}\\[3pt]\textbf{Classification}\\d'Images};
    \node[module] (m3) at (9,0) {\textcolor{darkGreen}{\faFilm}\\[3pt]\textbf{Analyse}\\Vidéo};
    \node[module] (m4) at (13.5,0) {\textcolor{darkGreen}{\faSatellite}\\[3pt]\textbf{Surveillance}\\Satellite};
    
    % Ligne 2
    \node[module] (m5) at (0,-2.5) {\textcolor{darkGreen}{\faMapMarkerAlt}\\[3pt]\textbf{Prévention}\\Hotspots};
    \node[module] (m6) at (4.5,-2.5) {\textcolor{darkGreen}{\faChartLine}\\[3pt]\textbf{Prédiction}\\Propagation};
    \node[module] (m7) at (9,-2.5) {\textcolor{darkGreen}{\faBell}\\[3pt]\textbf{Notifications}\\Multi-canaux};
    \node[module] (m8) at (13.5,-2.5) {\textcolor{darkGreen}{\faCloudSun}\\[3pt]\textbf{FWI}\\Météo};
\end{tikzpicture}
\end{center}
\end{greenbox}

\subsection{Module de Détection en Temps Réel}

Le module de détection en temps réel constitue le \textbf{cœur} du système AI Sentinel. Il permet l'analyse continue de flux vidéo provenant de caméras de surveillance pour détecter instantanément la présence de feu ou de fumée. Ce module exploite la puissance du modèle \textbf{YOLOv8} (You Only Look Once, version 8), un algorithme de détection d'objets state-of-the-art reconnu pour sa rapidité et sa précision.

\subsubsection{Contexte et Justification}

La détection en temps réel répond à un besoin critique : identifier les incendies \textbf{dès leur déclenchement}, avant qu'ils ne se propagent de manière incontrôlable. Les méthodes traditionnelles de surveillance (tours de guet, patrouilles) présentent des limitations importantes en termes de couverture, de disponibilité 24h/24, et de vitesse de détection. L'utilisation de l'intelligence artificielle permet de surmonter ces obstacles en offrant une surveillance automatisée, continue et objective.

\begin{alertbox}{Importance de la Réactivité}
Selon les études de l'Office National des Forêts (ONF), un feu de forêt peut doubler de taille toutes les \textbf{15 à 20 minutes} dans des conditions favorables à la propagation (vent fort, végétation sèche). Chaque minute gagnée dans la détection représente potentiellement des hectares de forêt préservés et des vies sauvées.
\end{alertbox}

\subsubsection{Besoins Fonctionnels Détaillés}

\begin{objectifbox}{RF01 : Capture Vidéo via Caméra/Webcam}
\textbf{Description :} Le système doit être capable de capturer un flux vidéo en temps réel à partir de différentes sources : webcam connectée, caméra IP, ou fichier vidéo local.

\textbf{Spécifications détaillées :}
\begin{itemize}[leftmargin=1cm, itemsep=5pt]
    \item Prise en charge des webcams USB standard via l'index de périphérique
    \item Support des flux RTSP pour les caméras IP professionnelles
    \item Capacité à traiter des fichiers vidéo uploadés (MP4, AVI, MOV)
    \item Résolution supportée : de 480p à 1080p
    \item Fréquence d'images : 25-30 FPS minimum
\end{itemize}

\textbf{Critère d'acceptation :} Le système doit démarrer la capture vidéo en moins de 2 secondes après activation par l'utilisateur.
\end{objectifbox}

\vspace{0.3cm}

\begin{objectifbox}{RF02 : Détection Automatique Feu/Fumée}
\textbf{Description :} Le système doit analyser chaque frame du flux vidéo pour détecter la présence de feu ou de fumée avec une précision élevée.

\textbf{Spécifications détaillées :}
\begin{itemize}[leftmargin=1cm, itemsep=5pt]
    \item Utilisation du modèle YOLOv8 personnalisé entraîné sur un dataset de feux de forêt
    \item Classification en deux classes : \texttt{Fire} (feu) et \texttt{Smoke} (fumée)
    \item Seuil de confiance configurable (par défaut : 0.5)
    \item Traitement à minimum 30 FPS sur GPU, 15 FPS sur CPU
    \item Détection multi-instances (plusieurs feux/fumées simultanés)
\end{itemize}

\textbf{Critère d'acceptation :} Le modèle doit atteindre une précision de détection $\geq$ 85\% avec un taux de faux positifs < 5\%.
\end{objectifbox}

\vspace{0.3cm}

\begin{objectifbox}{RF03 : Affichage des Bounding Boxes}
\textbf{Description :} Les objets détectés (feu, fumée) doivent être encadrés visuellement sur le flux vidéo avec des informations contextuelles.

\textbf{Spécifications détaillées :}
\begin{itemize}[leftmargin=1cm, itemsep=5pt]
    \item Rectangles de délimitation (bounding boxes) colorés selon la classe
    \item Code couleur : \textcolor{moroccanRed}{Rouge} pour le feu, \textcolor{textGray}{Gris} pour la fumée
    \item Affichage du nom de la classe et du score de confiance (ex: "Fire 0.92")
    \item Épaisseur et taille de police adaptatives selon la résolution
    \item Mise à jour en temps réel synchronisée avec le flux vidéo
\end{itemize}

\textbf{Critère d'acceptation :} Les annotations visuelles ne doivent pas dégrader les performances de plus de 5\%.
\end{objectifbox}

\vspace{0.3cm}

\begin{objectifbox}{RF04 : Génération d'Alertes Visuelles}
\textbf{Description :} Lorsqu'un feu ou une fumée est détecté avec un niveau de confiance suffisant, le système doit générer des alertes visuelles immédiates.

\textbf{Spécifications détaillées :}
\begin{itemize}[leftmargin=1cm, itemsep=5pt]
    \item Indicateur visuel clignotant sur l'interface lors d'une détection
    \item Changement de couleur de l'arrière-plan ou du cadre vidéo
    \item Notification toast affichant les détails de la détection
    \item Son d'alerte optionnel (configurable par l'utilisateur)
    \item Horodatage précis de chaque détection
\end{itemize}

\textbf{Critère d'acceptation :} L'alerte visuelle doit apparaître dans un délai maximum de 500ms après la détection.
\end{objectifbox}

\vspace{0.5cm}

\begin{center}
\begin{tikzpicture}[
    step/.style={rectangle, rounded corners=5pt, draw=accentTeal, line width=1.5pt, fill=skyBlue!20, text width=2.5cm, minimum height=1.5cm, align=center, font=\footnotesize},
    arrow/.style={->, >=stealth, line width=1.5pt, color=accentTeal}
]
    \node[step] (s1) at (0,0) {\faVideo\\Capture\\Vidéo};
    \node[step] (s2) at (3.5,0) {\faCropAlt\\Prétraitement\\Frame};
    \node[step] (s3) at (7,0) {\faBrain\\Inférence\\YOLOv8};
    \node[step] (s4) at (10.5,0) {\faVectorSquare\\Bounding\\Boxes};
    \node[step] (s5) at (14,0) {\faBell\\Alerte\\Visuelle};
    
    \draw[arrow] (s1) -- (s2);
    \draw[arrow] (s2) -- (s3);
    \draw[arrow] (s3) -- (s4);
    \draw[arrow] (s4) -- (s5);
    
    \node[font=\bfseries, text=darkGreen] at (7, 1.5) {Pipeline de Détection Temps Réel};
\end{tikzpicture}
\end{center}

\subsection{Module de Classification d'Images}

Le module de classification d'images permet aux utilisateurs d'analyser des photographies statiques pour déterminer si elles contiennent des signes de feu, de fumée, ou aucun des deux. Ce module utilise l'architecture \textbf{MobileNetV2} avec transfer learning, offrant un excellent compromis entre précision et performance.

\subsubsection{Contexte et Justification}

Complémentaire à la détection temps réel, la classification d'images répond à plusieurs cas d'usage importants :
\begin{itemize}[leftmargin=1cm, itemsep=5pt]
    \item \textbf{Vérification manuelle :} Confirmation par l'opérateur d'une image suspecte
    \item \textbf{Analyse rétrospective :} Examen d'images historiques
    \item \textbf{Traitement hors ligne :} Analyse sans connexion caméra en direct
    \item \textbf{Rapports et documentation :} Génération de preuves visuelles
\end{itemize}

\subsubsection{Besoins Fonctionnels Détaillés}

\begin{objectifbox}{RF05 : Upload d'Images}
\textbf{Description :} L'utilisateur doit pouvoir soumettre une ou plusieurs images au système pour analyse.

\textbf{Spécifications détaillées :}
\begin{itemize}[leftmargin=1cm, itemsep=5pt]
    \item Formats supportés : JPEG, PNG, WebP, BMP
    \item Taille maximale par image : 10 MB
    \item Interface drag-and-drop intuitive
    \item Prévisualisation de l'image avant soumission
    \item Upload multiple avec file d'attente de traitement
\end{itemize}

\textbf{Critère d'acceptation :} L'upload d'une image de 5 MB doit se terminer en moins de 3 secondes sur une connexion standard.
\end{objectifbox}

\vspace{0.3cm}

\begin{objectifbox}{RF06 : Classification Multi-Classes}
\textbf{Description :} Le système doit classifier l'image uploadée dans l'une des trois catégories : Fire, Smoke, ou Non-Fire.

\textbf{Spécifications détaillées :}
\begin{itemize}[leftmargin=1cm, itemsep=5pt]
    \item Modèle MobileNetV2 pré-entraîné sur ImageNet, fine-tuné sur dataset feu
    \item Trois classes de sortie avec probabilités associées
    \item Prétraitement automatique (redimensionnement 224×224, normalisation)
    \item Temps d'inférence < 500ms par image
    \item Précision globale $\geq$ 97\% sur le jeu de test
\end{itemize}

\textbf{Critère d'acceptation :} Le système doit retourner la classe prédite avec son score de confiance en moins de 1 seconde.
\end{objectifbox}

\vspace{0.3cm}

\begin{objectifbox}{RF07 : Affichage du Score de Confiance}
\textbf{Description :} Le résultat de la classification doit inclure un score de confiance permettant à l'utilisateur d'évaluer la fiabilité de la prédiction.

\textbf{Spécifications détaillées :}
\begin{itemize}[leftmargin=1cm, itemsep=5pt]
    \item Score de confiance exprimé en pourcentage (0-100\%)
    \item Affichage visuel avec code couleur (vert > 80\%, orange 50-80\%, rouge < 50\%)
    \item Distribution des probabilités pour les trois classes
    \item Indicateur visuel du niveau de certitude du modèle
    \item Recommandation d'action basée sur le niveau de confiance
\end{itemize}

\textbf{Critère d'acceptation :} L'interface doit afficher clairement le résultat avec les probabilités pour chaque classe.
\end{objectifbox}

\vspace{0.5cm}

\begin{table}[H]
\centering
\caption{Exemple de sortie du module de classification}
\label{tab:classification}
\rowcolors{2}{mintGreen!30}{white}
\begin{tabular}{l c l}
\toprule
\rowcolor{primaryGreen}
\textcolor{white}{\textbf{Classe}} & \textcolor{white}{\textbf{Probabilité}} & \textcolor{white}{\textbf{Interprétation}} \\
\midrule
\textbf{Fire} & 92.3\% & \textcolor{moroccanRed}{Détection confirmée} \\
Smoke & 5.1\% & Trace possible \\
Non-Fire & 2.6\% & Négligeable \\
\bottomrule
\end{tabular}
\end{table}

\subsection{Module d'Analyse Vidéo}

Le module d'analyse vidéo étend les capacités de détection aux fichiers vidéo préenregistrés. Il permet un traitement exhaustif frame par frame et génère une vidéo annotée avec les détections.

\subsubsection{Besoins Fonctionnels Détaillés}

\begin{objectifbox}{RF08 : Upload de Vidéos}
\textbf{Description :} L'utilisateur doit pouvoir soumettre des fichiers vidéo pour analyse complète.

\textbf{Spécifications détaillées :}
\begin{itemize}[leftmargin=1cm, itemsep=5pt]
    \item Formats supportés : MP4, AVI, MOV, MKV, WebM
    \item Taille maximale : 500 MB (configurable)
    \item Barre de progression de l'upload
    \item Validation du format avant traitement
    \item Support des différentes résolutions (480p à 4K)
\end{itemize}
\end{objectifbox}

\vspace{0.3cm}

\begin{objectifbox}{RF09 : Traitement Frame par Frame}
\textbf{Description :} Chaque frame de la vidéo doit être analysée individuellement par le modèle de détection.

\textbf{Spécifications détaillées :}
\begin{itemize}[leftmargin=1cm, itemsep=5pt]
    \item Extraction et analyse de chaque frame
    \item Affichage de la progression du traitement
    \item Statistiques en temps réel (frames traitées, détections)
    \item Possibilité d'annuler le traitement en cours
    \item Traitement asynchrone pour ne pas bloquer l'interface
\end{itemize}
\end{objectifbox}

\vspace{0.3cm}

\begin{objectifbox}{RF10 : Export Vidéo Annotée}
\textbf{Description :} Le système doit générer une version annotée de la vidéo avec les détections superposées.

\textbf{Spécifications détaillées :}
\begin{itemize}[leftmargin=1cm, itemsep=5pt]
    \item Génération d'une nouvelle vidéo avec bounding boxes
    \item Conservation de la résolution et du framerate d'origine
    \item Téléchargement direct du fichier résultat
    \item Rapport JSON des détections (timestamps, coordonnées, classes)
    \item Prévisualisation avant téléchargement
\end{itemize}
\end{objectifbox}

% ============================================
% CHAPITRE III - PARTIE 2 : MODULES SATELLITE, PRÉVENTION, PRÉDICTION
% ============================================

\subsection{Module de Surveillance Satellite}

Le module de surveillance satellite représente une composante stratégique du système AI Sentinel. Il permet d'étendre la couverture de détection au-delà des zones équipées de caméras en exploitant les images satellites du programme européen \textbf{Copernicus} via l'API \textbf{Sentinel Hub}, ainsi que les données de hotspots thermiques de \textbf{NASA FIRMS}.

\subsubsection{Contexte et Justification}

Les forêts marocaines s'étendent sur près de \textbf{9 millions d'hectares}, une superficie impossible à couvrir intégralement par des caméras de surveillance. L'imagerie satellite offre une solution complémentaire permettant une surveillance à grande échelle. Les satellites Sentinel-2 fournissent des images multi-spectrales avec une résolution spatiale de 10 à 60 mètres et une revisite de 5 jours, tandis que les satellites MODIS et VIIRS de NASA FIRMS détectent les anomalies thermiques en quasi temps réel.

\begin{moroccobox}{Couverture Géographique du Maroc}
Le système AI Sentinel surveille \textbf{8 zones géographiques} couvrant l'ensemble du territoire marocain à risque d'incendie :

\vspace{0.3cm}
\begin{center}
\begin{tabular}{l l l}
\textbf{Zone} & \textbf{Coordonnées} & \textbf{Caractéristiques} \\
\hline
North (Tanger-Tétouan) & -6.00, 34.00 & Forêts méditerranéennes \\
Rif & -5.00, 34.50 & Montagnes boisées \\
Oriental & -3.00, 33.50 & Steppe et maquis \\
Casablanca & -8.00, 33.00 & Zone périurbaine \\
Middle Atlas & -6.00, 32.50 & Cèdres et chênes verts \\
Marrakech & -8.50, 31.00 & Arganeraie \\
High Atlas & -8.00, 30.50 & Forêts d'altitude \\
Souss & -10.00, 29.50 & Formations semi-arides \\
\end{tabular}
\end{center}
\end{moroccobox}

\subsubsection{Besoins Fonctionnels Détaillés}

\begin{objectifbox}{RF11 : Acquisition d'Images Sentinel-2}
\textbf{Description :} Le système doit pouvoir récupérer automatiquement les images satellites Sentinel-2 pour les zones définies.

\textbf{Spécifications détaillées :}
\begin{itemize}[leftmargin=1cm, itemsep=5pt]
    \item Authentification OAuth2 avec l'API Sentinel Hub
    \item Récupération d'images Sentinel-2 L2A (atmosphériquement corrigées)
    \item Deux types de scripts d'évaluation (evalscripts) :
    \begin{itemize}
        \item \textbf{True Color :} RGB standard pour visualisation
        \item \textbf{Fire Detection :} SWIR enhanced pour détection thermique
    \end{itemize}
    \item Résolution de 10m/pixel pour les bandes visibles
    \item Gestion intelligente de la couverture nuageuse (seuil < 30\%)
    \item Stockage local des images pour analyse ultérieure
\end{itemize}

\textbf{Critère d'acceptation :} Le système doit récupérer une image satellite en moins de 10 secondes (hors temps de traitement Sentinel Hub).
\end{objectifbox}

\vspace{0.3cm}

\begin{objectifbox}{RF12 : Analyse IA des Images Satellites}
\textbf{Description :} Les images satellites doivent être analysées par un modèle IA spécialisé (CAM - Class Activation Map) pour détecter les zones de feu potentiel.

\textbf{Spécifications détaillées :}
\begin{itemize}[leftmargin=1cm, itemsep=5pt]
    \item Modèle CAM personnalisé avec sortie duale :
    \begin{itemize}
        \item Carte d'activation (10×10×64) pour visualisation des zones critiques
        \item Classification binaire (Fire / No Fire)
    \end{itemize}
    \item Prétraitement : redimensionnement 224×224, normalisation 0-1
    \item Génération de heatmaps colorées superposées à l'image originale
    \item Score de confiance pour chaque analyse
    \item Détection des anomalies thermiques via les bandes SWIR
\end{itemize}

\textbf{Critère d'acceptation :} L'analyse d'une image satellite doit se compléter en moins de 5 secondes avec un taux de détection > 90\%.
\end{objectifbox}

\vspace{0.3cm}

\begin{objectifbox}{RF13 : Scan Automatique Programmé}
\textbf{Description :} Le système doit pouvoir effectuer des scans automatiques périodiques de toutes les zones surveillées.

\textbf{Spécifications détaillées :}
\begin{itemize}[leftmargin=1cm, itemsep=5pt]
    \item Scheduler configurable (intervalle par défaut : 30 minutes)
    \item Scan séquentiel de toutes les zones pour éviter la surcharge API
    \item Détection des changements par comparaison avec le scan précédent
    \item Génération automatique d'alertes en cas de détection positive
    \item Interface de contrôle : démarrer/arrêter le monitoring
    \item Logs détaillés de chaque cycle de scan
    \item Résilience aux erreurs (retry automatique)
\end{itemize}

\textbf{Critère d'acceptation :} Le scan de toutes les zones doit se terminer en moins de 10 minutes avec génération d'un rapport de synthèse.
\end{objectifbox}

\vspace{0.5cm}

\begin{center}
\begin{tikzpicture}[
    node distance=1.5cm,
    block/.style={rectangle, rounded corners=8pt, draw=accentTeal, line width=1.5pt, fill=skyBlue!15, text width=2.8cm, minimum height=1.8cm, align=center, font=\footnotesize},
    decision/.style={diamond, draw=alertOrange, line width=1.5pt, fill=sunYellow!30, text width=1.5cm, align=center, font=\footnotesize, aspect=2},
    arrow/.style={->, >=stealth, line width=1pt, color=primaryGreen}
]
    \node[block] (start) {\faClock\\Déclenchement\\Scheduler};
    \node[block, right=of start] (fetch) {\faSatellite\\Récupération\\Image Sentinel};
    \node[block, right=of fetch] (analyze) {\faBrain\\Analyse\\CAM Model};
    \node[decision, right=of analyze] (detect) {Feu\\?};
    \node[block, above right=1cm and 1.5cm of detect] (alert) {\faBell\\Envoi\\Alerte};
    \node[block, below right=1cm and 1.5cm of detect] (log) {\faDatabase\\Log\\Résultat};
    \node[block, right=3.5cm of detect] (next) {\faRedo\\Zone\\Suivante};
    
    \draw[arrow] (start) -- (fetch);
    \draw[arrow] (fetch) -- (analyze);
    \draw[arrow] (analyze) -- (detect);
    \draw[arrow] (detect) -- node[above, font=\tiny, text=moroccanRed] {Oui} (alert);
    \draw[arrow] (detect) -- node[below, font=\tiny, text=primaryGreen] {Non} (log);
    \draw[arrow] (alert) -| (next);
    \draw[arrow] (log) -| (next);
    
    \node[font=\bfseries\small, text=darkGreen] at (5, -2.5) {Workflow de Surveillance Satellite Automatisée};
\end{tikzpicture}
\end{center}

\subsection{Module de Prévention (Hotspots NASA FIRMS)}

Le module de prévention exploite les données de \textbf{NASA FIRMS} (Fire Information for Resource Management System) pour afficher en temps quasi réel les hotspots thermiques détectés par les satellites MODIS et VIIRS. Ces données permettent une vision globale des zones à risque et des feux actifs.

\subsubsection{Contexte et Justification}

NASA FIRMS fournit gratuitement des données de détection thermique avec une latence de seulement \textbf{3 heures}, permettant une surveillance quasi temps réel à l'échelle globale. Ces données sont essentielles pour :
\begin{itemize}[leftmargin=1cm, itemsep=3pt]
    \item Identifier les feux actifs dans des zones non couvertes par les caméras
    \item Cartographier les tendances et patterns d'incendies
    \item Anticiper les risques de propagation vers les zones habitées
    \item Compléter les données de terrain avec une vue satellite
\end{itemize}

\begin{infobox}{Données NASA FIRMS}
Les satellites MODIS (Moderate Resolution Imaging Spectroradiometer) et VIIRS (Visible Infrared Imaging Radiometer Suite) détectent les anomalies thermiques en mesurant la luminosité dans les bandes infrarouges thermiques. Un pixel est marqué comme hotspot lorsque sa température dépasse significativement celle des pixels environnants.

\textbf{Caractéristiques des données :}
\begin{itemize}[leftmargin=0.5cm, itemsep=3pt]
    \item \textbf{Résolution MODIS :} 1 km
    \item \textbf{Résolution VIIRS :} 375 m
    \item \textbf{Latence :} 3 heures après passage satellite
    \item \textbf{Couverture :} Globale, plusieurs passages par jour
\end{itemize}
\end{infobox}

\subsubsection{Besoins Fonctionnels Détaillés}

\begin{objectifbox}{RF14 : Récupération des Données NASA FIRMS}
\textbf{Description :} Le système doit récupérer automatiquement les données de hotspots depuis l'API NASA FIRMS pour les régions surveillées.

\textbf{Spécifications détaillées :}
\begin{itemize}[leftmargin=1cm, itemsep=5pt]
    \item Authentification via MAP\_KEY NASA FIRMS
    \item Récupération des données VIIRS\_SNPP pour les dernières 24 heures
    \item Filtrage par zone géographique (bounding box)
    \item Parsing des attributs : latitude, longitude, brightness, confidence, frp, daynight
    \item Mise en cache pour optimiser les requêtes répétées
    \item Gestion des erreurs et retry automatique
\end{itemize}

\textbf{Critère d'acceptation :} Les données doivent être récupérées et parsées en moins de 5 secondes par zone.
\end{objectifbox}

\vspace{0.3cm}

\begin{objectifbox}{RF15 : Affichage sur Carte Interactive}
\textbf{Description :} Les hotspots doivent être affichés sur une carte interactive permettant la navigation et l'exploration des données.

\textbf{Spécifications détaillées :}
\begin{itemize}[leftmargin=1cm, itemsep=5pt]
    \item Carte interactive basée sur Leaflet/React-Leaflet
    \item Marqueurs colorés selon le niveau de confiance :
    \begin{itemize}
        \item \textcolor{moroccanRed}{\textbf{Rouge}} : High confidence
        \item \textcolor{alertOrange}{\textbf{Orange}} : Nominal confidence
        \item \textcolor{sunYellow}{\textbf{Jaune}} : Low confidence
    \end{itemize}
    \item Popup d'information au clic sur un marqueur
    \item Zoom et pan fluides
    \item Clustering automatique des marqueurs à faible zoom
    \item Filtres par région, date, niveau de confiance
    \item Lien vers Google Maps pour navigation
\end{itemize}

\textbf{Critère d'acceptation :} La carte doit afficher jusqu'à 500 marqueurs sans dégradation notable des performances.
\end{objectifbox}

\vspace{0.3cm}

\begin{objectifbox}{RF16 : Calcul du Rayon de Propagation}
\textbf{Description :} Pour chaque hotspot, le système doit calculer une estimation du rayon de propagation probable basée sur les données disponibles.

\textbf{Spécifications détaillées :}
\begin{itemize}[leftmargin=1cm, itemsep=5pt]
    \item Algorithme de calcul intégrant plusieurs facteurs :
    \begin{itemize}
        \item \textbf{Luminosité (brightness\_ti4/ti5)} : indicateur d'intensité
        \item \textbf{Confiance NASA} : fiabilité de la détection
        \item \textbf{FRP (Fire Radiative Power)} : puissance du feu en MW
    \end{itemize}
    \item Affichage d'un cercle de propagation sur la carte
    \item Rayon borné entre 1 km et 15 km
    \item Code couleur du cercle selon le niveau de risque
    \item Tooltip avec estimation de la surface menacée
\end{itemize}

\textbf{Critère d'acceptation :} Le rayon de propagation doit être calculé et affiché en moins de 100ms par hotspot.
\end{objectifbox}

\subsection{Module de Prédiction de Propagation}

Le module de prédiction de propagation constitue l'élément d'anticipation du système AI Sentinel. Il permet d'estimer, à horizon de 6 heures, la zone potentiellement affectée par la propagation d'un incendie en tenant compte de multiples facteurs environnementaux.

\subsubsection{Contexte et Justification}

La prédiction de la propagation des feux de forêt est un problème complexe qui dépend de nombreuses variables : topographie, type de végétation, conditions météorologiques, humidité du sol, etc. Notre approche empirique, basée sur les données réelles de NASA FIRMS et les données météorologiques, offre une estimation utilisable opérationnellement tout en restant computationnellement légère.

\begin{alertbox}{Importance de la Prédiction}
Une prédiction, même approximative, de la propagation d'un feu permet aux services d'intervention de :
\begin{itemize}[leftmargin=0.5cm, itemsep=3pt]
    \item \textbf{Anticiper} les zones à évacuer
    \item \textbf{Positionner} les moyens de lutte de manière optimale
    \item \textbf{Protéger} les infrastructures critiques en priorité
    \item \textbf{Informer} les populations à risque
\end{itemize}
\end{alertbox}

\subsubsection{Besoins Fonctionnels Détaillés}

\begin{objectifbox}{RF17 : Saisie des Paramètres de Simulation}
\textbf{Description :} L'utilisateur doit pouvoir saisir ou modifier les paramètres de simulation de propagation.

\textbf{Spécifications détaillées :}
\begin{itemize}[leftmargin=1cm, itemsep=5pt]
    \item Interface de saisie intuitive avec valeurs par défaut
    \item Paramètres configurables :
    \begin{itemize}
        \item \textbf{Luminosité (Kelvin)} : 300K - 500K (défaut : 350K)
        \item \textbf{Confiance NASA} : high, nominal, low
        \item \textbf{Vitesse du vent} : 0 - 100 km/h
        \item \textbf{Direction du vent} : 0° - 360°
    \end{itemize}
    \item Sliders et sélecteurs pour faciliter la saisie
    \item Validation des valeurs en temps réel
    \item Pré-remplissage automatique depuis les données réelles
\end{itemize}

\textbf{Critère d'acceptation :} Tous les paramètres doivent être modifiables avec feedback visuel immédiat.
\end{objectifbox}

\vspace{0.3cm}

\begin{objectifbox}{RF18 : Calcul Prédictif de Propagation}
\textbf{Description :} Le système doit calculer le rayon de propagation estimé à 6 heures basé sur l'algorithme empirique.

\textbf{Algorithme de calcul :}

\begin{center}
\begin{tcolorbox}[
    enhanced,
    colback=white,
    colframe=accentTeal,
    boxrule=2pt,
    rounded corners,
    width=12cm
]
\begin{align*}
\text{Rayon} &= \text{Base} + f(\text{Brightness}) + f(\text{Confidence}) + f(\text{Wind})
\end{align*}

Où :
\begin{itemize}[leftmargin=0.5cm]
    \item $\text{Base} = 3.0$ km (rayon de base)
    \item $f(\text{Brightness})$ :
    \begin{itemize}
        \item $> 350K \rightarrow +3.0$ km
        \item $> 320K \rightarrow +1.5$ km
        \item sinon $\rightarrow +0.0$ km
    \end{itemize}
    \item $f(\text{Confidence})$ :
    \begin{itemize}
        \item high $\rightarrow +1.0$ km
        \item low $\rightarrow -1.0$ km
    \end{itemize}
    \item $f(\text{Wind}) = (\text{vitesse} / 30) \times 2.0$ km
\end{itemize}

Résultat final : $\max(1.0, \min(15.0, \text{Rayon}))$ km
\end{tcolorbox}
\end{center}

\textbf{Critère d'acceptation :} Le calcul doit s'exécuter en moins de 50ms avec une précision de 2 décimales.
\end{objectifbox}

\vspace{0.3cm}

\begin{objectifbox}{RF19 : Visualisation de la Zone de Danger}
\textbf{Description :} La zone de danger prédite doit être visualisée sur une carte interactive avec des indicateurs clairs.

\textbf{Spécifications détaillées :}
\begin{itemize}[leftmargin=1cm, itemsep=5pt]
    \item Cercle de propagation centré sur le hotspot
    \item Dégradé de couleur du centre (rouge intense) vers l'extérieur (jaune)
    \item Flèche indiquant la direction du vent
    \item Affichage des coordonnées du centre
    \item Estimation de la surface en km²
    \item Animation optionnelle de l'expansion
    \item Légende explicative des zones de risque
\end{itemize}

\textbf{Critère d'acceptation :} La visualisation doit se mettre à jour en temps réel lors de la modification des paramètres.
\end{objectifbox}

\vspace{0.5cm}

\begin{table}[H]
\centering
\caption{Exemples de calcul de propagation}
\label{tab:propagation}
\rowcolors{2}{mintGreen!30}{white}
\begin{tabular}{c c c c c}
\toprule
\rowcolor{primaryGreen}
\textcolor{white}{\textbf{Brightness}} & \textcolor{white}{\textbf{Confiance}} & \textcolor{white}{\textbf{Vent (km/h)}} & \textcolor{white}{\textbf{Rayon (km)}} & \textcolor{white}{\textbf{Surface (km²)}} \\
\midrule
320 K & Nominal & 20 & 5.83 & 106.9 \\
355 K & High & 40 & 9.67 & 293.6 \\
380 K & High & 60 & 11.00 & 380.1 \\
310 K & Low & 10 & 2.67 & 22.4 \\
400 K & High & 80 & 12.33 & 477.7 \\
\bottomrule
\end{tabular}
\end{table}

\subsection{Module de Notifications}

Le module de notifications assure la diffusion rapide et fiable des alertes vers les parties prenantes via différents canaux de communication. Cette composante est critique pour garantir une réponse rapide aux détections.

\subsubsection{Contexte et Justification}

La détection n'a de valeur que si elle est suivie d'une action. Le module de notifications transforme une détection technique en information actionnable pour les équipes terrain. L'utilisation de canaux multiples (\textbf{Email} et \textbf{Telegram}) garantit que l'alerte atteindra ses destinataires même en cas de défaillance d'un canal.

\begin{greenbox}[\faBell\ Philosophie du Système d'Alertes]
Notre système d'alertes repose sur trois principes fondamentaux :

\begin{enumerate}[leftmargin=1cm, itemsep=5pt]
    \item \textbf{Rapidité :} L'alerte doit être envoyée dans les secondes suivant la détection
    \item \textbf{Fiabilité :} Utilisation de canaux redondants (Email + Telegram)
    \item \textbf{Non-intrusivité :} Mécanisme anti-spam pour éviter la fatigue d'alerte
\end{enumerate}
\end{greenbox}

\subsubsection{Besoins Fonctionnels Détaillés}

\begin{objectifbox}{RF20 : Envoi d'Email avec Détails}
\textbf{Description :} Le système doit envoyer des alertes par email contenant toutes les informations pertinentes sur la détection.

\textbf{Spécifications détaillées :}
\begin{itemize}[leftmargin=1cm, itemsep=5pt]
    \item Envoi via SMTP (Gmail ou serveur configuré)
    \item Contenu HTML richement formaté :
    \begin{itemize}
        \item Titre d'alerte avec niveau de sévérité
        \item Zone géographique concernée
        \item Coordonnées GPS du point de détection
        \item Image satellite ou capture d'écran intégrée
        \item Prédiction de propagation
        \item Lien Google Maps pour localisation rapide
        \item Horodatage précis
    \end{itemize}
    \item Gestion des pièces jointes (images)
    \item Configuration des destinataires multiples
    \item Retry automatique en cas d'échec (3 tentatives)
\end{itemize}

\textbf{Critère d'acceptation :} L'email doit être envoyé dans les 10 secondes suivant la détection.
\end{objectifbox}

\vspace{0.3cm}

\begin{objectifbox}{RF21 : Envoi de Message Telegram}
\textbf{Description :} Le système doit envoyer des alertes instantanées via un bot Telegram.

\textbf{Spécifications détaillées :}
\begin{itemize}[leftmargin=1cm, itemsep=5pt]
    \item Intégration avec l'API Telegram Bot
    \item Message formaté avec emojis pour lisibilité rapide :
    \begin{itemize}
        \item 🔥 \textbf{ALERTE FEU DÉTECTÉ}
        \item 📍 Zone : [nom de la région]
        \item 🌡️ Confiance : [score]\%
        \item 📐 Propagation estimée : [rayon] km
        \item 🗺️ [Lien Google Maps]
    \end{itemize}
    \item Envoi d'image satellite avec le message
    \item Boutons inline pour actions rapides
    \item Notification push instantanée sur mobile
    \item Support des groupes et channels
\end{itemize}

\textbf{Critère d'acceptation :} Le message Telegram doit être reçu dans les 5 secondes suivant la détection.
\end{objectifbox}

\vspace{0.3cm}

\begin{objectifbox}{RF22 : Gestion du Cooldown Anti-Spam}
\textbf{Description :} Le système doit implémenter un mécanisme de cooldown pour éviter l'envoi d'alertes répétées pour la même détection.

\textbf{Spécifications détaillées :}
\begin{itemize}[leftmargin=1cm, itemsep=5pt]
    \item Cooldown par défaut : 30 secondes
    \item Cooldown configurable par zone (5s - 300s)
    \item Tracking par zone géographique + type de détection
    \item Reset du cooldown lors d'un changement significatif
    \item Logs des alertes ignorées pour audit
    \item Interface de monitoring des cooldowns actifs
\end{itemize}

\textbf{Critère d'acceptation :} Aucune alerte dupliquée ne doit être envoyée pendant la période de cooldown.
\end{objectifbox}

\subsection{Module FWI (Fire Weather Index)}

Le module FWI intègre l'indice météorologique de risque d'incendie, un standard international développé par le Service canadien des forêts. Cet indice combine plusieurs paramètres météorologiques pour évaluer le potentiel de départ et de propagation des feux.

\subsubsection{Besoins Fonctionnels Détaillés}

\begin{objectifbox}{RF23 : Affichage des Indices Météo}
\textbf{Description :} Le système doit afficher les indices météorologiques pertinents pour l'évaluation du risque d'incendie.

\textbf{Spécifications détaillées :}
\begin{itemize}[leftmargin=1cm, itemsep=5pt]
    \item Récupération des données via Open-Meteo API
    \item Paramètres affichés :
    \begin{itemize}
        \item Température (°C)
        \item Humidité relative (\%)
        \item Vitesse et direction du vent
        \item Précipitations (mm)
        \item Indice FWI calculé
    \end{itemize}
    \item Mise à jour automatique toutes les heures
    \item Historique sur 24h avec graphiques
    \item Prévisions à 48h
\end{itemize}

\textbf{Critère d'acceptation :} Les données météo doivent être affichées avec une latence maximale de 2 secondes.
\end{objectifbox}

\vspace{0.3cm}

\begin{objectifbox}{RF24 : Carte de Risque}
\textbf{Description :} Le système doit afficher une carte de risque d'incendie basée sur les indices FWI.

\textbf{Spécifications détaillées :}
\begin{itemize}[leftmargin=1cm, itemsep=5pt]
    \item Carte choroplèthe avec zones colorées par niveau de risque
    \item Échelle de risque standardisée :
    \begin{itemize}
        \item \textcolor{primaryGreen}{\textbf{Vert}} : Faible (FWI 0-5)
        \item \textcolor{sunYellow}{\textbf{Jaune}} : Modéré (FWI 5-10)
        \item \textcolor{alertOrange}{\textbf{Orange}} : Élevé (FWI 10-20)
        \item \textcolor{moroccanRed}{\textbf{Rouge}} : Très élevé (FWI 20-30)
        \item \textbf{Pourpre} : Extrême (FWI > 30)
    \end{itemize}
    \item Superposition optionnelle avec les hotspots actifs
    \item Export de la carte en image
    \item Légende interactive
\end{itemize}

\textbf{Critère d'acceptation :} La carte de risque doit se charger en moins de 3 secondes.
\end{objectifbox}

% ============================================
% CHAPITRE III - PARTIE 3 : BESOINS NON FONCTIONNELS ET MÉTHODOLOGIE
% ============================================

\newpage
\section{Analyse des Besoins Non Fonctionnels}

Les besoins non fonctionnels définissent \textbf{comment} le système doit fonctionner, en termes de qualité, de performance et de contraintes techniques. Ces exigences sont tout aussi importantes que les besoins fonctionnels car elles déterminent l'acceptabilité du système par les utilisateurs et sa viabilité à long terme.

Dans le contexte critique de la détection d'incendies, où chaque seconde compte, les exigences de performance, de fiabilité et de disponibilité revêtent une importance capitale.

\subsection{Performance}

La performance du système AI Sentinel est un facteur clé de succès. Un système de détection d'incendies doit être capable de traiter les données rapidement pour permettre une intervention précoce.

\begin{techbox}{Exigences de Performance}

\textbf{NF01 : Temps de Réponse API}

\begin{itemize}[leftmargin=1cm, itemsep=5pt]
    \item Temps de réponse moyen : < \textbf{2 secondes}
    \item Temps de réponse au 95ème percentile : < \textbf{5 secondes}
    \item Endpoints critiques (détection) : < \textbf{500 ms}
\end{itemize}

\vspace{0.3cm}

\textbf{NF02 : Traitement Vidéo en Temps Réel}

\begin{itemize}[leftmargin=1cm, itemsep=5pt]
    \item Fréquence de traitement : $\geq$ \textbf{25 FPS} sur GPU
    \item Fréquence minimale acceptable : $\geq$ \textbf{15 FPS} sur CPU
    \item Latence bout en bout : < \textbf{500 ms}
    \item Résolution supportée : jusqu'à \textbf{1080p}
\end{itemize}

\vspace{0.3cm}

\textbf{NF03 : Capacité de Traitement}

\begin{itemize}[leftmargin=1cm, itemsep=5pt]
    \item Support de \textbf{10 utilisateurs} simultanés minimum
    \item Traitement de \textbf{100 images/heure} pour classification
    \item Scan satellite de \textbf{8 zones} en moins de 10 minutes
\end{itemize}
\end{techbox}

\vspace{0.5cm}

\begin{center}
\begin{tikzpicture}
    % Graphique de performance
    \begin{axis}[
        width=12cm,
        height=6cm,
        xlabel={Composant},
        ylabel={Temps de réponse (ms)},
        ymin=0, ymax=3000,
        symbolic x coords={YOLOv8, MobileNet, CAM, API REST, Satellite, Notification},
        xtick=data,
        xticklabel style={rotate=45, anchor=east, font=\footnotesize},
        ylabel style={font=\small},
        xlabel style={font=\small},
        bar width=15pt,
        nodes near coords,
        nodes near coords style={font=\tiny},
        ybar,
        enlarge x limits=0.15,
    ]
    \addplot[fill=primaryGreen!70, draw=primaryGreen] coordinates {
        (YOLOv8, 33)
        (MobileNet, 150)
        (CAM, 200)
        (API REST, 100)
        (Satellite, 2000)
        (Notification, 500)
    };
    \end{axis}
    \node[font=\bfseries\small, text=darkGreen] at (6, -1) {Temps de Réponse Cibles par Composant};
\end{tikzpicture}
\end{center}

\subsection{Fiabilité}

La fiabilité du système est essentielle dans un contexte où des vies et des biens sont en jeu. Le système doit fonctionner de manière continue et prévisible.

\begin{greenbox}[\faShieldAlt\ Exigences de Fiabilité]

\textbf{NF04 : Disponibilité}

Le système doit garantir une disponibilité de \textbf{99\%} minimum, soit moins de \textbf{87.6 heures} d'indisponibilité par an. Cette disponibilité couvre :
\begin{itemize}[leftmargin=0.5cm, itemsep=3pt]
    \item Le backend API (FastAPI)
    \item Le frontend web (React)
    \item Les services de notification (Telegram, Email)
\end{itemize}

\vspace{0.3cm}

\textbf{NF05 : Tolérance aux Pannes}

\begin{itemize}[leftmargin=0.5cm, itemsep=5pt]
    \item \textbf{Dégradation gracieuse :} En cas d'indisponibilité d'un service externe (NASA, Sentinel), le reste du système continue de fonctionner
    \item \textbf{Retry automatique :} Les appels échoués sont automatiquement relancés (3 tentatives avec backoff exponentiel)
    \item \textbf{Circuit breaker :} Protection contre les cascades de pannes
    \item \textbf{Logs complets :} Enregistrement de toutes les erreurs pour diagnostic
\end{itemize}

\vspace{0.3cm}

\textbf{NF06 : Précision des Modèles IA}

\begin{table}[H]
\centering
\rowcolors{2}{mintGreen!30}{white}
\begin{tabular}{l c c}
\toprule
\rowcolor{primaryGreen}
\textcolor{white}{\textbf{Modèle}} & \textcolor{white}{\textbf{Précision Cible}} & \textcolor{white}{\textbf{Faux Positifs}} \\
\midrule
YOLOv8 (Détection) & $\geq$ 85\% & < 5\% \\
MobileNetV2 (Classification) & $\geq$ 97\% & < 3\% \\
CAM (Satellite) & $\geq$ 90\% & < 10\% \\
\bottomrule
\end{tabular}
\end{table}
\end{greenbox}

\subsection{Sécurité}

Bien que le système AI Sentinel soit principalement orienté vers la surveillance environnementale, les aspects de sécurité informatique restent importants pour protéger les données et garantir l'intégrité du système.

\begin{alertbox}{Exigences de Sécurité}

\textbf{NF07 : Authentification API}

\begin{itemize}[leftmargin=0.5cm, itemsep=5pt]
    \item Protection des endpoints API par clés d'authentification
    \item Rotation régulière des clés API externes (NASA, Sentinel Hub)
    \item Stockage sécurisé des credentials dans des variables d'environnement
    \item Rate limiting pour prévenir les abus
\end{itemize}

\vspace{0.3cm}

\textbf{NF08 : Protection des Données}

\begin{itemize}[leftmargin=0.5cm, itemsep=5pt]
    \item Chiffrement HTTPS pour toutes les communications
    \item Pas de stockage de données personnelles sensibles
    \item Anonymisation des logs si nécessaire
    \item Conformité RGPD pour les données utilisateur
\end{itemize}

\vspace{0.3cm}

\textbf{NF09 : Intégrité du Système}

\begin{itemize}[leftmargin=0.5cm, itemsep=5pt]
    \item Validation de toutes les entrées utilisateur (images, paramètres)
    \item Protection contre les injections (SQL, command injection)
    \item Limitation de la taille des fichiers uploadés
    \item Sandboxing du traitement des fichiers
\end{itemize}
\end{alertbox}

\subsection{Maintenabilité}

La maintenabilité garantit que le système peut évoluer et être corrigé efficacement au fil du temps.

\begin{infobox}{Exigences de Maintenabilité}

\textbf{NF10 : Code Modulaire}

\begin{itemize}[leftmargin=0.5cm, itemsep=5pt]
    \item Architecture en microservices logiques (YoloService, FirmsService, PredictionService, etc.)
    \item Séparation claire des responsabilités (Single Responsibility Principle)
    \item Injection de dépendances pour faciliter les tests
    \item Interfaces bien définies entre les modules
\end{itemize}

\vspace{0.3cm}

\textbf{NF11 : Documentation}

\begin{itemize}[leftmargin=0.5cm, itemsep=5pt]
    \item Documentation API Swagger/OpenAPI automatique
    \item Commentaires de code pour les sections complexes
    \item README avec instructions d'installation et d'utilisation
    \item Documentation technique d'architecture
\end{itemize}

\vspace{0.3cm}

\textbf{NF12 : Testabilité}

\begin{itemize}[leftmargin=0.5cm, itemsep=5pt]
    \item Tests unitaires pour les fonctions critiques
    \item Tests d'intégration pour les flux principaux
    \item Couverture de code cible : > 70\%
    \item Environnement de test isolé
\end{itemize}
\end{infobox}

\subsection{Ergonomie}

L'ergonomie et l'expérience utilisateur sont essentielles pour garantir l'adoption du système par les opérateurs terrain qui doivent pouvoir réagir rapidement.

\begin{greenbox}[\faDesktop\ Exigences d'Ergonomie]

\textbf{NF13 : Interface Intuitive}

\begin{itemize}[leftmargin=0.5cm, itemsep=5pt]
    \item Navigation claire avec menu principal accessible
    \item Actions principales accessibles en \textbf{moins de 3 clics}
    \item Feedback visuel immédiat pour toutes les actions
    \item Messages d'erreur explicites et actionables
    \item Thème visuel cohérent (palette verte nature)
\end{itemize}

\vspace{0.3cm}

\textbf{NF14 : Design Responsive}

\begin{itemize}[leftmargin=0.5cm, itemsep=5pt]
    \item Compatibilité : Desktop (1920×1080), Tablette (768px), Mobile (375px)
    \item Adaptation automatique de la mise en page
    \item Carte interactive utilisable sur écran tactile
    \item Performances optimisées pour les appareils mobiles
\end{itemize}

\vspace{0.3cm}

\textbf{NF15 : Accessibilité}

\begin{itemize}[leftmargin=0.5cm, itemsep=5pt]
    \item Contraste suffisant pour lisibilité
    \item Labels pour tous les éléments de formulaire
    \item Navigation au clavier possible
    \item Compatibilité avec les lecteurs d'écran (ARIA)
\end{itemize}
\end{greenbox}

\vspace{0.5cm}

% Tableau récapitulatif des besoins non fonctionnels
\begin{table}[H]
\centering
\caption{Récapitulatif des Besoins Non Fonctionnels}
\label{tab:bnf}
\rowcolors{2}{mintGreen!30}{white}
\begin{tabular}{l l p{6cm} l}
\toprule
\rowcolor{primaryGreen}
\textcolor{white}{\textbf{ID}} & \textcolor{white}{\textbf{Catégorie}} & \textcolor{white}{\textbf{Description}} & \textcolor{white}{\textbf{Priorité}} \\
\midrule
NF01 & Performance & Temps de réponse API < 2s & Critique \\
NF02 & Performance & Traitement vidéo $\geq$ 25 FPS & Critique \\
NF03 & Performance & 10 utilisateurs simultanés & Haute \\
NF04 & Fiabilité & Disponibilité 99\% & Critique \\
NF05 & Fiabilité & Tolérance aux pannes & Haute \\
NF06 & Fiabilité & Précision modèles $\geq$ 85\% & Critique \\
NF07 & Sécurité & Authentification API & Haute \\
NF08 & Sécurité & Protection des données & Haute \\
NF09 & Sécurité & Validation des entrées & Haute \\
NF10 & Maintenabilité & Code modulaire & Moyenne \\
NF11 & Maintenabilité & Documentation complète & Moyenne \\
NF12 & Maintenabilité & Couverture tests > 70\% & Moyenne \\
NF13 & Ergonomie & Interface intuitive & Haute \\
NF14 & Ergonomie & Design responsive & Haute \\
NF15 & Ergonomie & Accessibilité & Moyenne \\
\bottomrule
\end{tabular}
\end{table}

% ============================================
\newpage
\section{Méthodologie de Développement}
% ============================================

Le choix d'une méthodologie de développement adaptée est crucial pour la réussite d'un projet logiciel. Dans le cadre d'AI Sentinel, nous avons opté pour une approche \textbf{Agile} avec le framework \textbf{Scrum}, particulièrement adapté aux projets innovants nécessitant flexibilité et itérations rapides.

\subsection{Choix de la Méthodologie : Agile/Scrum}

\subsubsection{Justification du Choix}

Le développement d'un système de détection d'incendies par IA présente plusieurs caractéristiques qui oriententnaturellement vers une méthodologie agile :

\begin{greenbox}[\faList\ Pourquoi Agile/Scrum ?]

\textbf{Complexité Technique}

Le projet intègre des technologies diverses (IA, satellite, temps réel) dont l'intégration peut révéler des défis imprévus. Scrum permet d'adapter le plan au fil des découvertes.

\vspace{0.3cm}

\textbf{Besoins Évolutifs}

Les exigences des utilisateurs (autorités, gestionnaires forestiers) peuvent évoluer à mesure qu'ils découvrent les possibilités du système. Les sprints courts permettent d'intégrer leurs retours.

\vspace{0.3cm}

\textbf{Livraisons Incrémentales}

Chaque module (détection temps réel, satellite, prédiction) peut être développé et livré indépendamment, permettant des tests terrain précoces.

\vspace{0.3cm}

\textbf{Gestion des Risques}

Les revues de sprint régulières permettent d'identifier et de traiter rapidement les problèmes techniques ou fonctionnels.
\end{greenbox}

\subsubsection{Principes Scrum Appliqués}

\begin{center}
\begin{tikzpicture}[
    role/.style={ellipse, draw=primaryGreen, line width=2pt, fill=mintGreen, text width=2cm, align=center, font=\small, minimum height=1.5cm},
    artifact/.style={rectangle, rounded corners=8pt, draw=accentTeal, line width=1.5pt, fill=skyBlue!20, text width=2.5cm, align=center, font=\small, minimum height=1.5cm},
    event/.style={rectangle, rounded corners=8pt, draw=leafGreen, line width=1.5pt, fill=leafGreen!20, text width=2cm, align=center, font=\small, minimum height=1.2cm}
]
    % Titre
    \node[font=\Large\bfseries, text=darkGreen] at (6, 4) {Framework Scrum appliqué à AI Sentinel};
    
    % Rôles
    \node[role] (po) at (0, 2) {Product\\Owner};
    \node[role] (sm) at (0, 0) {Scrum\\Master};
    \node[role] (team) at (0, -2) {Dev\\Team};
    
    % Artefacts
    \node[artifact] (pb) at (4, 2) {Product\\Backlog};
    \node[artifact] (sb) at (4, 0) {Sprint\\Backlog};
    \node[artifact] (inc) at (4, -2) {Incrément\\Produit};
    
    % Événements
    \node[event] (sprint) at (8, 2) {Sprint\\(2 sem.)};
    \node[event] (plan) at (8, 0.5) {Sprint\\Planning};
    \node[event] (daily) at (8, -0.8) {Daily\\Standup};
    \node[event] (review) at (8, -2) {Sprint\\Review};
    
    % Connexions
    \draw[->, >=stealth, primaryGreen, line width=1pt] (po) -- (pb);
    \draw[->, >=stealth, primaryGreen, line width=1pt] (pb) -- (sb);
    \draw[->, >=stealth, primaryGreen, line width=1pt] (sb) -- (inc);
    \draw[->, >=stealth, leafGreen, line width=1pt] (sb) -- (sprint);
    
\end{tikzpicture}
\end{center}

\subsection{Planning des Sprints}

Le projet AI Sentinel a été planifié sur \textbf{6 sprints} de 2 semaines chacun, soit une durée totale de 12 semaines.

\begin{table}[H]
\centering
\caption{Planning des Sprints AI Sentinel}
\label{tab:sprints}
\rowcolors{2}{mintGreen!30}{white}
\begin{tabular}{c p{4cm} p{5cm} c}
\toprule
\rowcolor{primaryGreen}
\textcolor{white}{\textbf{Sprint}} & \textcolor{white}{\textbf{Objectif}} & \textcolor{white}{\textbf{Livrables}} & \textcolor{white}{\textbf{Durée}} \\
\midrule
1 & Setup \& Architecture & Environnement dev, structure projet, API base & 2 sem. \\
2 & Module Détection & YOLOv8 intégré, détection temps réel & 2 sem. \\
3 & Module Classification & MobileNetV2, upload images & 2 sem. \\
4 & Module Satellite & Sentinel Hub, NASA FIRMS & 2 sem. \\
5 & Module Prédiction & Algorithme propagation, carte & 2 sem. \\
6 & Notifications \& Polish & Email, Telegram, UI final & 2 sem. \\
\bottomrule
\end{tabular}
\end{table}

\vspace{0.3cm}

\begin{center}
\begin{tikzpicture}
    % Timeline
    \draw[line width=3pt, primaryGreen] (0,0) -- (14,0);
    
    % Sprints
    \foreach \x/\s/\label in {0/1/Setup, 2.33/2/Détection, 4.66/3/Classification, 7/4/Satellite, 9.33/5/Prédiction, 11.66/6/Notif.} {
        \fill[primaryGreen] (\x, 0) circle (6pt);
        \node[above, font=\footnotesize\bfseries] at (\x, 0.3) {S\s};
        \node[below, font=\tiny, text width=1.5cm, align=center] at (\x, -0.3) {\label};
    }
    \fill[leafGreen] (14, 0) circle (8pt);
    \node[above, font=\footnotesize\bfseries] at (14, 0.3) {Release};
    
    % Durées
    \node[font=\tiny] at (1.16, -1) {2 sem.};
    \node[font=\tiny] at (3.5, -1) {2 sem.};
    \node[font=\tiny] at (5.83, -1) {2 sem.};
    \node[font=\tiny] at (8.16, -1) {2 sem.};
    \node[font=\tiny] at (10.5, -1) {2 sem.};
    \node[font=\tiny] at (12.83, -1) {2 sem.};
\end{tikzpicture}
\end{center}

\subsection{Outils de Gestion}

La gestion efficace du projet repose sur un ensemble d'outils complémentaires permettant le suivi des tâches, la gestion du code source et la collaboration d'équipe.

\begin{techbox}{Stack d'Outils de Gestion de Projet}

\textbf{\faTrello\ Trello --- Gestion des Tâches}

\begin{itemize}[leftmargin=0.5cm, itemsep=3pt]
    \item Tableau Kanban avec colonnes : Backlog, To Do, In Progress, Review, Done
    \item Cartes pour chaque User Story avec checklist de sous-tâches
    \item Labels par module (Détection, Satellite, UI, etc.)
    \item Power-ups pour estimation en points et burndown
\end{itemize}

\vspace{0.3cm}

\textbf{\faGitAlt\ Git/GitHub --- Gestion du Code}

\begin{itemize}[leftmargin=0.5cm, itemsep=3pt]
    \item Repository centralisé avec branches par feature
    \item Convention de nommage : \texttt{feature/module-description}
    \item Pull Requests obligatoires pour merge sur \texttt{main}
    \item Code review par au moins 1 développeur
    \item GitHub Actions pour CI/CD automatisé
\end{itemize}

\vspace{0.3cm}

\textbf{\faSlack\ Communication}

\begin{itemize}[leftmargin=0.5cm, itemsep=3pt]
    \item Channels dédiés par sujet (dev, bugs, releases)
    \item Intégrations avec GitHub et Trello
    \item Daily standups asynchrones
\end{itemize}

\vspace{0.3cm}

\textbf{\faFileAlt\ Documentation}

\begin{itemize}[leftmargin=0.5cm, itemsep=3pt]
    \item Confluence/Notion pour documentation technique
    \item Swagger pour documentation API
    \item README.md dans chaque module
\end{itemize}
\end{techbox}

\vspace{0.5cm}

% Workflow Git
\begin{center}
\begin{tikzpicture}[
    branch/.style={rectangle, rounded corners=5pt, draw=accentTeal, line width=1.5pt, fill=skyBlue!20, text width=2cm, minimum height=1cm, align=center, font=\small},
    commit/.style={circle, draw=primaryGreen, line width=1.5pt, fill=mintGreen, minimum size=0.6cm},
    arrow/.style={->, >=stealth, line width=1.5pt}
]
    % Branches
    \node[font=\bfseries, text=darkGreen] at (0, 2) {main};
    \node[font=\bfseries, text=accentTeal] at (0, 0.5) {develop};
    \node[font=\bfseries, text=leafGreen] at (0, -1) {feature/x};
    
    % Commits main
    \foreach \x in {2, 6, 10, 14} {
        \node[commit] at (\x, 2) {};
    }
    \draw[darkGreen, line width=2pt] (1, 2) -- (15, 2);
    
    % Commits develop
    \foreach \x in {2, 4, 6, 8, 10, 12, 14} {
        \node[commit, fill=skyBlue!50] at (\x, 0.5) {};
    }
    \draw[accentTeal, line width=2pt] (1, 0.5) -- (15, 0.5);
    
    % Feature branch
    \foreach \x in {4, 5, 6} {
        \node[commit, fill=leafGreen!50] at (\x, -1) {};
    }
    \draw[leafGreen, line width=1.5pt] (3, 0.5) -- (4, -1) -- (6, -1) -- (7, 0.5);
    
    % Merge arrows
    \draw[arrow, accentTeal] (6, 0.5) -- (6, 1.8);
    \draw[arrow, accentTeal] (14, 0.5) -- (14, 1.8);
    
    \node[font=\bfseries\small, text=darkGreen] at (8, -2.5) {Workflow Git Flow};
\end{tikzpicture}
\end{center}

% ============================================
% CHAPITRE III - PARTIE 4 : MODÉLISATION UML
% ============================================

\newpage
\section{Modélisation UML}

La modélisation UML (Unified Modeling Language) permet de visualiser l'architecture et le comportement du système de manière standardisée. Cette section présente les différents diagrammes qui décrivent AI Sentinel sous ses multiples facettes : fonctionnelle, structurelle et comportementale.

\subsection{Diagramme de Cas d'Utilisation Global}

Le diagramme de cas d'utilisation global offre une vue d'ensemble des fonctionnalités du système et des interactions entre les acteurs et le système.

\vspace{0.5cm}
\begin{center}
\begin{tikzpicture}[
    actor/.style={font=\small},
    usecase/.style={ellipse, draw=primaryGreen, line width=1.5pt, fill=mintGreen!50, text width=2.5cm, align=center, font=\footnotesize, minimum height=1cm},
    system/.style={rectangle, draw=darkGreen, line width=2pt, rounded corners=10pt, minimum width=11cm, minimum height=10cm}
]
    % Système
    \node[system, label={[font=\Large\bfseries, text=darkGreen]above:Système AI Sentinel}] (sys) at (5, 0) {};
    
    % Acteurs
    \node[actor] (user) at (-2, 1) {\includegraphics[width=1cm]{example-image}};
    \node[font=\small\bfseries, below] at (-2, 0.3) {Utilisateur};
    
    \node[actor] (admin) at (12, 1) {\includegraphics[width=1cm]{example-image}};
    \node[font=\small\bfseries, below] at (12, 0.3) {Admin};
    
    \node[actor] (ext) at (5, -6.5) {};
    \node[font=\small\bfseries, text=accentTeal] at (5, -6.5) {\faCloud\ Services Externes};
    
    % Cas d'utilisation - Colonne gauche
    \node[usecase] (uc1) at (2, 3) {Détection\\temps réel};
    \node[usecase] (uc2) at (2, 1.5) {Upload\\média};
    \node[usecase] (uc3) at (2, 0) {Surveillance\\satellite};
    \node[usecase] (uc4) at (2, -1.5) {Visualiser\\carte};
    
    % Cas d'utilisation - Colonne droite
    \node[usecase] (uc5) at (8, 3) {Prédiction\\propagation};
    \node[usecase] (uc6) at (8, 1.5) {Configurer\\alertes};
    \node[usecase] (uc7) at (8, 0) {Consulter\\historique};
    \node[usecase] (uc8) at (8, -1.5) {Exporter\\rapports};
    
    % Cas Admin
    \node[usecase, fill=skyBlue!30] (uca1) at (5, -3) {Gérer\\monitoring};
    \node[usecase, fill=skyBlue!30] (uca2) at (5, -4.5) {Configurer\\système};
    
    % Connexions Utilisateur
    \draw[primaryGreen, line width=1pt] (user) -- (uc1);
    \draw[primaryGreen, line width=1pt] (user) -- (uc2);
    \draw[primaryGreen, line width=1pt] (user) -- (uc3);
    \draw[primaryGreen, line width=1pt] (user) -- (uc4);
    \draw[primaryGreen, line width=1pt] (-0.5, 1) -- (uc5);
    \draw[primaryGreen, line width=1pt] (-0.5, 1) -- (uc7);
    
    % Connexions Admin
    \draw[accentTeal, line width=1pt] (admin) -- (uc6);
    \draw[accentTeal, line width=1pt] (admin) -- (uc8);
    \draw[accentTeal, line width=1pt] (admin) -- (uca1);
    \draw[accentTeal, line width=1pt] (admin) -- (uca2);
    
    % Connexions Services Externes
    \draw[textGray, line width=1pt, dashed] (uc3) -- (ext);
    \draw[textGray, line width=1pt, dashed] (uc5) -- (ext);
    \draw[textGray, line width=1pt, dashed] (uca1) -- (ext);
    
\end{tikzpicture}
\end{center}

\vspace{0.5cm}

\begin{greenbox}[\faListOl\ Légende des Cas d'Utilisation]
\begin{minipage}[t]{0.48\textwidth}
\textbf{Cas Utilisateur :}
\begin{itemize}[leftmargin=0.5cm, itemsep=3pt]
    \item Détection temps réel (vidéo/webcam)
    \item Upload média (images/vidéos)
    \item Surveillance satellite
    \item Visualiser carte interactive
    \item Prédiction de propagation
    \item Consulter historique
\end{itemize}
\end{minipage}
\hfill
\begin{minipage}[t]{0.48\textwidth}
\textbf{Cas Administrateur :}
\begin{itemize}[leftmargin=0.5cm, itemsep=3pt]
    \item Configurer alertes
    \item Exporter rapports
    \item Gérer monitoring satellite
    \item Configurer système
\end{itemize}
\end{minipage}
\end{greenbox}

\subsection{Diagrammes de Cas d'Utilisation Détaillés}

\subsubsection{UC01 : Détection en Temps Réel}

\begin{center}
\begin{tikzpicture}[
    usecase/.style={ellipse, draw=primaryGreen, line width=1.5pt, fill=mintGreen!50, text width=2.2cm, align=center, font=\footnotesize, minimum height=0.9cm},
    extends/.style={->, >=stealth, dashed, line width=1pt, textGray},
    includes/.style={->, >=stealth, dashed, line width=1pt, accentTeal}
]
    % Acteur
    \node (user) at (-3, 0) {\faUser};
    \node[font=\small\bfseries, below] at (-3, -0.5) {Utilisateur};
    
    % Cas principal
    \node[usecase, fill=leafGreen!30, text width=3cm] (main) at (3, 0) {Détection\\Temps Réel};
    
    % Sous-cas
    \node[usecase] (uc1) at (7, 2) {Activer\\webcam};
    \node[usecase] (uc2) at (7, 0) {Visualiser\\flux};
    \node[usecase] (uc3) at (7, -2) {Afficher\\détections};
    \node[usecase] (uc4) at (11, 1) {Envoyer\\alerte};
    \node[usecase] (uc5) at (11, -1) {Sauvegarder\\capture};
    
    % Connexions
    \draw[primaryGreen, line width=1.5pt] (user) -- (main);
    \draw[includes] (main) -- node[above, font=\tiny] {<<include>>} (uc1);
    \draw[includes] (main) -- node[above, font=\tiny] {<<include>>} (uc2);
    \draw[includes] (main) -- node[below, font=\tiny] {<<include>>} (uc3);
    \draw[extends] (uc4) -- node[above, font=\tiny] {<<extends>>} (uc3);
    \draw[extends] (uc5) -- node[below, font=\tiny] {<<extends>>} (uc3);
    
\end{tikzpicture}
\end{center}

\subsubsection{UC02 : Surveillance Satellite}

\begin{center}
\begin{tikzpicture}[
    usecase/.style={ellipse, draw=accentTeal, line width=1.5pt, fill=skyBlue!30, text width=2.2cm, align=center, font=\footnotesize, minimum height=0.9cm},
    extends/.style={->, >=stealth, dashed, line width=1pt, textGray},
    includes/.style={->, >=stealth, dashed, line width=1pt, primaryGreen}
]
    % Acteurs
    \node (user) at (-3, 1) {\faUser};
    \node[font=\small\bfseries, below] at (-3, 0.5) {Utilisateur};
    
    \node (admin) at (-3, -2) {\faUserCog};
    \node[font=\small\bfseries, below] at (-3, -2.5) {Admin};
    
    % Cas principal
    \node[usecase, fill=accentTeal!30, text width=3cm] (main) at (3, 0) {Surveillance\\Satellite};
    
    % Sous-cas
    \node[usecase] (uc1) at (7, 2.5) {Sélectionner\\zone};
    \node[usecase] (uc2) at (7, 1) {Récupérer\\image Sentinel};
    \node[usecase] (uc3) at (7, -0.5) {Analyser\\par CAM};
    \node[usecase] (uc4) at (7, -2) {Afficher\\résultat};
    \node[usecase] (uc5) at (11, 0) {Démarrer\\scan auto};
    
    % Connexions
    \draw[primaryGreen, line width=1.5pt] (user) -- (main);
    \draw[primaryGreen, line width=1.5pt] (admin) -- (main);
    \draw[includes] (main) -- node[above, font=\tiny] {<<include>>} (uc1);
    \draw[includes] (main) -- node[above, font=\tiny] {<<include>>} (uc2);
    \draw[includes] (main) -- node[above, font=\tiny] {<<include>>} (uc3);
    \draw[includes] (main) -- node[below, font=\tiny] {<<include>>} (uc4);
    \draw[extends] (uc5) -- node[above, font=\tiny] {<<extends>>} (main);
    
\end{tikzpicture}
\end{center}

\subsection{Description Textuelle des Cas d'Utilisation}

\begin{table}[H]
\centering
\caption{UC01 : Détection en Temps Réel}
\label{tab:uc01}
\rowcolors{2}{mintGreen!20}{white}
\begin{tabular}{>{\bfseries}p{3cm} p{10cm}}
\toprule
\rowcolor{primaryGreen}
\multicolumn{2}{l}{\textcolor{white}{\textbf{UC01 : Détection en Temps Réel}}} \\
\midrule
Acteur principal & Utilisateur (Opérateur de surveillance) \\
Acteurs secondaires & Système de notification, Modèle YOLOv8 \\
Description & Permet à l'utilisateur de détecter en temps réel la présence de feu ou de fumée via une webcam ou caméra connectée \\
Préconditions & \begin{itemize}[leftmargin=0.3cm, topsep=0pt, itemsep=0pt]
    \item Caméra/webcam connectée et fonctionnelle
    \item Modèle YOLOv8 chargé en mémoire
    \item Connexion au backend établie
\end{itemize} \\
\midrule
\multicolumn{2}{l}{\textbf{\textcolor{primaryGreen}{Scénario nominal :}}} \\
\multicolumn{2}{p{13cm}}{
\begin{enumerate}[leftmargin=0.5cm, topsep=0pt, itemsep=2pt]
    \item L'utilisateur accède à la page /realtime
    \item L'utilisateur clique sur "Démarrer la détection"
    \item Le système active le flux vidéo de la webcam
    \item Le système affiche le flux vidéo en temps réel
    \item Le système analyse chaque frame avec YOLOv8
    \item Si feu/fumée détecté : affichage des bounding boxes
    \item Le système génère une alerte visuelle
    \item [Optionnel] Le système envoie une notification
\end{enumerate}
} \\
\midrule
Postconditions & \begin{itemize}[leftmargin=0.3cm, topsep=0pt, itemsep=0pt]
    \item Flux vidéo affiché avec annotations
    \item Alertes générées si détection positive
    \item Logs de détection enregistrés
\end{itemize} \\
Exceptions & \begin{itemize}[leftmargin=0.3cm, topsep=0pt, itemsep=0pt]
    \item E1 : Caméra non accessible → Message d'erreur
    \item E2 : Modèle non chargé → Tentative de rechargement
\end{itemize} \\
\bottomrule
\end{tabular}
\end{table}

\vspace{0.5cm}

\begin{table}[H]
\centering
\caption{UC02 : Upload et Analyse d'Image}
\label{tab:uc02}
\rowcolors{2}{mintGreen!20}{white}
\begin{tabular}{>{\bfseries}p{3cm} p{10cm}}
\toprule
\rowcolor{primaryGreen}
\multicolumn{2}{l}{\textcolor{white}{\textbf{UC02 : Upload et Analyse d'Image}}} \\
\midrule
Acteur principal & Utilisateur \\
Acteurs secondaires & Modèle MobileNetV2 \\
Description & Permet d'uploader une image et d'obtenir sa classification (Fire/Smoke/Non-Fire) \\
Préconditions & \begin{itemize}[leftmargin=0.3cm, topsep=0pt, itemsep=0pt]
    \item Image au format valide (JPEG, PNG)
    \item Taille < 10 MB
\end{itemize} \\
\midrule
\multicolumn{2}{l}{\textbf{\textcolor{primaryGreen}{Scénario nominal :}}} \\
\multicolumn{2}{p{13cm}}{
\begin{enumerate}[leftmargin=0.5cm, topsep=0pt, itemsep=2pt]
    \item L'utilisateur accède à la page /upload
    \item L'utilisateur sélectionne une image (drag-and-drop ou bouton)
    \item L'utilisateur clique sur "Analyser"
    \item Le système uploade l'image vers le backend
    \item Le backend prétraite l'image (224×224, normalisation)
    \item Le modèle MobileNetV2 effectue la classification
    \item Le système affiche le résultat : classe + score de confiance
\end{enumerate}
} \\
\midrule
Postconditions & Résultat de classification affiché avec probabilités \\
Exceptions & \begin{itemize}[leftmargin=0.3cm, topsep=0pt, itemsep=0pt]
    \item E1 : Format invalide → Message d'erreur avec formats acceptés
    \item E2 : Image trop grande → Message avec taille maximale
\end{itemize} \\
\bottomrule
\end{tabular}
\end{table}

\subsection{Diagramme de Séquence - Détection Temps Réel}

Ce diagramme illustre les interactions entre les composants lors d'une session de détection en temps réel.

\vspace{0.5cm}
\begin{center}
\begin{tikzpicture}[
    lifeline/.style={dashed, line width=1pt, textGray},
    message/.style={->, >=stealth, line width=1pt, primaryGreen},
    return/.style={->, >=stealth, dashed, line width=1pt, accentTeal},
    actor/.style={rectangle, draw=primaryGreen, line width=1.5pt, fill=mintGreen, minimum width=1.8cm, minimum height=0.8cm, align=center, font=\footnotesize}
]
    % Acteurs
    \node[actor] (user) at (0, 0) {Utilisateur};
    \node[actor] (front) at (3.5, 0) {Frontend\\React};
    \node[actor] (back) at (7, 0) {Backend\\FastAPI};
    \node[actor] (yolo) at (10.5, 0) {YOLOv8\\Model};
    \node[actor] (notif) at (14, 0) {Telegram\\Bot};
    
    % Lifelines
    \draw[lifeline] (0, -0.5) -- (0, -11);
    \draw[lifeline] (3.5, -0.5) -- (3.5, -11);
    \draw[lifeline] (7, -0.5) -- (7, -11);
    \draw[lifeline] (10.5, -0.5) -- (10.5, -11);
    \draw[lifeline] (14, -0.5) -- (14, -11);
    
    % Messages
    \draw[message] (0, -1.5) -- node[above, font=\tiny] {1: Accéder /realtime} (3.5, -1.5);
    \draw[return] (3.5, -2) -- node[above, font=\tiny] {Page chargée} (0, -2);
    
    \draw[message] (0, -2.8) -- node[above, font=\tiny] {2: Clic "Démarrer"} (3.5, -2.8);
    \draw[message] (3.5, -3.3) -- node[above, font=\tiny] {3: GET /video\_feed} (7, -3.3);
    
    % Boucle de détection
    \draw[darkGreen, line width=1pt] (6.2, -4) rectangle (11.3, -7.5);
    \node[font=\tiny\bfseries, text=darkGreen] at (7.5, -3.8) {loop [pour chaque frame]};
    
    \draw[message] (7, -4.5) -- node[above, font=\tiny] {4: capture\_frame()} (7, -4.5);
    \draw[message] (7, -5) -- node[above, font=\tiny] {5: predict(frame)} (10.5, -5);
    \draw[return] (10.5, -5.5) -- node[above, font=\tiny] {boxes, scores} (7, -5.5);
    \draw[message] (7, -6) -- node[above, font=\tiny] {6: draw\_boxes()} (7, -6);
    \draw[message] (7, -7) -- node[above, font=\tiny] {7: encode MJPEG} (7, -7);
    
    % Condition alerte
    \draw[alertOrange, line width=1pt] (6.2, -8) rectangle (14.8, -9.5);
    \node[font=\tiny\bfseries, text=alertOrange] at (8, -7.8) {alt [si Fire détecté]};
    
    \draw[message] (7, -8.5) -- node[above, font=\tiny] {8: send\_alert()} (14, -8.5);
    \draw[return] (14, -9) -- node[above, font=\tiny] {OK} (7, -9);
    
    % Retour stream
    \draw[return] (7, -10) -- node[above, font=\tiny] {MJPEG stream} (3.5, -10);
    \draw[return] (3.5, -10.5) -- node[above, font=\tiny] {Affichage vidéo} (0, -10.5);
    
\end{tikzpicture}
\end{center}

\subsection{Diagramme de Séquence - Surveillance Satellite}

\vspace{0.5cm}
\begin{center}
\begin{tikzpicture}[
    lifeline/.style={dashed, line width=1pt, textGray},
    message/.style={->, >=stealth, line width=1pt, accentTeal},
    return/.style={->, >=stealth, dashed, line width=1pt, primaryGreen},
    actor/.style={rectangle, draw=accentTeal, line width=1.5pt, fill=skyBlue!20, minimum width=1.5cm, minimum height=0.8cm, align=center, font=\footnotesize}
]
    % Acteurs
    \node[actor] (user) at (0, 0) {Admin};
    \node[actor] (front) at (2.8, 0) {Frontend};
    \node[actor] (back) at (5.6, 0) {Backend};
    \node[actor] (sent) at (8.4, 0) {Sentinel\\Hub};
    \node[actor] (cam) at (11.2, 0) {CAM\\Model};
    \node[actor] (notif) at (14, 0) {Email\\Service};
    
    % Lifelines
    \draw[lifeline] (0, -0.5) -- (0, -10);
    \draw[lifeline] (2.8, -0.5) -- (2.8, -10);
    \draw[lifeline] (5.6, -0.5) -- (5.6, -10);
    \draw[lifeline] (8.4, -0.5) -- (8.4, -10);
    \draw[lifeline] (11.2, -0.5) -- (11.2, -10);
    \draw[lifeline] (14, -0.5) -- (14, -10);
    
    % Messages
    \draw[message] (0, -1.2) -- node[above, font=\tiny] {1: Démarrer scan} (2.8, -1.2);
    \draw[message] (2.8, -1.7) -- node[above, font=\tiny] {2: POST /start\_monitoring} (5.6, -1.7);
    
    % Boucle zones
    \draw[darkGreen, line width=1pt] (4.8, -2.3) rectangle (14.8, -8);
    \node[font=\tiny\bfseries, text=darkGreen] at (6.5, -2.1) {loop [pour chaque zone]};
    
    \draw[message] (5.6, -3) -- node[above, font=\tiny] {3: get\_image(zone)} (8.4, -3);
    \draw[return] (8.4, -3.5) -- node[above, font=\tiny] {image satellite} (5.6, -3.5);
    
    \draw[message] (5.6, -4.2) -- node[above, font=\tiny] {4: analyze(image)} (11.2, -4.2);
    \draw[return] (11.2, -4.7) -- node[above, font=\tiny] {prediction, heatmap} (5.6, -4.7);
    
    % Condition feu
    \draw[moroccanRed, line width=1pt] (4.8, -5.3) rectangle (14.8, -7.5);
    \node[font=\tiny\bfseries, text=moroccanRed] at (6.8, -5.1) {alt [si Fire détecté]};
    
    \draw[message] (5.6, -6) -- node[above, font=\tiny] {5: send\_email(details)} (14, -6);
    \draw[return] (14, -6.5) -- node[above, font=\tiny] {sent OK} (5.6, -6.5);
    \draw[message] (5.6, -7) -- node[above, font=\tiny] {6: log\_detection()} (5.6, -7);
    
    % Retour
    \draw[return] (5.6, -8.5) -- node[above, font=\tiny] {scan results} (2.8, -8.5);
    \draw[return] (2.8, -9) -- node[above, font=\tiny] {Affichage rapport} (0, -9);
    
\end{tikzpicture}
\end{center}

\newpage
\subsection{Diagramme de Classes}

Le diagramme de classes présente la structure statique du système, avec les principales classes et leurs relations.

\vspace{0.5cm}
\begin{center}
\begin{tikzpicture}[
    class/.style={rectangle, draw=primaryGreen, line width=1.5pt, fill=mintGreen!30, rounded corners=3pt, minimum width=4cm, align=left, font=\footnotesize},
    inherit/.style={->, >=open triangle 60, line width=1pt, primaryGreen},
    compose/.style={-*, line width=1pt, darkGreen},
    assoc/.style={-, line width=1pt, textGray}
]
    % Classes Services
    \node[class] (yolo) at (0, 6) {
        \textbf{\textcolor{darkGreen}{YoloService}}\\
        \rule{3.8cm}{0.5pt}\\
        - model: YOLO\\
        - class\_names: list\\
        - confidence: float\\
        \rule{3.8cm}{0.5pt}\\
        + load\_model()\\
        + detect(frame)\\
        + draw\_boxes()
    };
    
    \node[class] (mobile) at (5, 6) {
        \textbf{\textcolor{darkGreen}{ClassificationService}}\\
        \rule{3.8cm}{0.5pt}\\
        - model: tf.Model\\
        - img\_size: tuple\\
        \rule{3.8cm}{0.5pt}\\
        + load\_model()\\
        + classify(image)\\
        + preprocess()
    };
    
    \node[class] (firms) at (10, 6) {
        \textbf{\textcolor{darkGreen}{FirmsService}}\\
        \rule{3.8cm}{0.5pt}\\
        - api\_key: str\\
        - regions: dict\\
        \rule{3.8cm}{0.5pt}\\
        + fetch\_data(region)\\
        + get\_wind\_data()\\
        + parse\_hotspots()
    };
    
    \node[class] (sentinel) at (0, 1) {
        \textbf{\textcolor{darkGreen}{SentinelService}}\\
        \rule{3.8cm}{0.5pt}\\
        - client\_id: str\\
        - client\_secret: str\\
        - zones: list\\
        \rule{3.8cm}{0.5pt}\\
        + authenticate()\\
        + get\_image(zone)\\
        + get\_evalscript()
    };
    
    \node[class] (prediction) at (5, 1) {
        \textbf{\textcolor{darkGreen}{PredictionService}}\\
        \rule{3.8cm}{0.5pt}\\
        - base\_radius: float\\
        \rule{3.8cm}{0.5pt}\\
        + calculate(params)\\
        + get\_brightness\_factor()\\
        + get\_wind\_factor()
    };
    
    \node[class] (monitoring) at (10, 1) {
        \textbf{\textcolor{darkGreen}{MonitoringService}}\\
        \rule{3.8cm}{0.5pt}\\
        - scheduler: APScheduler\\
        - is\_running: bool\\
        - interval: int\\
        \rule{3.8cm}{0.5pt}\\
        + start()\\
        + stop()\\
        + scan\_all\_zones()
    };
    
    \node[class] (notif) at (5, -4) {
        \textbf{\textcolor{darkGreen}{NotificationService}}\\
        \rule{3.8cm}{0.5pt}\\
        - telegram\_token: str\\
        - email\_config: dict\\
        - cooldown: int\\
        \rule{3.8cm}{0.5pt}\\
        + send\_telegram()\\
        + send\_email()\\
        + check\_cooldown()
    };
    
    % Relations
    \draw[compose] (monitoring) -- (sentinel);
    \draw[compose] (monitoring) -- (notif);
    \draw[assoc] (monitoring) -- (firms);
    \draw[assoc] (prediction) -- (firms);
    
\end{tikzpicture}
\end{center}

\subsection{Diagramme d'Activité - Processus de Détection}

\vspace{0.5cm}
\begin{center}
\begin{tikzpicture}[
    start/.style={circle, fill=darkGreen, minimum size=0.5cm},
    stop/.style={circle, draw=darkGreen, line width=2pt, fill=darkGreen, minimum size=0.5cm},
    activity/.style={rectangle, rounded corners=8pt, draw=primaryGreen, line width=1.5pt, fill=mintGreen!50, text width=2.5cm, minimum height=1cm, align=center, font=\small},
    decision/.style={diamond, draw=alertOrange, line width=1.5pt, fill=sunYellow!30, aspect=2, font=\small},
    arrow/.style={->, >=stealth, line width=1.5pt, primaryGreen}
]
    % Start
    \node[start] (start) at (0, 0) {};
    
    % Activities
    \node[activity] (a1) at (3, 0) {Initialiser\\caméra};
    \node[activity] (a2) at (6.5, 0) {Capturer\\frame};
    \node[activity] (a3) at (10, 0) {Prétraiter\\image};
    \node[activity] (a4) at (13.5, 0) {Inférence\\YOLO};
    
    \node[decision] (d1) at (13.5, -2.5) {Feu?};
    
    \node[activity] (a5) at (10, -2.5) {Dessiner\\boxes};
    \node[activity] (a6) at (16.5, -2.5) {Envoyer\\alerte};
    
    \node[activity] (a7) at (6.5, -2.5) {Encoder\\MJPEG};
    \node[activity] (a8) at (3, -2.5) {Streamer\\vers client};
    
    \node[decision] (d2) at (3, -5) {Stop?};
    
    \node[stop] (stop) at (0, -5) {};
    
    % Arrows
    \draw[arrow] (start) -- (a1);
    \draw[arrow] (a1) -- (a2);
    \draw[arrow] (a2) -- (a3);
    \draw[arrow] (a3) -- (a4);
    \draw[arrow] (a4) -- (d1);
    \draw[arrow] (d1) -- node[left, font=\tiny] {Non} (a5);
    \draw[arrow] (d1) -- node[above, font=\tiny] {Oui} (a6);
    \draw[arrow] (a6) |- (a5);
    \draw[arrow] (a5) -- (a7);
    \draw[arrow] (a7) -- (a8);
    \draw[arrow] (a8) -- (d2);
    \draw[arrow] (d2) -- node[above, font=\tiny] {Oui} (stop);
    \draw[arrow] (d2.east) -- ++(1,0) |- node[right, font=\tiny, pos=0.25] {Non} (a2.south);
    
\end{tikzpicture}
\end{center}

\subsection{Diagramme de Déploiement}

Le diagramme de déploiement illustre l'architecture physique du système et la répartition des composants sur les différents nœuds.

\vspace{0.5cm}
\begin{center}
\begin{tikzpicture}[
    node/.style={rectangle, draw=darkGreen, line width=2pt, fill=mintGreen!20, rounded corners=5pt, minimum width=4.5cm, minimum height=3cm},
    component/.style={rectangle, draw=accentTeal, line width=1pt, fill=skyBlue!20, rounded corners=3pt, minimum width=3.5cm, minimum height=0.7cm, font=\footnotesize},
    external/.style={rectangle, draw=textGray, line width=1.5pt, fill=softGray, rounded corners=5pt, minimum width=3cm, minimum height=1.5cm},
    arrow/.style={<->, >=stealth, line width=1.5pt, primaryGreen}
]
    % Nœud Client
    \node[node, label={[font=\small\bfseries, text=darkGreen]above:Client (Browser)}] (client) at (0, 0) {};
    \node[component] at (0, 0.5) {React Application};
    \node[component] at (0, -0.4) {Leaflet Maps};
    \node[component] at (0, -1.3) {TailwindCSS};
    
    % Nœud Serveur
    \node[node, minimum height=5cm, label={[font=\small\bfseries, text=darkGreen]above:Serveur Backend}] (server) at (7, 0) {};
    \node[component] at (7, 1.5) {FastAPI};
    \node[component] at (7, 0.6) {TensorFlow};
    \node[component] at (7, -0.3) {YOLOv8/Ultralytics};
    \node[component] at (7, -1.2) {APScheduler};
    \node[component] at (7, -2.1) {SMTP/Telegram};
    
    % Services Externes
    \node[external] (nasa) at (13, 2) {NASA FIRMS};
    \node[external] (sentinel) at (13, 0) {Sentinel Hub};
    \node[external] (meteo) at (13, -2) {Open-Meteo};
    
    % Connexions
    \draw[arrow] (client) -- node[above, font=\footnotesize] {HTTP/REST} node[below, font=\footnotesize] {MJPEG} (server);
    \draw[arrow] (server) -- node[above, font=\tiny] {API} (nasa);
    \draw[arrow] (server) -- node[above, font=\tiny] {OAuth2} (sentinel);
    \draw[arrow] (server) -- node[above, font=\tiny] {API} (meteo);
    
\end{tikzpicture}
\end{center}

\vspace{1cm}

% Transition vers le chapitre suivant
\begin{center}
\begin{tikzpicture}
    \node[
        fill=primaryGreen!10,
        draw=primaryGreen,
        line width=1.5pt,
        rounded corners=12pt,
        inner sep=20pt,
        text width=13cm,
        align=center
    ] {
        \textcolor{primaryGreen}{\fontsize{24}{28}\selectfont\faArrowCircleRight}\\[15pt]
        \large\textbf{Chapitre Suivant}\\[10pt]
        \normalsize Le prochain chapitre détaille la conception et l'architecture technique\\
        du système AI Sentinel, incluant l'architecture logicielle,\\
        la conception des modèles IA et l'interface utilisateur.\\[10pt]
        \textit{\textcolor{textGray}{Chapitre IV --- Conception et Architecture}}
    };
\end{tikzpicture}
\end{center}


%
% Ou copiez son contenu à la place du chapitre correspondant.
%
% Structure du chapitre :
% - 3.1 Présentation du Projet (Vision, Parties Prenantes)
% - 3.2 Analyse des Besoins Fonctionnels (8 modules)
% - 3.3 Analyse des Besoins Non Fonctionnels (Performance, Fiabilité, etc.)
% - 3.4 Méthodologie de Développement (Agile/Scrum)
% - 3.5 Modélisation UML (Use Cases, Séquences, Classes, etc.)
%
% Les fichiers sources sont divisés en 4 parties pour faciliter la maintenance :
% - chapitre3_partie1.tex : Sections 3.1 et 3.2 (partiellement)
% - chapitre3_partie2.tex : Suite de 3.2 (Satellite, Prévention, Prédiction, Notifications)
% - chapitre3_partie3.tex : Sections 3.3 et 3.4
% - chapitre3_partie4.tex : Section 3.5 (UML)
% ============================================

% ============================================
% CHAPITRE III - ANALYSE ET SPÉCIFICATION DES BESOINS
% ============================================
\chapter{Analyse et Spécification des Besoins}
\thispagestyle{fancy}

\section{Présentation du Projet}

\lettrine[lines=3, lhang=0.15, loversize=0.1, findent=3pt]{\textcolor{primaryGreen}{L}}{a phase} d'analyse et de spécification des besoins constitue une étape fondamentale dans tout projet de développement logiciel. Elle permet de définir avec précision ce que le système doit accomplir, comment il doit se comporter, et quelles contraintes il doit respecter. Dans le cadre du projet \textbf{AI Sentinel}, cette phase revêt une importance particulière compte tenu de la criticité du domaine d'application : la détection précoce des incendies de forêt, où chaque minute gagnée peut sauver des vies et préserver des hectares de forêt.

Ce chapitre présente une analyse exhaustive des besoins fonctionnels et non fonctionnels du système, accompagnée d'une modélisation UML détaillée permettant de visualiser les interactions entre les différents acteurs et composants du système.

\subsection{Vision Globale}

Le projet \textbf{AI Sentinel} s'inscrit dans une vision ambitieuse : créer un écosystème technologique complet capable de détecter, surveiller, prédire et alerter en temps réel sur les risques d'incendies de forêt. Cette vision se concrétise à travers une plateforme web full-stack qui intègre les technologies les plus avancées en matière d'intelligence artificielle et de traitement d'images.

\begin{greenbox}[\faEye\ Vision du Projet AI Sentinel]
Notre ambition va au-delà de la simple détection : nous souhaitons fournir aux autorités et gestionnaires forestiers un \textbf{outil d'aide à la décision} complet qui leur permette d'anticiper les risques, d'optimiser leurs ressources, et d'intervenir de manière proactive plutôt que réactive.

La plateforme AI Sentinel se positionne comme un \textbf{hub centralisé} intégrant :
\begin{itemize}[leftmargin=1cm, itemsep=5pt]
    \item[\textcolor{primaryGreen}{\faVideo}] L'analyse vidéo en temps réel par intelligence artificielle
    \item[\textcolor{primaryGreen}{\faSatellite}] La surveillance satellite à couverture globale
    \item[\textcolor{primaryGreen}{\faChartLine}] Les algorithmes prédictifs de propagation des feux
    \item[\textcolor{primaryGreen}{\faBell}] Un système d'alertes multi-canaux réactif
    \item[\textcolor{primaryGreen}{\faCloudSun}] L'intégration des données météorologiques
\end{itemize}
\end{greenbox}

\vspace{0.5cm}

L'objectif principal est de réduire significativement le \textbf{temps de détection} des incendies, facteur critique dans la limitation des dégâts. Les études montrent qu'une intervention dans les 15 premières minutes suivant le départ d'un feu permet de contenir 90\% des incendies avant qu'ils ne deviennent incontrôlables.

\begin{center}
\begin{tikzpicture}[
    block/.style={rectangle, rounded corners=10pt, draw=primaryGreen, line width=2pt, fill=mintGreen, text width=3cm, minimum height=2cm, align=center, font=\small},
    arrow/.style={->, >=stealth, line width=2pt, color=leafGreen}
]
    % Blocs
    \node[block] (detect) at (0,0) {\textcolor{darkGreen}{\faSearch}\\[5pt]\textbf{Détection}\\Rapide};
    \node[block] (analyse) at (4,0) {\textcolor{darkGreen}{\faBrain}\\[5pt]\textbf{Analyse}\\IA};
    \node[block] (predict) at (8,0) {\textcolor{darkGreen}{\faChartArea}\\[5pt]\textbf{Prédiction}\\Propagation};
    \node[block] (alert) at (12,0) {\textcolor{darkGreen}{\faBell}\\[5pt]\textbf{Alerte}\\Instantanée};
    
    % Flèches
    \draw[arrow] (detect) -- (analyse);
    \draw[arrow] (analyse) -- (predict);
    \draw[arrow] (predict) -- (alert);
    
    % Temps
    \node[font=\footnotesize, text=textGray] at (2, -1.5) {< 500ms};
    \node[font=\footnotesize, text=textGray] at (6, -1.5) {< 1s};
    \node[font=\footnotesize, text=textGray] at (10, -1.5) {< 2s};
    
    % Titre
    \node[font=\bfseries\large, text=darkGreen] at (6, 2) {Chaîne de Traitement AI Sentinel};
\end{tikzpicture}
\end{center}

\subsection{Parties Prenantes}

L'identification et la compréhension des parties prenantes sont essentielles pour garantir que le système réponde aux besoins réels de ses utilisateurs. Le projet AI Sentinel implique plusieurs catégories d'acteurs, chacune ayant des attentes et des besoins spécifiques.

\subsubsection{Utilisateurs Finaux}

Les utilisateurs finaux représentent le cœur de notre cible. Ils constituent les bénéficiaires directs du système et leurs besoins orientent l'ensemble des choix de conception.

\begin{infobox}{Profils Utilisateurs Identifiés}

\textbf{\faUserShield\ Autorités de Protection Civile}

Les services de protection civile et les pompiers forestiers représentent les utilisateurs principaux du système. Leur mission consiste à surveiller les zones à risque, coordonner les interventions et gérer les ressources humaines et matérielles. Ils requièrent :
\begin{itemize}[leftmargin=0.5cm, itemsep=3pt]
    \item Des alertes instantanées et fiables
    \item Une localisation précise des foyers d'incendie
    \item Des informations sur la propagation probable
    \item Un accès rapide aux données météorologiques
\end{itemize}

\vspace{0.3cm}

\textbf{\faTree\ Gestionnaires Forestiers}

Les agents des Eaux et Forêts et les gestionnaires d'espaces naturels utilisent le système pour la surveillance quotidienne de leurs territoires. Leurs besoins incluent :
\begin{itemize}[leftmargin=0.5cm, itemsep=3pt]
    \item La surveillance continue des zones forestières
    \item L'identification des zones à haut risque
    \item L'historique des incidents pour l'analyse
    \item La planification préventive des interventions
\end{itemize}

\vspace{0.3cm}

\textbf{\faCity\ Collectivités Territoriales}

Les communes et régions concernées par les zones forestières peuvent utiliser le système pour :
\begin{itemize}[leftmargin=0.5cm, itemsep=3pt]
    \item La coordination avec les services d'urgence
    \item L'information et l'alerte des populations
    \item La gestion des évacuations si nécessaire
\end{itemize}
\end{infobox}

\subsubsection{Administrateurs Système}

Les administrateurs techniques sont responsables du bon fonctionnement, de la maintenance et de l'évolution du système. Leur rôle est crucial pour garantir la disponibilité et les performances de la plateforme.

\begin{techbox}{Responsabilités des Administrateurs}
\begin{itemize}[leftmargin=0.5cm, itemsep=8pt]
    \item[\textcolor{accentTeal}{\faServer}] \textbf{Gestion de l'Infrastructure :} Déploiement, monitoring et maintenance des serveurs backend et frontend
    
    \item[\textcolor{accentTeal}{\faCogs}] \textbf{Configuration des Services :} Paramétrage des API externes (NASA FIRMS, Sentinel Hub), gestion des clés API et des quotas
    
    \item[\textcolor{accentTeal}{\faBrain}] \textbf{Gestion des Modèles IA :} Mise à jour et réentraînement des modèles de détection (YOLOv8, MobileNetV2)
    
    \item[\textcolor{accentTeal}{\faBell}] \textbf{Configuration des Alertes :} Paramétrage des canaux de notification (Telegram, Email) et gestion des destinataires
    
    \item[\textcolor{accentTeal}{\faChartBar}] \textbf{Monitoring des Performances :} Suivi des métriques de performance et optimisation continue
\end{itemize}
\end{techbox}

\subsubsection{Équipe de Développement}

L'équipe de développement assure la conception, l'implémentation et l'évolution continue du système. Elle est composée de profils complémentaires travaillant en synergie.

\begin{table}[H]
\centering
\caption{Composition de l'équipe de développement}
\label{tab:equipe}
\rowcolors{2}{mintGreen!30}{white}
\begin{tabular}{l l p{7cm}}
\toprule
\rowcolor{primaryGreen}
\textcolor{white}{\textbf{Rôle}} & \textcolor{white}{\textbf{Compétences}} & \textcolor{white}{\textbf{Responsabilités}} \\
\midrule
Développeur Backend & Python, FastAPI, IA & API REST, intégration modèles IA, services externes \\
Développeur Frontend & React, TypeScript & Interface utilisateur, cartographie, UX/UI \\
Ingénieur IA/ML & TensorFlow, PyTorch & Entraînement modèles, optimisation, évaluation \\
DevOps & Docker, CI/CD & Déploiement, automatisation, monitoring \\
\bottomrule
\end{tabular}
\end{table}

\vspace{0.5cm}

\begin{center}
\begin{tikzpicture}[
    actor/.style={circle, draw=primaryGreen, line width=2pt, fill=mintGreen, minimum size=2cm, align=center, font=\small},
    system/.style={rectangle, rounded corners=15pt, draw=darkGreen, line width=3pt, fill=leafGreen!20, minimum width=6cm, minimum height=3cm, align=center}
]
    % Système central
    \node[system] (sys) at (0,0) {\textbf{\large AI Sentinel}\\[5pt]\footnotesize Système de Détection\\des Incendies de Forêt};
    
    % Acteurs
    \node[actor] (user) at (-5, 2) {\faUser\\Utilisateur};
    \node[actor] (admin) at (5, 2) {\faUserCog\\Admin};
    \node[actor] (dev) at (-5, -2) {\faCode\\Développeur};
    \node[actor] (ext) at (5, -2) {\faCloud\\Services\\Externes};
    
    % Connexions
    \draw[->, >=stealth, line width=1.5pt, primaryGreen] (user) -- (sys);
    \draw[->, >=stealth, line width=1.5pt, primaryGreen] (admin) -- (sys);
    \draw[->, >=stealth, line width=1.5pt, primaryGreen] (dev) -- (sys);
    \draw[<->, >=stealth, line width=1.5pt, accentTeal] (ext) -- (sys);
    
\end{tikzpicture}
\end{center}

% ============================================
\newpage
\section{Analyse des Besoins Fonctionnels}
% ============================================

L'analyse des besoins fonctionnels définit ce que le système doit \textbf{faire}. Chaque besoin fonctionnel décrit une fonctionnalité spécifique que le système doit offrir à ses utilisateurs. Dans le cadre d'AI Sentinel, nous avons organisé ces besoins en \textbf{huit modules} distincts, chacun répondant à un aspect particulier de la détection et de la prévention des incendies.

\begin{greenbox}[\faListOl\ Vue d'Ensemble des Modules Fonctionnels]
\begin{center}
\begin{tikzpicture}[
    module/.style={rectangle, rounded corners=8pt, draw=primaryGreen, line width=1.5pt, fill=mintGreen, text width=3.8cm, minimum height=1.8cm, align=center, font=\small}
]
    % Ligne 1
    \node[module] (m1) at (0,0) {\textcolor{darkGreen}{\faVideo}\\[3pt]\textbf{Détection}\\Temps Réel};
    \node[module] (m2) at (4.5,0) {\textcolor{darkGreen}{\faImage}\\[3pt]\textbf{Classification}\\d'Images};
    \node[module] (m3) at (9,0) {\textcolor{darkGreen}{\faFilm}\\[3pt]\textbf{Analyse}\\Vidéo};
    \node[module] (m4) at (13.5,0) {\textcolor{darkGreen}{\faSatellite}\\[3pt]\textbf{Surveillance}\\Satellite};
    
    % Ligne 2
    \node[module] (m5) at (0,-2.5) {\textcolor{darkGreen}{\faMapMarkerAlt}\\[3pt]\textbf{Prévention}\\Hotspots};
    \node[module] (m6) at (4.5,-2.5) {\textcolor{darkGreen}{\faChartLine}\\[3pt]\textbf{Prédiction}\\Propagation};
    \node[module] (m7) at (9,-2.5) {\textcolor{darkGreen}{\faBell}\\[3pt]\textbf{Notifications}\\Multi-canaux};
    \node[module] (m8) at (13.5,-2.5) {\textcolor{darkGreen}{\faCloudSun}\\[3pt]\textbf{FWI}\\Météo};
\end{tikzpicture}
\end{center}
\end{greenbox}

\subsection{Module de Détection en Temps Réel}

Le module de détection en temps réel constitue le \textbf{cœur} du système AI Sentinel. Il permet l'analyse continue de flux vidéo provenant de caméras de surveillance pour détecter instantanément la présence de feu ou de fumée. Ce module exploite la puissance du modèle \textbf{YOLOv8} (You Only Look Once, version 8), un algorithme de détection d'objets state-of-the-art reconnu pour sa rapidité et sa précision.

\subsubsection{Contexte et Justification}

La détection en temps réel répond à un besoin critique : identifier les incendies \textbf{dès leur déclenchement}, avant qu'ils ne se propagent de manière incontrôlable. Les méthodes traditionnelles de surveillance (tours de guet, patrouilles) présentent des limitations importantes en termes de couverture, de disponibilité 24h/24, et de vitesse de détection. L'utilisation de l'intelligence artificielle permet de surmonter ces obstacles en offrant une surveillance automatisée, continue et objective.

\begin{alertbox}{Importance de la Réactivité}
Selon les études de l'Office National des Forêts (ONF), un feu de forêt peut doubler de taille toutes les \textbf{15 à 20 minutes} dans des conditions favorables à la propagation (vent fort, végétation sèche). Chaque minute gagnée dans la détection représente potentiellement des hectares de forêt préservés et des vies sauvées.
\end{alertbox}

\subsubsection{Besoins Fonctionnels Détaillés}

\begin{objectifbox}{RF01 : Capture Vidéo via Caméra/Webcam}
\textbf{Description :} Le système doit être capable de capturer un flux vidéo en temps réel à partir de différentes sources : webcam connectée, caméra IP, ou fichier vidéo local.

\textbf{Spécifications détaillées :}
\begin{itemize}[leftmargin=1cm, itemsep=5pt]
    \item Prise en charge des webcams USB standard via l'index de périphérique
    \item Support des flux RTSP pour les caméras IP professionnelles
    \item Capacité à traiter des fichiers vidéo uploadés (MP4, AVI, MOV)
    \item Résolution supportée : de 480p à 1080p
    \item Fréquence d'images : 25-30 FPS minimum
\end{itemize}

\textbf{Critère d'acceptation :} Le système doit démarrer la capture vidéo en moins de 2 secondes après activation par l'utilisateur.
\end{objectifbox}

\vspace{0.3cm}

\begin{objectifbox}{RF02 : Détection Automatique Feu/Fumée}
\textbf{Description :} Le système doit analyser chaque frame du flux vidéo pour détecter la présence de feu ou de fumée avec une précision élevée.

\textbf{Spécifications détaillées :}
\begin{itemize}[leftmargin=1cm, itemsep=5pt]
    \item Utilisation du modèle YOLOv8 personnalisé entraîné sur un dataset de feux de forêt
    \item Classification en deux classes : \texttt{Fire} (feu) et \texttt{Smoke} (fumée)
    \item Seuil de confiance configurable (par défaut : 0.5)
    \item Traitement à minimum 30 FPS sur GPU, 15 FPS sur CPU
    \item Détection multi-instances (plusieurs feux/fumées simultanés)
\end{itemize}

\textbf{Critère d'acceptation :} Le modèle doit atteindre une précision de détection $\geq$ 85\% avec un taux de faux positifs < 5\%.
\end{objectifbox}

\vspace{0.3cm}

\begin{objectifbox}{RF03 : Affichage des Bounding Boxes}
\textbf{Description :} Les objets détectés (feu, fumée) doivent être encadrés visuellement sur le flux vidéo avec des informations contextuelles.

\textbf{Spécifications détaillées :}
\begin{itemize}[leftmargin=1cm, itemsep=5pt]
    \item Rectangles de délimitation (bounding boxes) colorés selon la classe
    \item Code couleur : \textcolor{moroccanRed}{Rouge} pour le feu, \textcolor{textGray}{Gris} pour la fumée
    \item Affichage du nom de la classe et du score de confiance (ex: "Fire 0.92")
    \item Épaisseur et taille de police adaptatives selon la résolution
    \item Mise à jour en temps réel synchronisée avec le flux vidéo
\end{itemize}

\textbf{Critère d'acceptation :} Les annotations visuelles ne doivent pas dégrader les performances de plus de 5\%.
\end{objectifbox}

\vspace{0.3cm}

\begin{objectifbox}{RF04 : Génération d'Alertes Visuelles}
\textbf{Description :} Lorsqu'un feu ou une fumée est détecté avec un niveau de confiance suffisant, le système doit générer des alertes visuelles immédiates.

\textbf{Spécifications détaillées :}
\begin{itemize}[leftmargin=1cm, itemsep=5pt]
    \item Indicateur visuel clignotant sur l'interface lors d'une détection
    \item Changement de couleur de l'arrière-plan ou du cadre vidéo
    \item Notification toast affichant les détails de la détection
    \item Son d'alerte optionnel (configurable par l'utilisateur)
    \item Horodatage précis de chaque détection
\end{itemize}

\textbf{Critère d'acceptation :} L'alerte visuelle doit apparaître dans un délai maximum de 500ms après la détection.
\end{objectifbox}

\vspace{0.5cm}

\begin{center}
\begin{tikzpicture}[
    step/.style={rectangle, rounded corners=5pt, draw=accentTeal, line width=1.5pt, fill=skyBlue!20, text width=2.5cm, minimum height=1.5cm, align=center, font=\footnotesize},
    arrow/.style={->, >=stealth, line width=1.5pt, color=accentTeal}
]
    \node[step] (s1) at (0,0) {\faVideo\\Capture\\Vidéo};
    \node[step] (s2) at (3.5,0) {\faCropAlt\\Prétraitement\\Frame};
    \node[step] (s3) at (7,0) {\faBrain\\Inférence\\YOLOv8};
    \node[step] (s4) at (10.5,0) {\faVectorSquare\\Bounding\\Boxes};
    \node[step] (s5) at (14,0) {\faBell\\Alerte\\Visuelle};
    
    \draw[arrow] (s1) -- (s2);
    \draw[arrow] (s2) -- (s3);
    \draw[arrow] (s3) -- (s4);
    \draw[arrow] (s4) -- (s5);
    
    \node[font=\bfseries, text=darkGreen] at (7, 1.5) {Pipeline de Détection Temps Réel};
\end{tikzpicture}
\end{center}

\subsection{Module de Classification d'Images}

Le module de classification d'images permet aux utilisateurs d'analyser des photographies statiques pour déterminer si elles contiennent des signes de feu, de fumée, ou aucun des deux. Ce module utilise l'architecture \textbf{MobileNetV2} avec transfer learning, offrant un excellent compromis entre précision et performance.

\subsubsection{Contexte et Justification}

Complémentaire à la détection temps réel, la classification d'images répond à plusieurs cas d'usage importants :
\begin{itemize}[leftmargin=1cm, itemsep=5pt]
    \item \textbf{Vérification manuelle :} Confirmation par l'opérateur d'une image suspecte
    \item \textbf{Analyse rétrospective :} Examen d'images historiques
    \item \textbf{Traitement hors ligne :} Analyse sans connexion caméra en direct
    \item \textbf{Rapports et documentation :} Génération de preuves visuelles
\end{itemize}

\subsubsection{Besoins Fonctionnels Détaillés}

\begin{objectifbox}{RF05 : Upload d'Images}
\textbf{Description :} L'utilisateur doit pouvoir soumettre une ou plusieurs images au système pour analyse.

\textbf{Spécifications détaillées :}
\begin{itemize}[leftmargin=1cm, itemsep=5pt]
    \item Formats supportés : JPEG, PNG, WebP, BMP
    \item Taille maximale par image : 10 MB
    \item Interface drag-and-drop intuitive
    \item Prévisualisation de l'image avant soumission
    \item Upload multiple avec file d'attente de traitement
\end{itemize}

\textbf{Critère d'acceptation :} L'upload d'une image de 5 MB doit se terminer en moins de 3 secondes sur une connexion standard.
\end{objectifbox}

\vspace{0.3cm}

\begin{objectifbox}{RF06 : Classification Multi-Classes}
\textbf{Description :} Le système doit classifier l'image uploadée dans l'une des trois catégories : Fire, Smoke, ou Non-Fire.

\textbf{Spécifications détaillées :}
\begin{itemize}[leftmargin=1cm, itemsep=5pt]
    \item Modèle MobileNetV2 pré-entraîné sur ImageNet, fine-tuné sur dataset feu
    \item Trois classes de sortie avec probabilités associées
    \item Prétraitement automatique (redimensionnement 224×224, normalisation)
    \item Temps d'inférence < 500ms par image
    \item Précision globale $\geq$ 97\% sur le jeu de test
\end{itemize}

\textbf{Critère d'acceptation :} Le système doit retourner la classe prédite avec son score de confiance en moins de 1 seconde.
\end{objectifbox}

\vspace{0.3cm}

\begin{objectifbox}{RF07 : Affichage du Score de Confiance}
\textbf{Description :} Le résultat de la classification doit inclure un score de confiance permettant à l'utilisateur d'évaluer la fiabilité de la prédiction.

\textbf{Spécifications détaillées :}
\begin{itemize}[leftmargin=1cm, itemsep=5pt]
    \item Score de confiance exprimé en pourcentage (0-100\%)
    \item Affichage visuel avec code couleur (vert > 80\%, orange 50-80\%, rouge < 50\%)
    \item Distribution des probabilités pour les trois classes
    \item Indicateur visuel du niveau de certitude du modèle
    \item Recommandation d'action basée sur le niveau de confiance
\end{itemize}

\textbf{Critère d'acceptation :} L'interface doit afficher clairement le résultat avec les probabilités pour chaque classe.
\end{objectifbox}

\vspace{0.5cm}

\begin{table}[H]
\centering
\caption{Exemple de sortie du module de classification}
\label{tab:classification}
\rowcolors{2}{mintGreen!30}{white}
\begin{tabular}{l c l}
\toprule
\rowcolor{primaryGreen}
\textcolor{white}{\textbf{Classe}} & \textcolor{white}{\textbf{Probabilité}} & \textcolor{white}{\textbf{Interprétation}} \\
\midrule
\textbf{Fire} & 92.3\% & \textcolor{moroccanRed}{Détection confirmée} \\
Smoke & 5.1\% & Trace possible \\
Non-Fire & 2.6\% & Négligeable \\
\bottomrule
\end{tabular}
\end{table}

\subsection{Module d'Analyse Vidéo}

Le module d'analyse vidéo étend les capacités de détection aux fichiers vidéo préenregistrés. Il permet un traitement exhaustif frame par frame et génère une vidéo annotée avec les détections.

\subsubsection{Besoins Fonctionnels Détaillés}

\begin{objectifbox}{RF08 : Upload de Vidéos}
\textbf{Description :} L'utilisateur doit pouvoir soumettre des fichiers vidéo pour analyse complète.

\textbf{Spécifications détaillées :}
\begin{itemize}[leftmargin=1cm, itemsep=5pt]
    \item Formats supportés : MP4, AVI, MOV, MKV, WebM
    \item Taille maximale : 500 MB (configurable)
    \item Barre de progression de l'upload
    \item Validation du format avant traitement
    \item Support des différentes résolutions (480p à 4K)
\end{itemize}
\end{objectifbox}

\vspace{0.3cm}

\begin{objectifbox}{RF09 : Traitement Frame par Frame}
\textbf{Description :} Chaque frame de la vidéo doit être analysée individuellement par le modèle de détection.

\textbf{Spécifications détaillées :}
\begin{itemize}[leftmargin=1cm, itemsep=5pt]
    \item Extraction et analyse de chaque frame
    \item Affichage de la progression du traitement
    \item Statistiques en temps réel (frames traitées, détections)
    \item Possibilité d'annuler le traitement en cours
    \item Traitement asynchrone pour ne pas bloquer l'interface
\end{itemize}
\end{objectifbox}

\vspace{0.3cm}

\begin{objectifbox}{RF10 : Export Vidéo Annotée}
\textbf{Description :} Le système doit générer une version annotée de la vidéo avec les détections superposées.

\textbf{Spécifications détaillées :}
\begin{itemize}[leftmargin=1cm, itemsep=5pt]
    \item Génération d'une nouvelle vidéo avec bounding boxes
    \item Conservation de la résolution et du framerate d'origine
    \item Téléchargement direct du fichier résultat
    \item Rapport JSON des détections (timestamps, coordonnées, classes)
    \item Prévisualisation avant téléchargement
\end{itemize}
\end{objectifbox}

% ============================================
% CHAPITRE III - PARTIE 2 : MODULES SATELLITE, PRÉVENTION, PRÉDICTION
% ============================================

\subsection{Module de Surveillance Satellite}

Le module de surveillance satellite représente une composante stratégique du système AI Sentinel. Il permet d'étendre la couverture de détection au-delà des zones équipées de caméras en exploitant les images satellites du programme européen \textbf{Copernicus} via l'API \textbf{Sentinel Hub}, ainsi que les données de hotspots thermiques de \textbf{NASA FIRMS}.

\subsubsection{Contexte et Justification}

Les forêts marocaines s'étendent sur près de \textbf{9 millions d'hectares}, une superficie impossible à couvrir intégralement par des caméras de surveillance. L'imagerie satellite offre une solution complémentaire permettant une surveillance à grande échelle. Les satellites Sentinel-2 fournissent des images multi-spectrales avec une résolution spatiale de 10 à 60 mètres et une revisite de 5 jours, tandis que les satellites MODIS et VIIRS de NASA FIRMS détectent les anomalies thermiques en quasi temps réel.

\begin{moroccobox}{Couverture Géographique du Maroc}
Le système AI Sentinel surveille \textbf{8 zones géographiques} couvrant l'ensemble du territoire marocain à risque d'incendie :

\vspace{0.3cm}
\begin{center}
\begin{tabular}{l l l}
\textbf{Zone} & \textbf{Coordonnées} & \textbf{Caractéristiques} \\
\hline
North (Tanger-Tétouan) & -6.00, 34.00 & Forêts méditerranéennes \\
Rif & -5.00, 34.50 & Montagnes boisées \\
Oriental & -3.00, 33.50 & Steppe et maquis \\
Casablanca & -8.00, 33.00 & Zone périurbaine \\
Middle Atlas & -6.00, 32.50 & Cèdres et chênes verts \\
Marrakech & -8.50, 31.00 & Arganeraie \\
High Atlas & -8.00, 30.50 & Forêts d'altitude \\
Souss & -10.00, 29.50 & Formations semi-arides \\
\end{tabular}
\end{center}
\end{moroccobox}

\subsubsection{Besoins Fonctionnels Détaillés}

\begin{objectifbox}{RF11 : Acquisition d'Images Sentinel-2}
\textbf{Description :} Le système doit pouvoir récupérer automatiquement les images satellites Sentinel-2 pour les zones définies.

\textbf{Spécifications détaillées :}
\begin{itemize}[leftmargin=1cm, itemsep=5pt]
    \item Authentification OAuth2 avec l'API Sentinel Hub
    \item Récupération d'images Sentinel-2 L2A (atmosphériquement corrigées)
    \item Deux types de scripts d'évaluation (evalscripts) :
    \begin{itemize}
        \item \textbf{True Color :} RGB standard pour visualisation
        \item \textbf{Fire Detection :} SWIR enhanced pour détection thermique
    \end{itemize}
    \item Résolution de 10m/pixel pour les bandes visibles
    \item Gestion intelligente de la couverture nuageuse (seuil < 30\%)
    \item Stockage local des images pour analyse ultérieure
\end{itemize}

\textbf{Critère d'acceptation :} Le système doit récupérer une image satellite en moins de 10 secondes (hors temps de traitement Sentinel Hub).
\end{objectifbox}

\vspace{0.3cm}

\begin{objectifbox}{RF12 : Analyse IA des Images Satellites}
\textbf{Description :} Les images satellites doivent être analysées par un modèle IA spécialisé (CAM - Class Activation Map) pour détecter les zones de feu potentiel.

\textbf{Spécifications détaillées :}
\begin{itemize}[leftmargin=1cm, itemsep=5pt]
    \item Modèle CAM personnalisé avec sortie duale :
    \begin{itemize}
        \item Carte d'activation (10×10×64) pour visualisation des zones critiques
        \item Classification binaire (Fire / No Fire)
    \end{itemize}
    \item Prétraitement : redimensionnement 224×224, normalisation 0-1
    \item Génération de heatmaps colorées superposées à l'image originale
    \item Score de confiance pour chaque analyse
    \item Détection des anomalies thermiques via les bandes SWIR
\end{itemize}

\textbf{Critère d'acceptation :} L'analyse d'une image satellite doit se compléter en moins de 5 secondes avec un taux de détection > 90\%.
\end{objectifbox}

\vspace{0.3cm}

\begin{objectifbox}{RF13 : Scan Automatique Programmé}
\textbf{Description :} Le système doit pouvoir effectuer des scans automatiques périodiques de toutes les zones surveillées.

\textbf{Spécifications détaillées :}
\begin{itemize}[leftmargin=1cm, itemsep=5pt]
    \item Scheduler configurable (intervalle par défaut : 30 minutes)
    \item Scan séquentiel de toutes les zones pour éviter la surcharge API
    \item Détection des changements par comparaison avec le scan précédent
    \item Génération automatique d'alertes en cas de détection positive
    \item Interface de contrôle : démarrer/arrêter le monitoring
    \item Logs détaillés de chaque cycle de scan
    \item Résilience aux erreurs (retry automatique)
\end{itemize}

\textbf{Critère d'acceptation :} Le scan de toutes les zones doit se terminer en moins de 10 minutes avec génération d'un rapport de synthèse.
\end{objectifbox}

\vspace{0.5cm}

\begin{center}
\begin{tikzpicture}[
    node distance=1.5cm,
    block/.style={rectangle, rounded corners=8pt, draw=accentTeal, line width=1.5pt, fill=skyBlue!15, text width=2.8cm, minimum height=1.8cm, align=center, font=\footnotesize},
    decision/.style={diamond, draw=alertOrange, line width=1.5pt, fill=sunYellow!30, text width=1.5cm, align=center, font=\footnotesize, aspect=2},
    arrow/.style={->, >=stealth, line width=1pt, color=primaryGreen}
]
    \node[block] (start) {\faClock\\Déclenchement\\Scheduler};
    \node[block, right=of start] (fetch) {\faSatellite\\Récupération\\Image Sentinel};
    \node[block, right=of fetch] (analyze) {\faBrain\\Analyse\\CAM Model};
    \node[decision, right=of analyze] (detect) {Feu\\?};
    \node[block, above right=1cm and 1.5cm of detect] (alert) {\faBell\\Envoi\\Alerte};
    \node[block, below right=1cm and 1.5cm of detect] (log) {\faDatabase\\Log\\Résultat};
    \node[block, right=3.5cm of detect] (next) {\faRedo\\Zone\\Suivante};
    
    \draw[arrow] (start) -- (fetch);
    \draw[arrow] (fetch) -- (analyze);
    \draw[arrow] (analyze) -- (detect);
    \draw[arrow] (detect) -- node[above, font=\tiny, text=moroccanRed] {Oui} (alert);
    \draw[arrow] (detect) -- node[below, font=\tiny, text=primaryGreen] {Non} (log);
    \draw[arrow] (alert) -| (next);
    \draw[arrow] (log) -| (next);
    
    \node[font=\bfseries\small, text=darkGreen] at (5, -2.5) {Workflow de Surveillance Satellite Automatisée};
\end{tikzpicture}
\end{center}

\subsection{Module de Prévention (Hotspots NASA FIRMS)}

Le module de prévention exploite les données de \textbf{NASA FIRMS} (Fire Information for Resource Management System) pour afficher en temps quasi réel les hotspots thermiques détectés par les satellites MODIS et VIIRS. Ces données permettent une vision globale des zones à risque et des feux actifs.

\subsubsection{Contexte et Justification}

NASA FIRMS fournit gratuitement des données de détection thermique avec une latence de seulement \textbf{3 heures}, permettant une surveillance quasi temps réel à l'échelle globale. Ces données sont essentielles pour :
\begin{itemize}[leftmargin=1cm, itemsep=3pt]
    \item Identifier les feux actifs dans des zones non couvertes par les caméras
    \item Cartographier les tendances et patterns d'incendies
    \item Anticiper les risques de propagation vers les zones habitées
    \item Compléter les données de terrain avec une vue satellite
\end{itemize}

\begin{infobox}{Données NASA FIRMS}
Les satellites MODIS (Moderate Resolution Imaging Spectroradiometer) et VIIRS (Visible Infrared Imaging Radiometer Suite) détectent les anomalies thermiques en mesurant la luminosité dans les bandes infrarouges thermiques. Un pixel est marqué comme hotspot lorsque sa température dépasse significativement celle des pixels environnants.

\textbf{Caractéristiques des données :}
\begin{itemize}[leftmargin=0.5cm, itemsep=3pt]
    \item \textbf{Résolution MODIS :} 1 km
    \item \textbf{Résolution VIIRS :} 375 m
    \item \textbf{Latence :} 3 heures après passage satellite
    \item \textbf{Couverture :} Globale, plusieurs passages par jour
\end{itemize}
\end{infobox}

\subsubsection{Besoins Fonctionnels Détaillés}

\begin{objectifbox}{RF14 : Récupération des Données NASA FIRMS}
\textbf{Description :} Le système doit récupérer automatiquement les données de hotspots depuis l'API NASA FIRMS pour les régions surveillées.

\textbf{Spécifications détaillées :}
\begin{itemize}[leftmargin=1cm, itemsep=5pt]
    \item Authentification via MAP\_KEY NASA FIRMS
    \item Récupération des données VIIRS\_SNPP pour les dernières 24 heures
    \item Filtrage par zone géographique (bounding box)
    \item Parsing des attributs : latitude, longitude, brightness, confidence, frp, daynight
    \item Mise en cache pour optimiser les requêtes répétées
    \item Gestion des erreurs et retry automatique
\end{itemize}

\textbf{Critère d'acceptation :} Les données doivent être récupérées et parsées en moins de 5 secondes par zone.
\end{objectifbox}

\vspace{0.3cm}

\begin{objectifbox}{RF15 : Affichage sur Carte Interactive}
\textbf{Description :} Les hotspots doivent être affichés sur une carte interactive permettant la navigation et l'exploration des données.

\textbf{Spécifications détaillées :}
\begin{itemize}[leftmargin=1cm, itemsep=5pt]
    \item Carte interactive basée sur Leaflet/React-Leaflet
    \item Marqueurs colorés selon le niveau de confiance :
    \begin{itemize}
        \item \textcolor{moroccanRed}{\textbf{Rouge}} : High confidence
        \item \textcolor{alertOrange}{\textbf{Orange}} : Nominal confidence
        \item \textcolor{sunYellow}{\textbf{Jaune}} : Low confidence
    \end{itemize}
    \item Popup d'information au clic sur un marqueur
    \item Zoom et pan fluides
    \item Clustering automatique des marqueurs à faible zoom
    \item Filtres par région, date, niveau de confiance
    \item Lien vers Google Maps pour navigation
\end{itemize}

\textbf{Critère d'acceptation :} La carte doit afficher jusqu'à 500 marqueurs sans dégradation notable des performances.
\end{objectifbox}

\vspace{0.3cm}

\begin{objectifbox}{RF16 : Calcul du Rayon de Propagation}
\textbf{Description :} Pour chaque hotspot, le système doit calculer une estimation du rayon de propagation probable basée sur les données disponibles.

\textbf{Spécifications détaillées :}
\begin{itemize}[leftmargin=1cm, itemsep=5pt]
    \item Algorithme de calcul intégrant plusieurs facteurs :
    \begin{itemize}
        \item \textbf{Luminosité (brightness\_ti4/ti5)} : indicateur d'intensité
        \item \textbf{Confiance NASA} : fiabilité de la détection
        \item \textbf{FRP (Fire Radiative Power)} : puissance du feu en MW
    \end{itemize}
    \item Affichage d'un cercle de propagation sur la carte
    \item Rayon borné entre 1 km et 15 km
    \item Code couleur du cercle selon le niveau de risque
    \item Tooltip avec estimation de la surface menacée
\end{itemize}

\textbf{Critère d'acceptation :} Le rayon de propagation doit être calculé et affiché en moins de 100ms par hotspot.
\end{objectifbox}

\subsection{Module de Prédiction de Propagation}

Le module de prédiction de propagation constitue l'élément d'anticipation du système AI Sentinel. Il permet d'estimer, à horizon de 6 heures, la zone potentiellement affectée par la propagation d'un incendie en tenant compte de multiples facteurs environnementaux.

\subsubsection{Contexte et Justification}

La prédiction de la propagation des feux de forêt est un problème complexe qui dépend de nombreuses variables : topographie, type de végétation, conditions météorologiques, humidité du sol, etc. Notre approche empirique, basée sur les données réelles de NASA FIRMS et les données météorologiques, offre une estimation utilisable opérationnellement tout en restant computationnellement légère.

\begin{alertbox}{Importance de la Prédiction}
Une prédiction, même approximative, de la propagation d'un feu permet aux services d'intervention de :
\begin{itemize}[leftmargin=0.5cm, itemsep=3pt]
    \item \textbf{Anticiper} les zones à évacuer
    \item \textbf{Positionner} les moyens de lutte de manière optimale
    \item \textbf{Protéger} les infrastructures critiques en priorité
    \item \textbf{Informer} les populations à risque
\end{itemize}
\end{alertbox}

\subsubsection{Besoins Fonctionnels Détaillés}

\begin{objectifbox}{RF17 : Saisie des Paramètres de Simulation}
\textbf{Description :} L'utilisateur doit pouvoir saisir ou modifier les paramètres de simulation de propagation.

\textbf{Spécifications détaillées :}
\begin{itemize}[leftmargin=1cm, itemsep=5pt]
    \item Interface de saisie intuitive avec valeurs par défaut
    \item Paramètres configurables :
    \begin{itemize}
        \item \textbf{Luminosité (Kelvin)} : 300K - 500K (défaut : 350K)
        \item \textbf{Confiance NASA} : high, nominal, low
        \item \textbf{Vitesse du vent} : 0 - 100 km/h
        \item \textbf{Direction du vent} : 0° - 360°
    \end{itemize}
    \item Sliders et sélecteurs pour faciliter la saisie
    \item Validation des valeurs en temps réel
    \item Pré-remplissage automatique depuis les données réelles
\end{itemize}

\textbf{Critère d'acceptation :} Tous les paramètres doivent être modifiables avec feedback visuel immédiat.
\end{objectifbox}

\vspace{0.3cm}

\begin{objectifbox}{RF18 : Calcul Prédictif de Propagation}
\textbf{Description :} Le système doit calculer le rayon de propagation estimé à 6 heures basé sur l'algorithme empirique.

\textbf{Algorithme de calcul :}

\begin{center}
\begin{tcolorbox}[
    enhanced,
    colback=white,
    colframe=accentTeal,
    boxrule=2pt,
    rounded corners,
    width=12cm
]
\begin{align*}
\text{Rayon} &= \text{Base} + f(\text{Brightness}) + f(\text{Confidence}) + f(\text{Wind})
\end{align*}

Où :
\begin{itemize}[leftmargin=0.5cm]
    \item $\text{Base} = 3.0$ km (rayon de base)
    \item $f(\text{Brightness})$ :
    \begin{itemize}
        \item $> 350K \rightarrow +3.0$ km
        \item $> 320K \rightarrow +1.5$ km
        \item sinon $\rightarrow +0.0$ km
    \end{itemize}
    \item $f(\text{Confidence})$ :
    \begin{itemize}
        \item high $\rightarrow +1.0$ km
        \item low $\rightarrow -1.0$ km
    \end{itemize}
    \item $f(\text{Wind}) = (\text{vitesse} / 30) \times 2.0$ km
\end{itemize}

Résultat final : $\max(1.0, \min(15.0, \text{Rayon}))$ km
\end{tcolorbox}
\end{center}

\textbf{Critère d'acceptation :} Le calcul doit s'exécuter en moins de 50ms avec une précision de 2 décimales.
\end{objectifbox}

\vspace{0.3cm}

\begin{objectifbox}{RF19 : Visualisation de la Zone de Danger}
\textbf{Description :} La zone de danger prédite doit être visualisée sur une carte interactive avec des indicateurs clairs.

\textbf{Spécifications détaillées :}
\begin{itemize}[leftmargin=1cm, itemsep=5pt]
    \item Cercle de propagation centré sur le hotspot
    \item Dégradé de couleur du centre (rouge intense) vers l'extérieur (jaune)
    \item Flèche indiquant la direction du vent
    \item Affichage des coordonnées du centre
    \item Estimation de la surface en km²
    \item Animation optionnelle de l'expansion
    \item Légende explicative des zones de risque
\end{itemize}

\textbf{Critère d'acceptation :} La visualisation doit se mettre à jour en temps réel lors de la modification des paramètres.
\end{objectifbox}

\vspace{0.5cm}

\begin{table}[H]
\centering
\caption{Exemples de calcul de propagation}
\label{tab:propagation}
\rowcolors{2}{mintGreen!30}{white}
\begin{tabular}{c c c c c}
\toprule
\rowcolor{primaryGreen}
\textcolor{white}{\textbf{Brightness}} & \textcolor{white}{\textbf{Confiance}} & \textcolor{white}{\textbf{Vent (km/h)}} & \textcolor{white}{\textbf{Rayon (km)}} & \textcolor{white}{\textbf{Surface (km²)}} \\
\midrule
320 K & Nominal & 20 & 5.83 & 106.9 \\
355 K & High & 40 & 9.67 & 293.6 \\
380 K & High & 60 & 11.00 & 380.1 \\
310 K & Low & 10 & 2.67 & 22.4 \\
400 K & High & 80 & 12.33 & 477.7 \\
\bottomrule
\end{tabular}
\end{table}

\subsection{Module de Notifications}

Le module de notifications assure la diffusion rapide et fiable des alertes vers les parties prenantes via différents canaux de communication. Cette composante est critique pour garantir une réponse rapide aux détections.

\subsubsection{Contexte et Justification}

La détection n'a de valeur que si elle est suivie d'une action. Le module de notifications transforme une détection technique en information actionnable pour les équipes terrain. L'utilisation de canaux multiples (\textbf{Email} et \textbf{Telegram}) garantit que l'alerte atteindra ses destinataires même en cas de défaillance d'un canal.

\begin{greenbox}[\faBell\ Philosophie du Système d'Alertes]
Notre système d'alertes repose sur trois principes fondamentaux :

\begin{enumerate}[leftmargin=1cm, itemsep=5pt]
    \item \textbf{Rapidité :} L'alerte doit être envoyée dans les secondes suivant la détection
    \item \textbf{Fiabilité :} Utilisation de canaux redondants (Email + Telegram)
    \item \textbf{Non-intrusivité :} Mécanisme anti-spam pour éviter la fatigue d'alerte
\end{enumerate}
\end{greenbox}

\subsubsection{Besoins Fonctionnels Détaillés}

\begin{objectifbox}{RF20 : Envoi d'Email avec Détails}
\textbf{Description :} Le système doit envoyer des alertes par email contenant toutes les informations pertinentes sur la détection.

\textbf{Spécifications détaillées :}
\begin{itemize}[leftmargin=1cm, itemsep=5pt]
    \item Envoi via SMTP (Gmail ou serveur configuré)
    \item Contenu HTML richement formaté :
    \begin{itemize}
        \item Titre d'alerte avec niveau de sévérité
        \item Zone géographique concernée
        \item Coordonnées GPS du point de détection
        \item Image satellite ou capture d'écran intégrée
        \item Prédiction de propagation
        \item Lien Google Maps pour localisation rapide
        \item Horodatage précis
    \end{itemize}
    \item Gestion des pièces jointes (images)
    \item Configuration des destinataires multiples
    \item Retry automatique en cas d'échec (3 tentatives)
\end{itemize}

\textbf{Critère d'acceptation :} L'email doit être envoyé dans les 10 secondes suivant la détection.
\end{objectifbox}

\vspace{0.3cm}

\begin{objectifbox}{RF21 : Envoi de Message Telegram}
\textbf{Description :} Le système doit envoyer des alertes instantanées via un bot Telegram.

\textbf{Spécifications détaillées :}
\begin{itemize}[leftmargin=1cm, itemsep=5pt]
    \item Intégration avec l'API Telegram Bot
    \item Message formaté avec emojis pour lisibilité rapide :
    \begin{itemize}
        \item 🔥 \textbf{ALERTE FEU DÉTECTÉ}
        \item 📍 Zone : [nom de la région]
        \item 🌡️ Confiance : [score]\%
        \item 📐 Propagation estimée : [rayon] km
        \item 🗺️ [Lien Google Maps]
    \end{itemize}
    \item Envoi d'image satellite avec le message
    \item Boutons inline pour actions rapides
    \item Notification push instantanée sur mobile
    \item Support des groupes et channels
\end{itemize}

\textbf{Critère d'acceptation :} Le message Telegram doit être reçu dans les 5 secondes suivant la détection.
\end{objectifbox}

\vspace{0.3cm}

\begin{objectifbox}{RF22 : Gestion du Cooldown Anti-Spam}
\textbf{Description :} Le système doit implémenter un mécanisme de cooldown pour éviter l'envoi d'alertes répétées pour la même détection.

\textbf{Spécifications détaillées :}
\begin{itemize}[leftmargin=1cm, itemsep=5pt]
    \item Cooldown par défaut : 30 secondes
    \item Cooldown configurable par zone (5s - 300s)
    \item Tracking par zone géographique + type de détection
    \item Reset du cooldown lors d'un changement significatif
    \item Logs des alertes ignorées pour audit
    \item Interface de monitoring des cooldowns actifs
\end{itemize}

\textbf{Critère d'acceptation :} Aucune alerte dupliquée ne doit être envoyée pendant la période de cooldown.
\end{objectifbox}

\subsection{Module FWI (Fire Weather Index)}

Le module FWI intègre l'indice météorologique de risque d'incendie, un standard international développé par le Service canadien des forêts. Cet indice combine plusieurs paramètres météorologiques pour évaluer le potentiel de départ et de propagation des feux.

\subsubsection{Besoins Fonctionnels Détaillés}

\begin{objectifbox}{RF23 : Affichage des Indices Météo}
\textbf{Description :} Le système doit afficher les indices météorologiques pertinents pour l'évaluation du risque d'incendie.

\textbf{Spécifications détaillées :}
\begin{itemize}[leftmargin=1cm, itemsep=5pt]
    \item Récupération des données via Open-Meteo API
    \item Paramètres affichés :
    \begin{itemize}
        \item Température (°C)
        \item Humidité relative (\%)
        \item Vitesse et direction du vent
        \item Précipitations (mm)
        \item Indice FWI calculé
    \end{itemize}
    \item Mise à jour automatique toutes les heures
    \item Historique sur 24h avec graphiques
    \item Prévisions à 48h
\end{itemize}

\textbf{Critère d'acceptation :} Les données météo doivent être affichées avec une latence maximale de 2 secondes.
\end{objectifbox}

\vspace{0.3cm}

\begin{objectifbox}{RF24 : Carte de Risque}
\textbf{Description :} Le système doit afficher une carte de risque d'incendie basée sur les indices FWI.

\textbf{Spécifications détaillées :}
\begin{itemize}[leftmargin=1cm, itemsep=5pt]
    \item Carte choroplèthe avec zones colorées par niveau de risque
    \item Échelle de risque standardisée :
    \begin{itemize}
        \item \textcolor{primaryGreen}{\textbf{Vert}} : Faible (FWI 0-5)
        \item \textcolor{sunYellow}{\textbf{Jaune}} : Modéré (FWI 5-10)
        \item \textcolor{alertOrange}{\textbf{Orange}} : Élevé (FWI 10-20)
        \item \textcolor{moroccanRed}{\textbf{Rouge}} : Très élevé (FWI 20-30)
        \item \textbf{Pourpre} : Extrême (FWI > 30)
    \end{itemize}
    \item Superposition optionnelle avec les hotspots actifs
    \item Export de la carte en image
    \item Légende interactive
\end{itemize}

\textbf{Critère d'acceptation :} La carte de risque doit se charger en moins de 3 secondes.
\end{objectifbox}

% ============================================
% CHAPITRE III - PARTIE 3 : BESOINS NON FONCTIONNELS ET MÉTHODOLOGIE
% ============================================

\newpage
\section{Analyse des Besoins Non Fonctionnels}

Les besoins non fonctionnels définissent \textbf{comment} le système doit fonctionner, en termes de qualité, de performance et de contraintes techniques. Ces exigences sont tout aussi importantes que les besoins fonctionnels car elles déterminent l'acceptabilité du système par les utilisateurs et sa viabilité à long terme.

Dans le contexte critique de la détection d'incendies, où chaque seconde compte, les exigences de performance, de fiabilité et de disponibilité revêtent une importance capitale.

\subsection{Performance}

La performance du système AI Sentinel est un facteur clé de succès. Un système de détection d'incendies doit être capable de traiter les données rapidement pour permettre une intervention précoce.

\begin{techbox}{Exigences de Performance}

\textbf{NF01 : Temps de Réponse API}

\begin{itemize}[leftmargin=1cm, itemsep=5pt]
    \item Temps de réponse moyen : < \textbf{2 secondes}
    \item Temps de réponse au 95ème percentile : < \textbf{5 secondes}
    \item Endpoints critiques (détection) : < \textbf{500 ms}
\end{itemize}

\vspace{0.3cm}

\textbf{NF02 : Traitement Vidéo en Temps Réel}

\begin{itemize}[leftmargin=1cm, itemsep=5pt]
    \item Fréquence de traitement : $\geq$ \textbf{25 FPS} sur GPU
    \item Fréquence minimale acceptable : $\geq$ \textbf{15 FPS} sur CPU
    \item Latence bout en bout : < \textbf{500 ms}
    \item Résolution supportée : jusqu'à \textbf{1080p}
\end{itemize}

\vspace{0.3cm}

\textbf{NF03 : Capacité de Traitement}

\begin{itemize}[leftmargin=1cm, itemsep=5pt]
    \item Support de \textbf{10 utilisateurs} simultanés minimum
    \item Traitement de \textbf{100 images/heure} pour classification
    \item Scan satellite de \textbf{8 zones} en moins de 10 minutes
\end{itemize}
\end{techbox}

\vspace{0.5cm}

\begin{center}
\begin{tikzpicture}
    % Graphique de performance
    \begin{axis}[
        width=12cm,
        height=6cm,
        xlabel={Composant},
        ylabel={Temps de réponse (ms)},
        ymin=0, ymax=3000,
        symbolic x coords={YOLOv8, MobileNet, CAM, API REST, Satellite, Notification},
        xtick=data,
        xticklabel style={rotate=45, anchor=east, font=\footnotesize},
        ylabel style={font=\small},
        xlabel style={font=\small},
        bar width=15pt,
        nodes near coords,
        nodes near coords style={font=\tiny},
        ybar,
        enlarge x limits=0.15,
    ]
    \addplot[fill=primaryGreen!70, draw=primaryGreen] coordinates {
        (YOLOv8, 33)
        (MobileNet, 150)
        (CAM, 200)
        (API REST, 100)
        (Satellite, 2000)
        (Notification, 500)
    };
    \end{axis}
    \node[font=\bfseries\small, text=darkGreen] at (6, -1) {Temps de Réponse Cibles par Composant};
\end{tikzpicture}
\end{center}

\subsection{Fiabilité}

La fiabilité du système est essentielle dans un contexte où des vies et des biens sont en jeu. Le système doit fonctionner de manière continue et prévisible.

\begin{greenbox}[\faShieldAlt\ Exigences de Fiabilité]

\textbf{NF04 : Disponibilité}

Le système doit garantir une disponibilité de \textbf{99\%} minimum, soit moins de \textbf{87.6 heures} d'indisponibilité par an. Cette disponibilité couvre :
\begin{itemize}[leftmargin=0.5cm, itemsep=3pt]
    \item Le backend API (FastAPI)
    \item Le frontend web (React)
    \item Les services de notification (Telegram, Email)
\end{itemize}

\vspace{0.3cm}

\textbf{NF05 : Tolérance aux Pannes}

\begin{itemize}[leftmargin=0.5cm, itemsep=5pt]
    \item \textbf{Dégradation gracieuse :} En cas d'indisponibilité d'un service externe (NASA, Sentinel), le reste du système continue de fonctionner
    \item \textbf{Retry automatique :} Les appels échoués sont automatiquement relancés (3 tentatives avec backoff exponentiel)
    \item \textbf{Circuit breaker :} Protection contre les cascades de pannes
    \item \textbf{Logs complets :} Enregistrement de toutes les erreurs pour diagnostic
\end{itemize}

\vspace{0.3cm}

\textbf{NF06 : Précision des Modèles IA}

\begin{table}[H]
\centering
\rowcolors{2}{mintGreen!30}{white}
\begin{tabular}{l c c}
\toprule
\rowcolor{primaryGreen}
\textcolor{white}{\textbf{Modèle}} & \textcolor{white}{\textbf{Précision Cible}} & \textcolor{white}{\textbf{Faux Positifs}} \\
\midrule
YOLOv8 (Détection) & $\geq$ 85\% & < 5\% \\
MobileNetV2 (Classification) & $\geq$ 97\% & < 3\% \\
CAM (Satellite) & $\geq$ 90\% & < 10\% \\
\bottomrule
\end{tabular}
\end{table}
\end{greenbox}

\subsection{Sécurité}

Bien que le système AI Sentinel soit principalement orienté vers la surveillance environnementale, les aspects de sécurité informatique restent importants pour protéger les données et garantir l'intégrité du système.

\begin{alertbox}{Exigences de Sécurité}

\textbf{NF07 : Authentification API}

\begin{itemize}[leftmargin=0.5cm, itemsep=5pt]
    \item Protection des endpoints API par clés d'authentification
    \item Rotation régulière des clés API externes (NASA, Sentinel Hub)
    \item Stockage sécurisé des credentials dans des variables d'environnement
    \item Rate limiting pour prévenir les abus
\end{itemize}

\vspace{0.3cm}

\textbf{NF08 : Protection des Données}

\begin{itemize}[leftmargin=0.5cm, itemsep=5pt]
    \item Chiffrement HTTPS pour toutes les communications
    \item Pas de stockage de données personnelles sensibles
    \item Anonymisation des logs si nécessaire
    \item Conformité RGPD pour les données utilisateur
\end{itemize}

\vspace{0.3cm}

\textbf{NF09 : Intégrité du Système}

\begin{itemize}[leftmargin=0.5cm, itemsep=5pt]
    \item Validation de toutes les entrées utilisateur (images, paramètres)
    \item Protection contre les injections (SQL, command injection)
    \item Limitation de la taille des fichiers uploadés
    \item Sandboxing du traitement des fichiers
\end{itemize}
\end{alertbox}

\subsection{Maintenabilité}

La maintenabilité garantit que le système peut évoluer et être corrigé efficacement au fil du temps.

\begin{infobox}{Exigences de Maintenabilité}

\textbf{NF10 : Code Modulaire}

\begin{itemize}[leftmargin=0.5cm, itemsep=5pt]
    \item Architecture en microservices logiques (YoloService, FirmsService, PredictionService, etc.)
    \item Séparation claire des responsabilités (Single Responsibility Principle)
    \item Injection de dépendances pour faciliter les tests
    \item Interfaces bien définies entre les modules
\end{itemize}

\vspace{0.3cm}

\textbf{NF11 : Documentation}

\begin{itemize}[leftmargin=0.5cm, itemsep=5pt]
    \item Documentation API Swagger/OpenAPI automatique
    \item Commentaires de code pour les sections complexes
    \item README avec instructions d'installation et d'utilisation
    \item Documentation technique d'architecture
\end{itemize}

\vspace{0.3cm}

\textbf{NF12 : Testabilité}

\begin{itemize}[leftmargin=0.5cm, itemsep=5pt]
    \item Tests unitaires pour les fonctions critiques
    \item Tests d'intégration pour les flux principaux
    \item Couverture de code cible : > 70\%
    \item Environnement de test isolé
\end{itemize}
\end{infobox}

\subsection{Ergonomie}

L'ergonomie et l'expérience utilisateur sont essentielles pour garantir l'adoption du système par les opérateurs terrain qui doivent pouvoir réagir rapidement.

\begin{greenbox}[\faDesktop\ Exigences d'Ergonomie]

\textbf{NF13 : Interface Intuitive}

\begin{itemize}[leftmargin=0.5cm, itemsep=5pt]
    \item Navigation claire avec menu principal accessible
    \item Actions principales accessibles en \textbf{moins de 3 clics}
    \item Feedback visuel immédiat pour toutes les actions
    \item Messages d'erreur explicites et actionables
    \item Thème visuel cohérent (palette verte nature)
\end{itemize}

\vspace{0.3cm}

\textbf{NF14 : Design Responsive}

\begin{itemize}[leftmargin=0.5cm, itemsep=5pt]
    \item Compatibilité : Desktop (1920×1080), Tablette (768px), Mobile (375px)
    \item Adaptation automatique de la mise en page
    \item Carte interactive utilisable sur écran tactile
    \item Performances optimisées pour les appareils mobiles
\end{itemize}

\vspace{0.3cm}

\textbf{NF15 : Accessibilité}

\begin{itemize}[leftmargin=0.5cm, itemsep=5pt]
    \item Contraste suffisant pour lisibilité
    \item Labels pour tous les éléments de formulaire
    \item Navigation au clavier possible
    \item Compatibilité avec les lecteurs d'écran (ARIA)
\end{itemize}
\end{greenbox}

\vspace{0.5cm}

% Tableau récapitulatif des besoins non fonctionnels
\begin{table}[H]
\centering
\caption{Récapitulatif des Besoins Non Fonctionnels}
\label{tab:bnf}
\rowcolors{2}{mintGreen!30}{white}
\begin{tabular}{l l p{6cm} l}
\toprule
\rowcolor{primaryGreen}
\textcolor{white}{\textbf{ID}} & \textcolor{white}{\textbf{Catégorie}} & \textcolor{white}{\textbf{Description}} & \textcolor{white}{\textbf{Priorité}} \\
\midrule
NF01 & Performance & Temps de réponse API < 2s & Critique \\
NF02 & Performance & Traitement vidéo $\geq$ 25 FPS & Critique \\
NF03 & Performance & 10 utilisateurs simultanés & Haute \\
NF04 & Fiabilité & Disponibilité 99\% & Critique \\
NF05 & Fiabilité & Tolérance aux pannes & Haute \\
NF06 & Fiabilité & Précision modèles $\geq$ 85\% & Critique \\
NF07 & Sécurité & Authentification API & Haute \\
NF08 & Sécurité & Protection des données & Haute \\
NF09 & Sécurité & Validation des entrées & Haute \\
NF10 & Maintenabilité & Code modulaire & Moyenne \\
NF11 & Maintenabilité & Documentation complète & Moyenne \\
NF12 & Maintenabilité & Couverture tests > 70\% & Moyenne \\
NF13 & Ergonomie & Interface intuitive & Haute \\
NF14 & Ergonomie & Design responsive & Haute \\
NF15 & Ergonomie & Accessibilité & Moyenne \\
\bottomrule
\end{tabular}
\end{table}

% ============================================
\newpage
\section{Méthodologie de Développement}
% ============================================

Le choix d'une méthodologie de développement adaptée est crucial pour la réussite d'un projet logiciel. Dans le cadre d'AI Sentinel, nous avons opté pour une approche \textbf{Agile} avec le framework \textbf{Scrum}, particulièrement adapté aux projets innovants nécessitant flexibilité et itérations rapides.

\subsection{Choix de la Méthodologie : Agile/Scrum}

\subsubsection{Justification du Choix}

Le développement d'un système de détection d'incendies par IA présente plusieurs caractéristiques qui oriententnaturellement vers une méthodologie agile :

\begin{greenbox}[\faList\ Pourquoi Agile/Scrum ?]

\textbf{Complexité Technique}

Le projet intègre des technologies diverses (IA, satellite, temps réel) dont l'intégration peut révéler des défis imprévus. Scrum permet d'adapter le plan au fil des découvertes.

\vspace{0.3cm}

\textbf{Besoins Évolutifs}

Les exigences des utilisateurs (autorités, gestionnaires forestiers) peuvent évoluer à mesure qu'ils découvrent les possibilités du système. Les sprints courts permettent d'intégrer leurs retours.

\vspace{0.3cm}

\textbf{Livraisons Incrémentales}

Chaque module (détection temps réel, satellite, prédiction) peut être développé et livré indépendamment, permettant des tests terrain précoces.

\vspace{0.3cm}

\textbf{Gestion des Risques}

Les revues de sprint régulières permettent d'identifier et de traiter rapidement les problèmes techniques ou fonctionnels.
\end{greenbox}

\subsubsection{Principes Scrum Appliqués}

\begin{center}
\begin{tikzpicture}[
    role/.style={ellipse, draw=primaryGreen, line width=2pt, fill=mintGreen, text width=2cm, align=center, font=\small, minimum height=1.5cm},
    artifact/.style={rectangle, rounded corners=8pt, draw=accentTeal, line width=1.5pt, fill=skyBlue!20, text width=2.5cm, align=center, font=\small, minimum height=1.5cm},
    event/.style={rectangle, rounded corners=8pt, draw=leafGreen, line width=1.5pt, fill=leafGreen!20, text width=2cm, align=center, font=\small, minimum height=1.2cm}
]
    % Titre
    \node[font=\Large\bfseries, text=darkGreen] at (6, 4) {Framework Scrum appliqué à AI Sentinel};
    
    % Rôles
    \node[role] (po) at (0, 2) {Product\\Owner};
    \node[role] (sm) at (0, 0) {Scrum\\Master};
    \node[role] (team) at (0, -2) {Dev\\Team};
    
    % Artefacts
    \node[artifact] (pb) at (4, 2) {Product\\Backlog};
    \node[artifact] (sb) at (4, 0) {Sprint\\Backlog};
    \node[artifact] (inc) at (4, -2) {Incrément\\Produit};
    
    % Événements
    \node[event] (sprint) at (8, 2) {Sprint\\(2 sem.)};
    \node[event] (plan) at (8, 0.5) {Sprint\\Planning};
    \node[event] (daily) at (8, -0.8) {Daily\\Standup};
    \node[event] (review) at (8, -2) {Sprint\\Review};
    
    % Connexions
    \draw[->, >=stealth, primaryGreen, line width=1pt] (po) -- (pb);
    \draw[->, >=stealth, primaryGreen, line width=1pt] (pb) -- (sb);
    \draw[->, >=stealth, primaryGreen, line width=1pt] (sb) -- (inc);
    \draw[->, >=stealth, leafGreen, line width=1pt] (sb) -- (sprint);
    
\end{tikzpicture}
\end{center}

\subsection{Planning des Sprints}

Le projet AI Sentinel a été planifié sur \textbf{6 sprints} de 2 semaines chacun, soit une durée totale de 12 semaines.

\begin{table}[H]
\centering
\caption{Planning des Sprints AI Sentinel}
\label{tab:sprints}
\rowcolors{2}{mintGreen!30}{white}
\begin{tabular}{c p{4cm} p{5cm} c}
\toprule
\rowcolor{primaryGreen}
\textcolor{white}{\textbf{Sprint}} & \textcolor{white}{\textbf{Objectif}} & \textcolor{white}{\textbf{Livrables}} & \textcolor{white}{\textbf{Durée}} \\
\midrule
1 & Setup \& Architecture & Environnement dev, structure projet, API base & 2 sem. \\
2 & Module Détection & YOLOv8 intégré, détection temps réel & 2 sem. \\
3 & Module Classification & MobileNetV2, upload images & 2 sem. \\
4 & Module Satellite & Sentinel Hub, NASA FIRMS & 2 sem. \\
5 & Module Prédiction & Algorithme propagation, carte & 2 sem. \\
6 & Notifications \& Polish & Email, Telegram, UI final & 2 sem. \\
\bottomrule
\end{tabular}
\end{table}

\vspace{0.3cm}

\begin{center}
\begin{tikzpicture}
    % Timeline
    \draw[line width=3pt, primaryGreen] (0,0) -- (14,0);
    
    % Sprints
    \foreach \x/\s/\label in {0/1/Setup, 2.33/2/Détection, 4.66/3/Classification, 7/4/Satellite, 9.33/5/Prédiction, 11.66/6/Notif.} {
        \fill[primaryGreen] (\x, 0) circle (6pt);
        \node[above, font=\footnotesize\bfseries] at (\x, 0.3) {S\s};
        \node[below, font=\tiny, text width=1.5cm, align=center] at (\x, -0.3) {\label};
    }
    \fill[leafGreen] (14, 0) circle (8pt);
    \node[above, font=\footnotesize\bfseries] at (14, 0.3) {Release};
    
    % Durées
    \node[font=\tiny] at (1.16, -1) {2 sem.};
    \node[font=\tiny] at (3.5, -1) {2 sem.};
    \node[font=\tiny] at (5.83, -1) {2 sem.};
    \node[font=\tiny] at (8.16, -1) {2 sem.};
    \node[font=\tiny] at (10.5, -1) {2 sem.};
    \node[font=\tiny] at (12.83, -1) {2 sem.};
\end{tikzpicture}
\end{center}

\subsection{Outils de Gestion}

La gestion efficace du projet repose sur un ensemble d'outils complémentaires permettant le suivi des tâches, la gestion du code source et la collaboration d'équipe.

\begin{techbox}{Stack d'Outils de Gestion de Projet}

\textbf{\faTrello\ Trello --- Gestion des Tâches}

\begin{itemize}[leftmargin=0.5cm, itemsep=3pt]
    \item Tableau Kanban avec colonnes : Backlog, To Do, In Progress, Review, Done
    \item Cartes pour chaque User Story avec checklist de sous-tâches
    \item Labels par module (Détection, Satellite, UI, etc.)
    \item Power-ups pour estimation en points et burndown
\end{itemize}

\vspace{0.3cm}

\textbf{\faGitAlt\ Git/GitHub --- Gestion du Code}

\begin{itemize}[leftmargin=0.5cm, itemsep=3pt]
    \item Repository centralisé avec branches par feature
    \item Convention de nommage : \texttt{feature/module-description}
    \item Pull Requests obligatoires pour merge sur \texttt{main}
    \item Code review par au moins 1 développeur
    \item GitHub Actions pour CI/CD automatisé
\end{itemize}

\vspace{0.3cm}

\textbf{\faSlack\ Communication}

\begin{itemize}[leftmargin=0.5cm, itemsep=3pt]
    \item Channels dédiés par sujet (dev, bugs, releases)
    \item Intégrations avec GitHub et Trello
    \item Daily standups asynchrones
\end{itemize}

\vspace{0.3cm}

\textbf{\faFileAlt\ Documentation}

\begin{itemize}[leftmargin=0.5cm, itemsep=3pt]
    \item Confluence/Notion pour documentation technique
    \item Swagger pour documentation API
    \item README.md dans chaque module
\end{itemize}
\end{techbox}

\vspace{0.5cm}

% Workflow Git
\begin{center}
\begin{tikzpicture}[
    branch/.style={rectangle, rounded corners=5pt, draw=accentTeal, line width=1.5pt, fill=skyBlue!20, text width=2cm, minimum height=1cm, align=center, font=\small},
    commit/.style={circle, draw=primaryGreen, line width=1.5pt, fill=mintGreen, minimum size=0.6cm},
    arrow/.style={->, >=stealth, line width=1.5pt}
]
    % Branches
    \node[font=\bfseries, text=darkGreen] at (0, 2) {main};
    \node[font=\bfseries, text=accentTeal] at (0, 0.5) {develop};
    \node[font=\bfseries, text=leafGreen] at (0, -1) {feature/x};
    
    % Commits main
    \foreach \x in {2, 6, 10, 14} {
        \node[commit] at (\x, 2) {};
    }
    \draw[darkGreen, line width=2pt] (1, 2) -- (15, 2);
    
    % Commits develop
    \foreach \x in {2, 4, 6, 8, 10, 12, 14} {
        \node[commit, fill=skyBlue!50] at (\x, 0.5) {};
    }
    \draw[accentTeal, line width=2pt] (1, 0.5) -- (15, 0.5);
    
    % Feature branch
    \foreach \x in {4, 5, 6} {
        \node[commit, fill=leafGreen!50] at (\x, -1) {};
    }
    \draw[leafGreen, line width=1.5pt] (3, 0.5) -- (4, -1) -- (6, -1) -- (7, 0.5);
    
    % Merge arrows
    \draw[arrow, accentTeal] (6, 0.5) -- (6, 1.8);
    \draw[arrow, accentTeal] (14, 0.5) -- (14, 1.8);
    
    \node[font=\bfseries\small, text=darkGreen] at (8, -2.5) {Workflow Git Flow};
\end{tikzpicture}
\end{center}

% ============================================
% CHAPITRE III - PARTIE 4 : MODÉLISATION UML
% ============================================

\newpage
\section{Modélisation UML}

La modélisation UML (Unified Modeling Language) permet de visualiser l'architecture et le comportement du système de manière standardisée. Cette section présente les différents diagrammes qui décrivent AI Sentinel sous ses multiples facettes : fonctionnelle, structurelle et comportementale.

\subsection{Diagramme de Cas d'Utilisation Global}

Le diagramme de cas d'utilisation global offre une vue d'ensemble des fonctionnalités du système et des interactions entre les acteurs et le système.

\vspace{0.5cm}
\begin{center}
\begin{tikzpicture}[
    actor/.style={font=\small},
    usecase/.style={ellipse, draw=primaryGreen, line width=1.5pt, fill=mintGreen!50, text width=2.5cm, align=center, font=\footnotesize, minimum height=1cm},
    system/.style={rectangle, draw=darkGreen, line width=2pt, rounded corners=10pt, minimum width=11cm, minimum height=10cm}
]
    % Système
    \node[system, label={[font=\Large\bfseries, text=darkGreen]above:Système AI Sentinel}] (sys) at (5, 0) {};
    
    % Acteurs
    \node[actor] (user) at (-2, 1) {\includegraphics[width=1cm]{example-image}};
    \node[font=\small\bfseries, below] at (-2, 0.3) {Utilisateur};
    
    \node[actor] (admin) at (12, 1) {\includegraphics[width=1cm]{example-image}};
    \node[font=\small\bfseries, below] at (12, 0.3) {Admin};
    
    \node[actor] (ext) at (5, -6.5) {};
    \node[font=\small\bfseries, text=accentTeal] at (5, -6.5) {\faCloud\ Services Externes};
    
    % Cas d'utilisation - Colonne gauche
    \node[usecase] (uc1) at (2, 3) {Détection\\temps réel};
    \node[usecase] (uc2) at (2, 1.5) {Upload\\média};
    \node[usecase] (uc3) at (2, 0) {Surveillance\\satellite};
    \node[usecase] (uc4) at (2, -1.5) {Visualiser\\carte};
    
    % Cas d'utilisation - Colonne droite
    \node[usecase] (uc5) at (8, 3) {Prédiction\\propagation};
    \node[usecase] (uc6) at (8, 1.5) {Configurer\\alertes};
    \node[usecase] (uc7) at (8, 0) {Consulter\\historique};
    \node[usecase] (uc8) at (8, -1.5) {Exporter\\rapports};
    
    % Cas Admin
    \node[usecase, fill=skyBlue!30] (uca1) at (5, -3) {Gérer\\monitoring};
    \node[usecase, fill=skyBlue!30] (uca2) at (5, -4.5) {Configurer\\système};
    
    % Connexions Utilisateur
    \draw[primaryGreen, line width=1pt] (user) -- (uc1);
    \draw[primaryGreen, line width=1pt] (user) -- (uc2);
    \draw[primaryGreen, line width=1pt] (user) -- (uc3);
    \draw[primaryGreen, line width=1pt] (user) -- (uc4);
    \draw[primaryGreen, line width=1pt] (-0.5, 1) -- (uc5);
    \draw[primaryGreen, line width=1pt] (-0.5, 1) -- (uc7);
    
    % Connexions Admin
    \draw[accentTeal, line width=1pt] (admin) -- (uc6);
    \draw[accentTeal, line width=1pt] (admin) -- (uc8);
    \draw[accentTeal, line width=1pt] (admin) -- (uca1);
    \draw[accentTeal, line width=1pt] (admin) -- (uca2);
    
    % Connexions Services Externes
    \draw[textGray, line width=1pt, dashed] (uc3) -- (ext);
    \draw[textGray, line width=1pt, dashed] (uc5) -- (ext);
    \draw[textGray, line width=1pt, dashed] (uca1) -- (ext);
    
\end{tikzpicture}
\end{center}

\vspace{0.5cm}

\begin{greenbox}[\faListOl\ Légende des Cas d'Utilisation]
\begin{minipage}[t]{0.48\textwidth}
\textbf{Cas Utilisateur :}
\begin{itemize}[leftmargin=0.5cm, itemsep=3pt]
    \item Détection temps réel (vidéo/webcam)
    \item Upload média (images/vidéos)
    \item Surveillance satellite
    \item Visualiser carte interactive
    \item Prédiction de propagation
    \item Consulter historique
\end{itemize}
\end{minipage}
\hfill
\begin{minipage}[t]{0.48\textwidth}
\textbf{Cas Administrateur :}
\begin{itemize}[leftmargin=0.5cm, itemsep=3pt]
    \item Configurer alertes
    \item Exporter rapports
    \item Gérer monitoring satellite
    \item Configurer système
\end{itemize}
\end{minipage}
\end{greenbox}

\subsection{Diagrammes de Cas d'Utilisation Détaillés}

\subsubsection{UC01 : Détection en Temps Réel}

\begin{center}
\begin{tikzpicture}[
    usecase/.style={ellipse, draw=primaryGreen, line width=1.5pt, fill=mintGreen!50, text width=2.2cm, align=center, font=\footnotesize, minimum height=0.9cm},
    extends/.style={->, >=stealth, dashed, line width=1pt, textGray},
    includes/.style={->, >=stealth, dashed, line width=1pt, accentTeal}
]
    % Acteur
    \node (user) at (-3, 0) {\faUser};
    \node[font=\small\bfseries, below] at (-3, -0.5) {Utilisateur};
    
    % Cas principal
    \node[usecase, fill=leafGreen!30, text width=3cm] (main) at (3, 0) {Détection\\Temps Réel};
    
    % Sous-cas
    \node[usecase] (uc1) at (7, 2) {Activer\\webcam};
    \node[usecase] (uc2) at (7, 0) {Visualiser\\flux};
    \node[usecase] (uc3) at (7, -2) {Afficher\\détections};
    \node[usecase] (uc4) at (11, 1) {Envoyer\\alerte};
    \node[usecase] (uc5) at (11, -1) {Sauvegarder\\capture};
    
    % Connexions
    \draw[primaryGreen, line width=1.5pt] (user) -- (main);
    \draw[includes] (main) -- node[above, font=\tiny] {<<include>>} (uc1);
    \draw[includes] (main) -- node[above, font=\tiny] {<<include>>} (uc2);
    \draw[includes] (main) -- node[below, font=\tiny] {<<include>>} (uc3);
    \draw[extends] (uc4) -- node[above, font=\tiny] {<<extends>>} (uc3);
    \draw[extends] (uc5) -- node[below, font=\tiny] {<<extends>>} (uc3);
    
\end{tikzpicture}
\end{center}

\subsubsection{UC02 : Surveillance Satellite}

\begin{center}
\begin{tikzpicture}[
    usecase/.style={ellipse, draw=accentTeal, line width=1.5pt, fill=skyBlue!30, text width=2.2cm, align=center, font=\footnotesize, minimum height=0.9cm},
    extends/.style={->, >=stealth, dashed, line width=1pt, textGray},
    includes/.style={->, >=stealth, dashed, line width=1pt, primaryGreen}
]
    % Acteurs
    \node (user) at (-3, 1) {\faUser};
    \node[font=\small\bfseries, below] at (-3, 0.5) {Utilisateur};
    
    \node (admin) at (-3, -2) {\faUserCog};
    \node[font=\small\bfseries, below] at (-3, -2.5) {Admin};
    
    % Cas principal
    \node[usecase, fill=accentTeal!30, text width=3cm] (main) at (3, 0) {Surveillance\\Satellite};
    
    % Sous-cas
    \node[usecase] (uc1) at (7, 2.5) {Sélectionner\\zone};
    \node[usecase] (uc2) at (7, 1) {Récupérer\\image Sentinel};
    \node[usecase] (uc3) at (7, -0.5) {Analyser\\par CAM};
    \node[usecase] (uc4) at (7, -2) {Afficher\\résultat};
    \node[usecase] (uc5) at (11, 0) {Démarrer\\scan auto};
    
    % Connexions
    \draw[primaryGreen, line width=1.5pt] (user) -- (main);
    \draw[primaryGreen, line width=1.5pt] (admin) -- (main);
    \draw[includes] (main) -- node[above, font=\tiny] {<<include>>} (uc1);
    \draw[includes] (main) -- node[above, font=\tiny] {<<include>>} (uc2);
    \draw[includes] (main) -- node[above, font=\tiny] {<<include>>} (uc3);
    \draw[includes] (main) -- node[below, font=\tiny] {<<include>>} (uc4);
    \draw[extends] (uc5) -- node[above, font=\tiny] {<<extends>>} (main);
    
\end{tikzpicture}
\end{center}

\subsection{Description Textuelle des Cas d'Utilisation}

\begin{table}[H]
\centering
\caption{UC01 : Détection en Temps Réel}
\label{tab:uc01}
\rowcolors{2}{mintGreen!20}{white}
\begin{tabular}{>{\bfseries}p{3cm} p{10cm}}
\toprule
\rowcolor{primaryGreen}
\multicolumn{2}{l}{\textcolor{white}{\textbf{UC01 : Détection en Temps Réel}}} \\
\midrule
Acteur principal & Utilisateur (Opérateur de surveillance) \\
Acteurs secondaires & Système de notification, Modèle YOLOv8 \\
Description & Permet à l'utilisateur de détecter en temps réel la présence de feu ou de fumée via une webcam ou caméra connectée \\
Préconditions & \begin{itemize}[leftmargin=0.3cm, topsep=0pt, itemsep=0pt]
    \item Caméra/webcam connectée et fonctionnelle
    \item Modèle YOLOv8 chargé en mémoire
    \item Connexion au backend établie
\end{itemize} \\
\midrule
\multicolumn{2}{l}{\textbf{\textcolor{primaryGreen}{Scénario nominal :}}} \\
\multicolumn{2}{p{13cm}}{
\begin{enumerate}[leftmargin=0.5cm, topsep=0pt, itemsep=2pt]
    \item L'utilisateur accède à la page /realtime
    \item L'utilisateur clique sur "Démarrer la détection"
    \item Le système active le flux vidéo de la webcam
    \item Le système affiche le flux vidéo en temps réel
    \item Le système analyse chaque frame avec YOLOv8
    \item Si feu/fumée détecté : affichage des bounding boxes
    \item Le système génère une alerte visuelle
    \item [Optionnel] Le système envoie une notification
\end{enumerate}
} \\
\midrule
Postconditions & \begin{itemize}[leftmargin=0.3cm, topsep=0pt, itemsep=0pt]
    \item Flux vidéo affiché avec annotations
    \item Alertes générées si détection positive
    \item Logs de détection enregistrés
\end{itemize} \\
Exceptions & \begin{itemize}[leftmargin=0.3cm, topsep=0pt, itemsep=0pt]
    \item E1 : Caméra non accessible → Message d'erreur
    \item E2 : Modèle non chargé → Tentative de rechargement
\end{itemize} \\
\bottomrule
\end{tabular}
\end{table}

\vspace{0.5cm}

\begin{table}[H]
\centering
\caption{UC02 : Upload et Analyse d'Image}
\label{tab:uc02}
\rowcolors{2}{mintGreen!20}{white}
\begin{tabular}{>{\bfseries}p{3cm} p{10cm}}
\toprule
\rowcolor{primaryGreen}
\multicolumn{2}{l}{\textcolor{white}{\textbf{UC02 : Upload et Analyse d'Image}}} \\
\midrule
Acteur principal & Utilisateur \\
Acteurs secondaires & Modèle MobileNetV2 \\
Description & Permet d'uploader une image et d'obtenir sa classification (Fire/Smoke/Non-Fire) \\
Préconditions & \begin{itemize}[leftmargin=0.3cm, topsep=0pt, itemsep=0pt]
    \item Image au format valide (JPEG, PNG)
    \item Taille < 10 MB
\end{itemize} \\
\midrule
\multicolumn{2}{l}{\textbf{\textcolor{primaryGreen}{Scénario nominal :}}} \\
\multicolumn{2}{p{13cm}}{
\begin{enumerate}[leftmargin=0.5cm, topsep=0pt, itemsep=2pt]
    \item L'utilisateur accède à la page /upload
    \item L'utilisateur sélectionne une image (drag-and-drop ou bouton)
    \item L'utilisateur clique sur "Analyser"
    \item Le système uploade l'image vers le backend
    \item Le backend prétraite l'image (224×224, normalisation)
    \item Le modèle MobileNetV2 effectue la classification
    \item Le système affiche le résultat : classe + score de confiance
\end{enumerate}
} \\
\midrule
Postconditions & Résultat de classification affiché avec probabilités \\
Exceptions & \begin{itemize}[leftmargin=0.3cm, topsep=0pt, itemsep=0pt]
    \item E1 : Format invalide → Message d'erreur avec formats acceptés
    \item E2 : Image trop grande → Message avec taille maximale
\end{itemize} \\
\bottomrule
\end{tabular}
\end{table}

\subsection{Diagramme de Séquence - Détection Temps Réel}

Ce diagramme illustre les interactions entre les composants lors d'une session de détection en temps réel.

\vspace{0.5cm}
\begin{center}
\begin{tikzpicture}[
    lifeline/.style={dashed, line width=1pt, textGray},
    message/.style={->, >=stealth, line width=1pt, primaryGreen},
    return/.style={->, >=stealth, dashed, line width=1pt, accentTeal},
    actor/.style={rectangle, draw=primaryGreen, line width=1.5pt, fill=mintGreen, minimum width=1.8cm, minimum height=0.8cm, align=center, font=\footnotesize}
]
    % Acteurs
    \node[actor] (user) at (0, 0) {Utilisateur};
    \node[actor] (front) at (3.5, 0) {Frontend\\React};
    \node[actor] (back) at (7, 0) {Backend\\FastAPI};
    \node[actor] (yolo) at (10.5, 0) {YOLOv8\\Model};
    \node[actor] (notif) at (14, 0) {Telegram\\Bot};
    
    % Lifelines
    \draw[lifeline] (0, -0.5) -- (0, -11);
    \draw[lifeline] (3.5, -0.5) -- (3.5, -11);
    \draw[lifeline] (7, -0.5) -- (7, -11);
    \draw[lifeline] (10.5, -0.5) -- (10.5, -11);
    \draw[lifeline] (14, -0.5) -- (14, -11);
    
    % Messages
    \draw[message] (0, -1.5) -- node[above, font=\tiny] {1: Accéder /realtime} (3.5, -1.5);
    \draw[return] (3.5, -2) -- node[above, font=\tiny] {Page chargée} (0, -2);
    
    \draw[message] (0, -2.8) -- node[above, font=\tiny] {2: Clic "Démarrer"} (3.5, -2.8);
    \draw[message] (3.5, -3.3) -- node[above, font=\tiny] {3: GET /video\_feed} (7, -3.3);
    
    % Boucle de détection
    \draw[darkGreen, line width=1pt] (6.2, -4) rectangle (11.3, -7.5);
    \node[font=\tiny\bfseries, text=darkGreen] at (7.5, -3.8) {loop [pour chaque frame]};
    
    \draw[message] (7, -4.5) -- node[above, font=\tiny] {4: capture\_frame()} (7, -4.5);
    \draw[message] (7, -5) -- node[above, font=\tiny] {5: predict(frame)} (10.5, -5);
    \draw[return] (10.5, -5.5) -- node[above, font=\tiny] {boxes, scores} (7, -5.5);
    \draw[message] (7, -6) -- node[above, font=\tiny] {6: draw\_boxes()} (7, -6);
    \draw[message] (7, -7) -- node[above, font=\tiny] {7: encode MJPEG} (7, -7);
    
    % Condition alerte
    \draw[alertOrange, line width=1pt] (6.2, -8) rectangle (14.8, -9.5);
    \node[font=\tiny\bfseries, text=alertOrange] at (8, -7.8) {alt [si Fire détecté]};
    
    \draw[message] (7, -8.5) -- node[above, font=\tiny] {8: send\_alert()} (14, -8.5);
    \draw[return] (14, -9) -- node[above, font=\tiny] {OK} (7, -9);
    
    % Retour stream
    \draw[return] (7, -10) -- node[above, font=\tiny] {MJPEG stream} (3.5, -10);
    \draw[return] (3.5, -10.5) -- node[above, font=\tiny] {Affichage vidéo} (0, -10.5);
    
\end{tikzpicture}
\end{center}

\subsection{Diagramme de Séquence - Surveillance Satellite}

\vspace{0.5cm}
\begin{center}
\begin{tikzpicture}[
    lifeline/.style={dashed, line width=1pt, textGray},
    message/.style={->, >=stealth, line width=1pt, accentTeal},
    return/.style={->, >=stealth, dashed, line width=1pt, primaryGreen},
    actor/.style={rectangle, draw=accentTeal, line width=1.5pt, fill=skyBlue!20, minimum width=1.5cm, minimum height=0.8cm, align=center, font=\footnotesize}
]
    % Acteurs
    \node[actor] (user) at (0, 0) {Admin};
    \node[actor] (front) at (2.8, 0) {Frontend};
    \node[actor] (back) at (5.6, 0) {Backend};
    \node[actor] (sent) at (8.4, 0) {Sentinel\\Hub};
    \node[actor] (cam) at (11.2, 0) {CAM\\Model};
    \node[actor] (notif) at (14, 0) {Email\\Service};
    
    % Lifelines
    \draw[lifeline] (0, -0.5) -- (0, -10);
    \draw[lifeline] (2.8, -0.5) -- (2.8, -10);
    \draw[lifeline] (5.6, -0.5) -- (5.6, -10);
    \draw[lifeline] (8.4, -0.5) -- (8.4, -10);
    \draw[lifeline] (11.2, -0.5) -- (11.2, -10);
    \draw[lifeline] (14, -0.5) -- (14, -10);
    
    % Messages
    \draw[message] (0, -1.2) -- node[above, font=\tiny] {1: Démarrer scan} (2.8, -1.2);
    \draw[message] (2.8, -1.7) -- node[above, font=\tiny] {2: POST /start\_monitoring} (5.6, -1.7);
    
    % Boucle zones
    \draw[darkGreen, line width=1pt] (4.8, -2.3) rectangle (14.8, -8);
    \node[font=\tiny\bfseries, text=darkGreen] at (6.5, -2.1) {loop [pour chaque zone]};
    
    \draw[message] (5.6, -3) -- node[above, font=\tiny] {3: get\_image(zone)} (8.4, -3);
    \draw[return] (8.4, -3.5) -- node[above, font=\tiny] {image satellite} (5.6, -3.5);
    
    \draw[message] (5.6, -4.2) -- node[above, font=\tiny] {4: analyze(image)} (11.2, -4.2);
    \draw[return] (11.2, -4.7) -- node[above, font=\tiny] {prediction, heatmap} (5.6, -4.7);
    
    % Condition feu
    \draw[moroccanRed, line width=1pt] (4.8, -5.3) rectangle (14.8, -7.5);
    \node[font=\tiny\bfseries, text=moroccanRed] at (6.8, -5.1) {alt [si Fire détecté]};
    
    \draw[message] (5.6, -6) -- node[above, font=\tiny] {5: send\_email(details)} (14, -6);
    \draw[return] (14, -6.5) -- node[above, font=\tiny] {sent OK} (5.6, -6.5);
    \draw[message] (5.6, -7) -- node[above, font=\tiny] {6: log\_detection()} (5.6, -7);
    
    % Retour
    \draw[return] (5.6, -8.5) -- node[above, font=\tiny] {scan results} (2.8, -8.5);
    \draw[return] (2.8, -9) -- node[above, font=\tiny] {Affichage rapport} (0, -9);
    
\end{tikzpicture}
\end{center}

\newpage
\subsection{Diagramme de Classes}

Le diagramme de classes présente la structure statique du système, avec les principales classes et leurs relations.

\vspace{0.5cm}
\begin{center}
\begin{tikzpicture}[
    class/.style={rectangle, draw=primaryGreen, line width=1.5pt, fill=mintGreen!30, rounded corners=3pt, minimum width=4cm, align=left, font=\footnotesize},
    inherit/.style={->, >=open triangle 60, line width=1pt, primaryGreen},
    compose/.style={-*, line width=1pt, darkGreen},
    assoc/.style={-, line width=1pt, textGray}
]
    % Classes Services
    \node[class] (yolo) at (0, 6) {
        \textbf{\textcolor{darkGreen}{YoloService}}\\
        \rule{3.8cm}{0.5pt}\\
        - model: YOLO\\
        - class\_names: list\\
        - confidence: float\\
        \rule{3.8cm}{0.5pt}\\
        + load\_model()\\
        + detect(frame)\\
        + draw\_boxes()
    };
    
    \node[class] (mobile) at (5, 6) {
        \textbf{\textcolor{darkGreen}{ClassificationService}}\\
        \rule{3.8cm}{0.5pt}\\
        - model: tf.Model\\
        - img\_size: tuple\\
        \rule{3.8cm}{0.5pt}\\
        + load\_model()\\
        + classify(image)\\
        + preprocess()
    };
    
    \node[class] (firms) at (10, 6) {
        \textbf{\textcolor{darkGreen}{FirmsService}}\\
        \rule{3.8cm}{0.5pt}\\
        - api\_key: str\\
        - regions: dict\\
        \rule{3.8cm}{0.5pt}\\
        + fetch\_data(region)\\
        + get\_wind\_data()\\
        + parse\_hotspots()
    };
    
    \node[class] (sentinel) at (0, 1) {
        \textbf{\textcolor{darkGreen}{SentinelService}}\\
        \rule{3.8cm}{0.5pt}\\
        - client\_id: str\\
        - client\_secret: str\\
        - zones: list\\
        \rule{3.8cm}{0.5pt}\\
        + authenticate()\\
        + get\_image(zone)\\
        + get\_evalscript()
    };
    
    \node[class] (prediction) at (5, 1) {
        \textbf{\textcolor{darkGreen}{PredictionService}}\\
        \rule{3.8cm}{0.5pt}\\
        - base\_radius: float\\
        \rule{3.8cm}{0.5pt}\\
        + calculate(params)\\
        + get\_brightness\_factor()\\
        + get\_wind\_factor()
    };
    
    \node[class] (monitoring) at (10, 1) {
        \textbf{\textcolor{darkGreen}{MonitoringService}}\\
        \rule{3.8cm}{0.5pt}\\
        - scheduler: APScheduler\\
        - is\_running: bool\\
        - interval: int\\
        \rule{3.8cm}{0.5pt}\\
        + start()\\
        + stop()\\
        + scan\_all\_zones()
    };
    
    \node[class] (notif) at (5, -4) {
        \textbf{\textcolor{darkGreen}{NotificationService}}\\
        \rule{3.8cm}{0.5pt}\\
        - telegram\_token: str\\
        - email\_config: dict\\
        - cooldown: int\\
        \rule{3.8cm}{0.5pt}\\
        + send\_telegram()\\
        + send\_email()\\
        + check\_cooldown()
    };
    
    % Relations
    \draw[compose] (monitoring) -- (sentinel);
    \draw[compose] (monitoring) -- (notif);
    \draw[assoc] (monitoring) -- (firms);
    \draw[assoc] (prediction) -- (firms);
    
\end{tikzpicture}
\end{center}

\subsection{Diagramme d'Activité - Processus de Détection}

\vspace{0.5cm}
\begin{center}
\begin{tikzpicture}[
    start/.style={circle, fill=darkGreen, minimum size=0.5cm},
    stop/.style={circle, draw=darkGreen, line width=2pt, fill=darkGreen, minimum size=0.5cm},
    activity/.style={rectangle, rounded corners=8pt, draw=primaryGreen, line width=1.5pt, fill=mintGreen!50, text width=2.5cm, minimum height=1cm, align=center, font=\small},
    decision/.style={diamond, draw=alertOrange, line width=1.5pt, fill=sunYellow!30, aspect=2, font=\small},
    arrow/.style={->, >=stealth, line width=1.5pt, primaryGreen}
]
    % Start
    \node[start] (start) at (0, 0) {};
    
    % Activities
    \node[activity] (a1) at (3, 0) {Initialiser\\caméra};
    \node[activity] (a2) at (6.5, 0) {Capturer\\frame};
    \node[activity] (a3) at (10, 0) {Prétraiter\\image};
    \node[activity] (a4) at (13.5, 0) {Inférence\\YOLO};
    
    \node[decision] (d1) at (13.5, -2.5) {Feu?};
    
    \node[activity] (a5) at (10, -2.5) {Dessiner\\boxes};
    \node[activity] (a6) at (16.5, -2.5) {Envoyer\\alerte};
    
    \node[activity] (a7) at (6.5, -2.5) {Encoder\\MJPEG};
    \node[activity] (a8) at (3, -2.5) {Streamer\\vers client};
    
    \node[decision] (d2) at (3, -5) {Stop?};
    
    \node[stop] (stop) at (0, -5) {};
    
    % Arrows
    \draw[arrow] (start) -- (a1);
    \draw[arrow] (a1) -- (a2);
    \draw[arrow] (a2) -- (a3);
    \draw[arrow] (a3) -- (a4);
    \draw[arrow] (a4) -- (d1);
    \draw[arrow] (d1) -- node[left, font=\tiny] {Non} (a5);
    \draw[arrow] (d1) -- node[above, font=\tiny] {Oui} (a6);
    \draw[arrow] (a6) |- (a5);
    \draw[arrow] (a5) -- (a7);
    \draw[arrow] (a7) -- (a8);
    \draw[arrow] (a8) -- (d2);
    \draw[arrow] (d2) -- node[above, font=\tiny] {Oui} (stop);
    \draw[arrow] (d2.east) -- ++(1,0) |- node[right, font=\tiny, pos=0.25] {Non} (a2.south);
    
\end{tikzpicture}
\end{center}

\subsection{Diagramme de Déploiement}

Le diagramme de déploiement illustre l'architecture physique du système et la répartition des composants sur les différents nœuds.

\vspace{0.5cm}
\begin{center}
\begin{tikzpicture}[
    node/.style={rectangle, draw=darkGreen, line width=2pt, fill=mintGreen!20, rounded corners=5pt, minimum width=4.5cm, minimum height=3cm},
    component/.style={rectangle, draw=accentTeal, line width=1pt, fill=skyBlue!20, rounded corners=3pt, minimum width=3.5cm, minimum height=0.7cm, font=\footnotesize},
    external/.style={rectangle, draw=textGray, line width=1.5pt, fill=softGray, rounded corners=5pt, minimum width=3cm, minimum height=1.5cm},
    arrow/.style={<->, >=stealth, line width=1.5pt, primaryGreen}
]
    % Nœud Client
    \node[node, label={[font=\small\bfseries, text=darkGreen]above:Client (Browser)}] (client) at (0, 0) {};
    \node[component] at (0, 0.5) {React Application};
    \node[component] at (0, -0.4) {Leaflet Maps};
    \node[component] at (0, -1.3) {TailwindCSS};
    
    % Nœud Serveur
    \node[node, minimum height=5cm, label={[font=\small\bfseries, text=darkGreen]above:Serveur Backend}] (server) at (7, 0) {};
    \node[component] at (7, 1.5) {FastAPI};
    \node[component] at (7, 0.6) {TensorFlow};
    \node[component] at (7, -0.3) {YOLOv8/Ultralytics};
    \node[component] at (7, -1.2) {APScheduler};
    \node[component] at (7, -2.1) {SMTP/Telegram};
    
    % Services Externes
    \node[external] (nasa) at (13, 2) {NASA FIRMS};
    \node[external] (sentinel) at (13, 0) {Sentinel Hub};
    \node[external] (meteo) at (13, -2) {Open-Meteo};
    
    % Connexions
    \draw[arrow] (client) -- node[above, font=\footnotesize] {HTTP/REST} node[below, font=\footnotesize] {MJPEG} (server);
    \draw[arrow] (server) -- node[above, font=\tiny] {API} (nasa);
    \draw[arrow] (server) -- node[above, font=\tiny] {OAuth2} (sentinel);
    \draw[arrow] (server) -- node[above, font=\tiny] {API} (meteo);
    
\end{tikzpicture}
\end{center}

\vspace{1cm}

% Transition vers le chapitre suivant
\begin{center}
\begin{tikzpicture}
    \node[
        fill=primaryGreen!10,
        draw=primaryGreen,
        line width=1.5pt,
        rounded corners=12pt,
        inner sep=20pt,
        text width=13cm,
        align=center
    ] {
        \textcolor{primaryGreen}{\fontsize{24}{28}\selectfont\faArrowCircleRight}\\[15pt]
        \large\textbf{Chapitre Suivant}\\[10pt]
        \normalsize Le prochain chapitre détaille la conception et l'architecture technique\\
        du système AI Sentinel, incluant l'architecture logicielle,\\
        la conception des modèles IA et l'interface utilisateur.\\[10pt]
        \textit{\textcolor{textGray}{Chapitre IV --- Conception et Architecture}}
    };
\end{tikzpicture}
\end{center}


%
% Ou copiez son contenu à la place du chapitre correspondant.
%
% Structure du chapitre :
% - 3.1 Présentation du Projet (Vision, Parties Prenantes)
% - 3.2 Analyse des Besoins Fonctionnels (8 modules)
% - 3.3 Analyse des Besoins Non Fonctionnels (Performance, Fiabilité, etc.)
% - 3.4 Méthodologie de Développement (Agile/Scrum)
% - 3.5 Modélisation UML (Use Cases, Séquences, Classes, etc.)
%
% Les fichiers sources sont divisés en 4 parties pour faciliter la maintenance :
% - chapitre3_partie1.tex : Sections 3.1 et 3.2 (partiellement)
% - chapitre3_partie2.tex : Suite de 3.2 (Satellite, Prévention, Prédiction, Notifications)
% - chapitre3_partie3.tex : Sections 3.3 et 3.4
% - chapitre3_partie4.tex : Section 3.5 (UML)
% ============================================

% ============================================
% CHAPITRE III - ANALYSE ET SPÉCIFICATION DES BESOINS
% ============================================
\chapter{Analyse et Spécification des Besoins}
\thispagestyle{fancy}

\section{Présentation du Projet}

\lettrine[lines=3, lhang=0.15, loversize=0.1, findent=3pt]{\textcolor{primaryGreen}{L}}{a phase} d'analyse et de spécification des besoins constitue une étape fondamentale dans tout projet de développement logiciel. Elle permet de définir avec précision ce que le système doit accomplir, comment il doit se comporter, et quelles contraintes il doit respecter. Dans le cadre du projet \textbf{AI Sentinel}, cette phase revêt une importance particulière compte tenu de la criticité du domaine d'application : la détection précoce des incendies de forêt, où chaque minute gagnée peut sauver des vies et préserver des hectares de forêt.

Ce chapitre présente une analyse exhaustive des besoins fonctionnels et non fonctionnels du système, accompagnée d'une modélisation UML détaillée permettant de visualiser les interactions entre les différents acteurs et composants du système.

\subsection{Vision Globale}

Le projet \textbf{AI Sentinel} s'inscrit dans une vision ambitieuse : créer un écosystème technologique complet capable de détecter, surveiller, prédire et alerter en temps réel sur les risques d'incendies de forêt. Cette vision se concrétise à travers une plateforme web full-stack qui intègre les technologies les plus avancées en matière d'intelligence artificielle et de traitement d'images.

\begin{greenbox}[\faEye\ Vision du Projet AI Sentinel]
Notre ambition va au-delà de la simple détection : nous souhaitons fournir aux autorités et gestionnaires forestiers un \textbf{outil d'aide à la décision} complet qui leur permette d'anticiper les risques, d'optimiser leurs ressources, et d'intervenir de manière proactive plutôt que réactive.

La plateforme AI Sentinel se positionne comme un \textbf{hub centralisé} intégrant :
\begin{itemize}[leftmargin=1cm, itemsep=5pt]
    \item[\textcolor{primaryGreen}{\faVideo}] L'analyse vidéo en temps réel par intelligence artificielle
    \item[\textcolor{primaryGreen}{\faSatellite}] La surveillance satellite à couverture globale
    \item[\textcolor{primaryGreen}{\faChartLine}] Les algorithmes prédictifs de propagation des feux
    \item[\textcolor{primaryGreen}{\faBell}] Un système d'alertes multi-canaux réactif
    \item[\textcolor{primaryGreen}{\faCloudSun}] L'intégration des données météorologiques
\end{itemize}
\end{greenbox}

\vspace{0.5cm}

L'objectif principal est de réduire significativement le \textbf{temps de détection} des incendies, facteur critique dans la limitation des dégâts. Les études montrent qu'une intervention dans les 15 premières minutes suivant le départ d'un feu permet de contenir 90\% des incendies avant qu'ils ne deviennent incontrôlables.

\begin{center}
\begin{tikzpicture}[
    block/.style={rectangle, rounded corners=10pt, draw=primaryGreen, line width=2pt, fill=mintGreen, text width=3cm, minimum height=2cm, align=center, font=\small},
    arrow/.style={->, >=stealth, line width=2pt, color=leafGreen}
]
    % Blocs
    \node[block] (detect) at (0,0) {\textcolor{darkGreen}{\faSearch}\\[5pt]\textbf{Détection}\\Rapide};
    \node[block] (analyse) at (4,0) {\textcolor{darkGreen}{\faBrain}\\[5pt]\textbf{Analyse}\\IA};
    \node[block] (predict) at (8,0) {\textcolor{darkGreen}{\faChartArea}\\[5pt]\textbf{Prédiction}\\Propagation};
    \node[block] (alert) at (12,0) {\textcolor{darkGreen}{\faBell}\\[5pt]\textbf{Alerte}\\Instantanée};
    
    % Flèches
    \draw[arrow] (detect) -- (analyse);
    \draw[arrow] (analyse) -- (predict);
    \draw[arrow] (predict) -- (alert);
    
    % Temps
    \node[font=\footnotesize, text=textGray] at (2, -1.5) {< 500ms};
    \node[font=\footnotesize, text=textGray] at (6, -1.5) {< 1s};
    \node[font=\footnotesize, text=textGray] at (10, -1.5) {< 2s};
    
    % Titre
    \node[font=\bfseries\large, text=darkGreen] at (6, 2) {Chaîne de Traitement AI Sentinel};
\end{tikzpicture}
\end{center}

\subsection{Parties Prenantes}

L'identification et la compréhension des parties prenantes sont essentielles pour garantir que le système réponde aux besoins réels de ses utilisateurs. Le projet AI Sentinel implique plusieurs catégories d'acteurs, chacune ayant des attentes et des besoins spécifiques.

\subsubsection{Utilisateurs Finaux}

Les utilisateurs finaux représentent le cœur de notre cible. Ils constituent les bénéficiaires directs du système et leurs besoins orientent l'ensemble des choix de conception.

\begin{infobox}{Profils Utilisateurs Identifiés}

\textbf{\faUserShield\ Autorités de Protection Civile}

Les services de protection civile et les pompiers forestiers représentent les utilisateurs principaux du système. Leur mission consiste à surveiller les zones à risque, coordonner les interventions et gérer les ressources humaines et matérielles. Ils requièrent :
\begin{itemize}[leftmargin=0.5cm, itemsep=3pt]
    \item Des alertes instantanées et fiables
    \item Une localisation précise des foyers d'incendie
    \item Des informations sur la propagation probable
    \item Un accès rapide aux données météorologiques
\end{itemize}

\vspace{0.3cm}

\textbf{\faTree\ Gestionnaires Forestiers}

Les agents des Eaux et Forêts et les gestionnaires d'espaces naturels utilisent le système pour la surveillance quotidienne de leurs territoires. Leurs besoins incluent :
\begin{itemize}[leftmargin=0.5cm, itemsep=3pt]
    \item La surveillance continue des zones forestières
    \item L'identification des zones à haut risque
    \item L'historique des incidents pour l'analyse
    \item La planification préventive des interventions
\end{itemize}

\vspace{0.3cm}

\textbf{\faCity\ Collectivités Territoriales}

Les communes et régions concernées par les zones forestières peuvent utiliser le système pour :
\begin{itemize}[leftmargin=0.5cm, itemsep=3pt]
    \item La coordination avec les services d'urgence
    \item L'information et l'alerte des populations
    \item La gestion des évacuations si nécessaire
\end{itemize}
\end{infobox}

\subsubsection{Administrateurs Système}

Les administrateurs techniques sont responsables du bon fonctionnement, de la maintenance et de l'évolution du système. Leur rôle est crucial pour garantir la disponibilité et les performances de la plateforme.

\begin{techbox}{Responsabilités des Administrateurs}
\begin{itemize}[leftmargin=0.5cm, itemsep=8pt]
    \item[\textcolor{accentTeal}{\faServer}] \textbf{Gestion de l'Infrastructure :} Déploiement, monitoring et maintenance des serveurs backend et frontend
    
    \item[\textcolor{accentTeal}{\faCogs}] \textbf{Configuration des Services :} Paramétrage des API externes (NASA FIRMS, Sentinel Hub), gestion des clés API et des quotas
    
    \item[\textcolor{accentTeal}{\faBrain}] \textbf{Gestion des Modèles IA :} Mise à jour et réentraînement des modèles de détection (YOLOv8, MobileNetV2)
    
    \item[\textcolor{accentTeal}{\faBell}] \textbf{Configuration des Alertes :} Paramétrage des canaux de notification (Telegram, Email) et gestion des destinataires
    
    \item[\textcolor{accentTeal}{\faChartBar}] \textbf{Monitoring des Performances :} Suivi des métriques de performance et optimisation continue
\end{itemize}
\end{techbox}

\subsubsection{Équipe de Développement}

L'équipe de développement assure la conception, l'implémentation et l'évolution continue du système. Elle est composée de profils complémentaires travaillant en synergie.

\begin{table}[H]
\centering
\caption{Composition de l'équipe de développement}
\label{tab:equipe}
\rowcolors{2}{mintGreen!30}{white}
\begin{tabular}{l l p{7cm}}
\toprule
\rowcolor{primaryGreen}
\textcolor{white}{\textbf{Rôle}} & \textcolor{white}{\textbf{Compétences}} & \textcolor{white}{\textbf{Responsabilités}} \\
\midrule
Développeur Backend & Python, FastAPI, IA & API REST, intégration modèles IA, services externes \\
Développeur Frontend & React, TypeScript & Interface utilisateur, cartographie, UX/UI \\
Ingénieur IA/ML & TensorFlow, PyTorch & Entraînement modèles, optimisation, évaluation \\
DevOps & Docker, CI/CD & Déploiement, automatisation, monitoring \\
\bottomrule
\end{tabular}
\end{table}

\vspace{0.5cm}

\begin{center}
\begin{tikzpicture}[
    actor/.style={circle, draw=primaryGreen, line width=2pt, fill=mintGreen, minimum size=2cm, align=center, font=\small},
    system/.style={rectangle, rounded corners=15pt, draw=darkGreen, line width=3pt, fill=leafGreen!20, minimum width=6cm, minimum height=3cm, align=center}
]
    % Système central
    \node[system] (sys) at (0,0) {\textbf{\large AI Sentinel}\\[5pt]\footnotesize Système de Détection\\des Incendies de Forêt};
    
    % Acteurs
    \node[actor] (user) at (-5, 2) {\faUser\\Utilisateur};
    \node[actor] (admin) at (5, 2) {\faUserCog\\Admin};
    \node[actor] (dev) at (-5, -2) {\faCode\\Développeur};
    \node[actor] (ext) at (5, -2) {\faCloud\\Services\\Externes};
    
    % Connexions
    \draw[->, >=stealth, line width=1.5pt, primaryGreen] (user) -- (sys);
    \draw[->, >=stealth, line width=1.5pt, primaryGreen] (admin) -- (sys);
    \draw[->, >=stealth, line width=1.5pt, primaryGreen] (dev) -- (sys);
    \draw[<->, >=stealth, line width=1.5pt, accentTeal] (ext) -- (sys);
    
\end{tikzpicture}
\end{center}

% ============================================
\newpage
\section{Analyse des Besoins Fonctionnels}
% ============================================

L'analyse des besoins fonctionnels définit ce que le système doit \textbf{faire}. Chaque besoin fonctionnel décrit une fonctionnalité spécifique que le système doit offrir à ses utilisateurs. Dans le cadre d'AI Sentinel, nous avons organisé ces besoins en \textbf{huit modules} distincts, chacun répondant à un aspect particulier de la détection et de la prévention des incendies.

\begin{greenbox}[\faListOl\ Vue d'Ensemble des Modules Fonctionnels]
\begin{center}
\begin{tikzpicture}[
    module/.style={rectangle, rounded corners=8pt, draw=primaryGreen, line width=1.5pt, fill=mintGreen, text width=3.8cm, minimum height=1.8cm, align=center, font=\small}
]
    % Ligne 1
    \node[module] (m1) at (0,0) {\textcolor{darkGreen}{\faVideo}\\[3pt]\textbf{Détection}\\Temps Réel};
    \node[module] (m2) at (4.5,0) {\textcolor{darkGreen}{\faImage}\\[3pt]\textbf{Classification}\\d'Images};
    \node[module] (m3) at (9,0) {\textcolor{darkGreen}{\faFilm}\\[3pt]\textbf{Analyse}\\Vidéo};
    \node[module] (m4) at (13.5,0) {\textcolor{darkGreen}{\faSatellite}\\[3pt]\textbf{Surveillance}\\Satellite};
    
    % Ligne 2
    \node[module] (m5) at (0,-2.5) {\textcolor{darkGreen}{\faMapMarkerAlt}\\[3pt]\textbf{Prévention}\\Hotspots};
    \node[module] (m6) at (4.5,-2.5) {\textcolor{darkGreen}{\faChartLine}\\[3pt]\textbf{Prédiction}\\Propagation};
    \node[module] (m7) at (9,-2.5) {\textcolor{darkGreen}{\faBell}\\[3pt]\textbf{Notifications}\\Multi-canaux};
    \node[module] (m8) at (13.5,-2.5) {\textcolor{darkGreen}{\faCloudSun}\\[3pt]\textbf{FWI}\\Météo};
\end{tikzpicture}
\end{center}
\end{greenbox}

\subsection{Module de Détection en Temps Réel}

Le module de détection en temps réel constitue le \textbf{cœur} du système AI Sentinel. Il permet l'analyse continue de flux vidéo provenant de caméras de surveillance pour détecter instantanément la présence de feu ou de fumée. Ce module exploite la puissance du modèle \textbf{YOLOv8} (You Only Look Once, version 8), un algorithme de détection d'objets state-of-the-art reconnu pour sa rapidité et sa précision.

\subsubsection{Contexte et Justification}

La détection en temps réel répond à un besoin critique : identifier les incendies \textbf{dès leur déclenchement}, avant qu'ils ne se propagent de manière incontrôlable. Les méthodes traditionnelles de surveillance (tours de guet, patrouilles) présentent des limitations importantes en termes de couverture, de disponibilité 24h/24, et de vitesse de détection. L'utilisation de l'intelligence artificielle permet de surmonter ces obstacles en offrant une surveillance automatisée, continue et objective.

\begin{alertbox}{Importance de la Réactivité}
Selon les études de l'Office National des Forêts (ONF), un feu de forêt peut doubler de taille toutes les \textbf{15 à 20 minutes} dans des conditions favorables à la propagation (vent fort, végétation sèche). Chaque minute gagnée dans la détection représente potentiellement des hectares de forêt préservés et des vies sauvées.
\end{alertbox}

\subsubsection{Besoins Fonctionnels Détaillés}

\begin{objectifbox}{RF01 : Capture Vidéo via Caméra/Webcam}
\textbf{Description :} Le système doit être capable de capturer un flux vidéo en temps réel à partir de différentes sources : webcam connectée, caméra IP, ou fichier vidéo local.

\textbf{Spécifications détaillées :}
\begin{itemize}[leftmargin=1cm, itemsep=5pt]
    \item Prise en charge des webcams USB standard via l'index de périphérique
    \item Support des flux RTSP pour les caméras IP professionnelles
    \item Capacité à traiter des fichiers vidéo uploadés (MP4, AVI, MOV)
    \item Résolution supportée : de 480p à 1080p
    \item Fréquence d'images : 25-30 FPS minimum
\end{itemize}

\textbf{Critère d'acceptation :} Le système doit démarrer la capture vidéo en moins de 2 secondes après activation par l'utilisateur.
\end{objectifbox}

\vspace{0.3cm}

\begin{objectifbox}{RF02 : Détection Automatique Feu/Fumée}
\textbf{Description :} Le système doit analyser chaque frame du flux vidéo pour détecter la présence de feu ou de fumée avec une précision élevée.

\textbf{Spécifications détaillées :}
\begin{itemize}[leftmargin=1cm, itemsep=5pt]
    \item Utilisation du modèle YOLOv8 personnalisé entraîné sur un dataset de feux de forêt
    \item Classification en deux classes : \texttt{Fire} (feu) et \texttt{Smoke} (fumée)
    \item Seuil de confiance configurable (par défaut : 0.5)
    \item Traitement à minimum 30 FPS sur GPU, 15 FPS sur CPU
    \item Détection multi-instances (plusieurs feux/fumées simultanés)
\end{itemize}

\textbf{Critère d'acceptation :} Le modèle doit atteindre une précision de détection $\geq$ 85\% avec un taux de faux positifs < 5\%.
\end{objectifbox}

\vspace{0.3cm}

\begin{objectifbox}{RF03 : Affichage des Bounding Boxes}
\textbf{Description :} Les objets détectés (feu, fumée) doivent être encadrés visuellement sur le flux vidéo avec des informations contextuelles.

\textbf{Spécifications détaillées :}
\begin{itemize}[leftmargin=1cm, itemsep=5pt]
    \item Rectangles de délimitation (bounding boxes) colorés selon la classe
    \item Code couleur : \textcolor{moroccanRed}{Rouge} pour le feu, \textcolor{textGray}{Gris} pour la fumée
    \item Affichage du nom de la classe et du score de confiance (ex: "Fire 0.92")
    \item Épaisseur et taille de police adaptatives selon la résolution
    \item Mise à jour en temps réel synchronisée avec le flux vidéo
\end{itemize}

\textbf{Critère d'acceptation :} Les annotations visuelles ne doivent pas dégrader les performances de plus de 5\%.
\end{objectifbox}

\vspace{0.3cm}

\begin{objectifbox}{RF04 : Génération d'Alertes Visuelles}
\textbf{Description :} Lorsqu'un feu ou une fumée est détecté avec un niveau de confiance suffisant, le système doit générer des alertes visuelles immédiates.

\textbf{Spécifications détaillées :}
\begin{itemize}[leftmargin=1cm, itemsep=5pt]
    \item Indicateur visuel clignotant sur l'interface lors d'une détection
    \item Changement de couleur de l'arrière-plan ou du cadre vidéo
    \item Notification toast affichant les détails de la détection
    \item Son d'alerte optionnel (configurable par l'utilisateur)
    \item Horodatage précis de chaque détection
\end{itemize}

\textbf{Critère d'acceptation :} L'alerte visuelle doit apparaître dans un délai maximum de 500ms après la détection.
\end{objectifbox}

\vspace{0.5cm}

\begin{center}
\begin{tikzpicture}[
    step/.style={rectangle, rounded corners=5pt, draw=accentTeal, line width=1.5pt, fill=skyBlue!20, text width=2.5cm, minimum height=1.5cm, align=center, font=\footnotesize},
    arrow/.style={->, >=stealth, line width=1.5pt, color=accentTeal}
]
    \node[step] (s1) at (0,0) {\faVideo\\Capture\\Vidéo};
    \node[step] (s2) at (3.5,0) {\faCropAlt\\Prétraitement\\Frame};
    \node[step] (s3) at (7,0) {\faBrain\\Inférence\\YOLOv8};
    \node[step] (s4) at (10.5,0) {\faVectorSquare\\Bounding\\Boxes};
    \node[step] (s5) at (14,0) {\faBell\\Alerte\\Visuelle};
    
    \draw[arrow] (s1) -- (s2);
    \draw[arrow] (s2) -- (s3);
    \draw[arrow] (s3) -- (s4);
    \draw[arrow] (s4) -- (s5);
    
    \node[font=\bfseries, text=darkGreen] at (7, 1.5) {Pipeline de Détection Temps Réel};
\end{tikzpicture}
\end{center}

\subsection{Module de Classification d'Images}

Le module de classification d'images permet aux utilisateurs d'analyser des photographies statiques pour déterminer si elles contiennent des signes de feu, de fumée, ou aucun des deux. Ce module utilise l'architecture \textbf{MobileNetV2} avec transfer learning, offrant un excellent compromis entre précision et performance.

\subsubsection{Contexte et Justification}

Complémentaire à la détection temps réel, la classification d'images répond à plusieurs cas d'usage importants :
\begin{itemize}[leftmargin=1cm, itemsep=5pt]
    \item \textbf{Vérification manuelle :} Confirmation par l'opérateur d'une image suspecte
    \item \textbf{Analyse rétrospective :} Examen d'images historiques
    \item \textbf{Traitement hors ligne :} Analyse sans connexion caméra en direct
    \item \textbf{Rapports et documentation :} Génération de preuves visuelles
\end{itemize}

\subsubsection{Besoins Fonctionnels Détaillés}

\begin{objectifbox}{RF05 : Upload d'Images}
\textbf{Description :} L'utilisateur doit pouvoir soumettre une ou plusieurs images au système pour analyse.

\textbf{Spécifications détaillées :}
\begin{itemize}[leftmargin=1cm, itemsep=5pt]
    \item Formats supportés : JPEG, PNG, WebP, BMP
    \item Taille maximale par image : 10 MB
    \item Interface drag-and-drop intuitive
    \item Prévisualisation de l'image avant soumission
    \item Upload multiple avec file d'attente de traitement
\end{itemize}

\textbf{Critère d'acceptation :} L'upload d'une image de 5 MB doit se terminer en moins de 3 secondes sur une connexion standard.
\end{objectifbox}

\vspace{0.3cm}

\begin{objectifbox}{RF06 : Classification Multi-Classes}
\textbf{Description :} Le système doit classifier l'image uploadée dans l'une des trois catégories : Fire, Smoke, ou Non-Fire.

\textbf{Spécifications détaillées :}
\begin{itemize}[leftmargin=1cm, itemsep=5pt]
    \item Modèle MobileNetV2 pré-entraîné sur ImageNet, fine-tuné sur dataset feu
    \item Trois classes de sortie avec probabilités associées
    \item Prétraitement automatique (redimensionnement 224×224, normalisation)
    \item Temps d'inférence < 500ms par image
    \item Précision globale $\geq$ 97\% sur le jeu de test
\end{itemize}

\textbf{Critère d'acceptation :} Le système doit retourner la classe prédite avec son score de confiance en moins de 1 seconde.
\end{objectifbox}

\vspace{0.3cm}

\begin{objectifbox}{RF07 : Affichage du Score de Confiance}
\textbf{Description :} Le résultat de la classification doit inclure un score de confiance permettant à l'utilisateur d'évaluer la fiabilité de la prédiction.

\textbf{Spécifications détaillées :}
\begin{itemize}[leftmargin=1cm, itemsep=5pt]
    \item Score de confiance exprimé en pourcentage (0-100\%)
    \item Affichage visuel avec code couleur (vert > 80\%, orange 50-80\%, rouge < 50\%)
    \item Distribution des probabilités pour les trois classes
    \item Indicateur visuel du niveau de certitude du modèle
    \item Recommandation d'action basée sur le niveau de confiance
\end{itemize}

\textbf{Critère d'acceptation :} L'interface doit afficher clairement le résultat avec les probabilités pour chaque classe.
\end{objectifbox}

\vspace{0.5cm}

\begin{table}[H]
\centering
\caption{Exemple de sortie du module de classification}
\label{tab:classification}
\rowcolors{2}{mintGreen!30}{white}
\begin{tabular}{l c l}
\toprule
\rowcolor{primaryGreen}
\textcolor{white}{\textbf{Classe}} & \textcolor{white}{\textbf{Probabilité}} & \textcolor{white}{\textbf{Interprétation}} \\
\midrule
\textbf{Fire} & 92.3\% & \textcolor{moroccanRed}{Détection confirmée} \\
Smoke & 5.1\% & Trace possible \\
Non-Fire & 2.6\% & Négligeable \\
\bottomrule
\end{tabular}
\end{table}

\subsection{Module d'Analyse Vidéo}

Le module d'analyse vidéo étend les capacités de détection aux fichiers vidéo préenregistrés. Il permet un traitement exhaustif frame par frame et génère une vidéo annotée avec les détections.

\subsubsection{Besoins Fonctionnels Détaillés}

\begin{objectifbox}{RF08 : Upload de Vidéos}
\textbf{Description :} L'utilisateur doit pouvoir soumettre des fichiers vidéo pour analyse complète.

\textbf{Spécifications détaillées :}
\begin{itemize}[leftmargin=1cm, itemsep=5pt]
    \item Formats supportés : MP4, AVI, MOV, MKV, WebM
    \item Taille maximale : 500 MB (configurable)
    \item Barre de progression de l'upload
    \item Validation du format avant traitement
    \item Support des différentes résolutions (480p à 4K)
\end{itemize}
\end{objectifbox}

\vspace{0.3cm}

\begin{objectifbox}{RF09 : Traitement Frame par Frame}
\textbf{Description :} Chaque frame de la vidéo doit être analysée individuellement par le modèle de détection.

\textbf{Spécifications détaillées :}
\begin{itemize}[leftmargin=1cm, itemsep=5pt]
    \item Extraction et analyse de chaque frame
    \item Affichage de la progression du traitement
    \item Statistiques en temps réel (frames traitées, détections)
    \item Possibilité d'annuler le traitement en cours
    \item Traitement asynchrone pour ne pas bloquer l'interface
\end{itemize}
\end{objectifbox}

\vspace{0.3cm}

\begin{objectifbox}{RF10 : Export Vidéo Annotée}
\textbf{Description :} Le système doit générer une version annotée de la vidéo avec les détections superposées.

\textbf{Spécifications détaillées :}
\begin{itemize}[leftmargin=1cm, itemsep=5pt]
    \item Génération d'une nouvelle vidéo avec bounding boxes
    \item Conservation de la résolution et du framerate d'origine
    \item Téléchargement direct du fichier résultat
    \item Rapport JSON des détections (timestamps, coordonnées, classes)
    \item Prévisualisation avant téléchargement
\end{itemize}
\end{objectifbox}

% ============================================
% CHAPITRE III - PARTIE 2 : MODULES SATELLITE, PRÉVENTION, PRÉDICTION
% ============================================

\subsection{Module de Surveillance Satellite}

Le module de surveillance satellite représente une composante stratégique du système AI Sentinel. Il permet d'étendre la couverture de détection au-delà des zones équipées de caméras en exploitant les images satellites du programme européen \textbf{Copernicus} via l'API \textbf{Sentinel Hub}, ainsi que les données de hotspots thermiques de \textbf{NASA FIRMS}.

\subsubsection{Contexte et Justification}

Les forêts marocaines s'étendent sur près de \textbf{9 millions d'hectares}, une superficie impossible à couvrir intégralement par des caméras de surveillance. L'imagerie satellite offre une solution complémentaire permettant une surveillance à grande échelle. Les satellites Sentinel-2 fournissent des images multi-spectrales avec une résolution spatiale de 10 à 60 mètres et une revisite de 5 jours, tandis que les satellites MODIS et VIIRS de NASA FIRMS détectent les anomalies thermiques en quasi temps réel.

\begin{moroccobox}{Couverture Géographique du Maroc}
Le système AI Sentinel surveille \textbf{8 zones géographiques} couvrant l'ensemble du territoire marocain à risque d'incendie :

\vspace{0.3cm}
\begin{center}
\begin{tabular}{l l l}
\textbf{Zone} & \textbf{Coordonnées} & \textbf{Caractéristiques} \\
\hline
North (Tanger-Tétouan) & -6.00, 34.00 & Forêts méditerranéennes \\
Rif & -5.00, 34.50 & Montagnes boisées \\
Oriental & -3.00, 33.50 & Steppe et maquis \\
Casablanca & -8.00, 33.00 & Zone périurbaine \\
Middle Atlas & -6.00, 32.50 & Cèdres et chênes verts \\
Marrakech & -8.50, 31.00 & Arganeraie \\
High Atlas & -8.00, 30.50 & Forêts d'altitude \\
Souss & -10.00, 29.50 & Formations semi-arides \\
\end{tabular}
\end{center}
\end{moroccobox}

\subsubsection{Besoins Fonctionnels Détaillés}

\begin{objectifbox}{RF11 : Acquisition d'Images Sentinel-2}
\textbf{Description :} Le système doit pouvoir récupérer automatiquement les images satellites Sentinel-2 pour les zones définies.

\textbf{Spécifications détaillées :}
\begin{itemize}[leftmargin=1cm, itemsep=5pt]
    \item Authentification OAuth2 avec l'API Sentinel Hub
    \item Récupération d'images Sentinel-2 L2A (atmosphériquement corrigées)
    \item Deux types de scripts d'évaluation (evalscripts) :
    \begin{itemize}
        \item \textbf{True Color :} RGB standard pour visualisation
        \item \textbf{Fire Detection :} SWIR enhanced pour détection thermique
    \end{itemize}
    \item Résolution de 10m/pixel pour les bandes visibles
    \item Gestion intelligente de la couverture nuageuse (seuil < 30\%)
    \item Stockage local des images pour analyse ultérieure
\end{itemize}

\textbf{Critère d'acceptation :} Le système doit récupérer une image satellite en moins de 10 secondes (hors temps de traitement Sentinel Hub).
\end{objectifbox}

\vspace{0.3cm}

\begin{objectifbox}{RF12 : Analyse IA des Images Satellites}
\textbf{Description :} Les images satellites doivent être analysées par un modèle IA spécialisé (CAM - Class Activation Map) pour détecter les zones de feu potentiel.

\textbf{Spécifications détaillées :}
\begin{itemize}[leftmargin=1cm, itemsep=5pt]
    \item Modèle CAM personnalisé avec sortie duale :
    \begin{itemize}
        \item Carte d'activation (10×10×64) pour visualisation des zones critiques
        \item Classification binaire (Fire / No Fire)
    \end{itemize}
    \item Prétraitement : redimensionnement 224×224, normalisation 0-1
    \item Génération de heatmaps colorées superposées à l'image originale
    \item Score de confiance pour chaque analyse
    \item Détection des anomalies thermiques via les bandes SWIR
\end{itemize}

\textbf{Critère d'acceptation :} L'analyse d'une image satellite doit se compléter en moins de 5 secondes avec un taux de détection > 90\%.
\end{objectifbox}

\vspace{0.3cm}

\begin{objectifbox}{RF13 : Scan Automatique Programmé}
\textbf{Description :} Le système doit pouvoir effectuer des scans automatiques périodiques de toutes les zones surveillées.

\textbf{Spécifications détaillées :}
\begin{itemize}[leftmargin=1cm, itemsep=5pt]
    \item Scheduler configurable (intervalle par défaut : 30 minutes)
    \item Scan séquentiel de toutes les zones pour éviter la surcharge API
    \item Détection des changements par comparaison avec le scan précédent
    \item Génération automatique d'alertes en cas de détection positive
    \item Interface de contrôle : démarrer/arrêter le monitoring
    \item Logs détaillés de chaque cycle de scan
    \item Résilience aux erreurs (retry automatique)
\end{itemize}

\textbf{Critère d'acceptation :} Le scan de toutes les zones doit se terminer en moins de 10 minutes avec génération d'un rapport de synthèse.
\end{objectifbox}

\vspace{0.5cm}

\begin{center}
\begin{tikzpicture}[
    node distance=1.5cm,
    block/.style={rectangle, rounded corners=8pt, draw=accentTeal, line width=1.5pt, fill=skyBlue!15, text width=2.8cm, minimum height=1.8cm, align=center, font=\footnotesize},
    decision/.style={diamond, draw=alertOrange, line width=1.5pt, fill=sunYellow!30, text width=1.5cm, align=center, font=\footnotesize, aspect=2},
    arrow/.style={->, >=stealth, line width=1pt, color=primaryGreen}
]
    \node[block] (start) {\faClock\\Déclenchement\\Scheduler};
    \node[block, right=of start] (fetch) {\faSatellite\\Récupération\\Image Sentinel};
    \node[block, right=of fetch] (analyze) {\faBrain\\Analyse\\CAM Model};
    \node[decision, right=of analyze] (detect) {Feu\\?};
    \node[block, above right=1cm and 1.5cm of detect] (alert) {\faBell\\Envoi\\Alerte};
    \node[block, below right=1cm and 1.5cm of detect] (log) {\faDatabase\\Log\\Résultat};
    \node[block, right=3.5cm of detect] (next) {\faRedo\\Zone\\Suivante};
    
    \draw[arrow] (start) -- (fetch);
    \draw[arrow] (fetch) -- (analyze);
    \draw[arrow] (analyze) -- (detect);
    \draw[arrow] (detect) -- node[above, font=\tiny, text=moroccanRed] {Oui} (alert);
    \draw[arrow] (detect) -- node[below, font=\tiny, text=primaryGreen] {Non} (log);
    \draw[arrow] (alert) -| (next);
    \draw[arrow] (log) -| (next);
    
    \node[font=\bfseries\small, text=darkGreen] at (5, -2.5) {Workflow de Surveillance Satellite Automatisée};
\end{tikzpicture}
\end{center}

\subsection{Module de Prévention (Hotspots NASA FIRMS)}

Le module de prévention exploite les données de \textbf{NASA FIRMS} (Fire Information for Resource Management System) pour afficher en temps quasi réel les hotspots thermiques détectés par les satellites MODIS et VIIRS. Ces données permettent une vision globale des zones à risque et des feux actifs.

\subsubsection{Contexte et Justification}

NASA FIRMS fournit gratuitement des données de détection thermique avec une latence de seulement \textbf{3 heures}, permettant une surveillance quasi temps réel à l'échelle globale. Ces données sont essentielles pour :
\begin{itemize}[leftmargin=1cm, itemsep=3pt]
    \item Identifier les feux actifs dans des zones non couvertes par les caméras
    \item Cartographier les tendances et patterns d'incendies
    \item Anticiper les risques de propagation vers les zones habitées
    \item Compléter les données de terrain avec une vue satellite
\end{itemize}

\begin{infobox}{Données NASA FIRMS}
Les satellites MODIS (Moderate Resolution Imaging Spectroradiometer) et VIIRS (Visible Infrared Imaging Radiometer Suite) détectent les anomalies thermiques en mesurant la luminosité dans les bandes infrarouges thermiques. Un pixel est marqué comme hotspot lorsque sa température dépasse significativement celle des pixels environnants.

\textbf{Caractéristiques des données :}
\begin{itemize}[leftmargin=0.5cm, itemsep=3pt]
    \item \textbf{Résolution MODIS :} 1 km
    \item \textbf{Résolution VIIRS :} 375 m
    \item \textbf{Latence :} 3 heures après passage satellite
    \item \textbf{Couverture :} Globale, plusieurs passages par jour
\end{itemize}
\end{infobox}

\subsubsection{Besoins Fonctionnels Détaillés}

\begin{objectifbox}{RF14 : Récupération des Données NASA FIRMS}
\textbf{Description :} Le système doit récupérer automatiquement les données de hotspots depuis l'API NASA FIRMS pour les régions surveillées.

\textbf{Spécifications détaillées :}
\begin{itemize}[leftmargin=1cm, itemsep=5pt]
    \item Authentification via MAP\_KEY NASA FIRMS
    \item Récupération des données VIIRS\_SNPP pour les dernières 24 heures
    \item Filtrage par zone géographique (bounding box)
    \item Parsing des attributs : latitude, longitude, brightness, confidence, frp, daynight
    \item Mise en cache pour optimiser les requêtes répétées
    \item Gestion des erreurs et retry automatique
\end{itemize}

\textbf{Critère d'acceptation :} Les données doivent être récupérées et parsées en moins de 5 secondes par zone.
\end{objectifbox}

\vspace{0.3cm}

\begin{objectifbox}{RF15 : Affichage sur Carte Interactive}
\textbf{Description :} Les hotspots doivent être affichés sur une carte interactive permettant la navigation et l'exploration des données.

\textbf{Spécifications détaillées :}
\begin{itemize}[leftmargin=1cm, itemsep=5pt]
    \item Carte interactive basée sur Leaflet/React-Leaflet
    \item Marqueurs colorés selon le niveau de confiance :
    \begin{itemize}
        \item \textcolor{moroccanRed}{\textbf{Rouge}} : High confidence
        \item \textcolor{alertOrange}{\textbf{Orange}} : Nominal confidence
        \item \textcolor{sunYellow}{\textbf{Jaune}} : Low confidence
    \end{itemize}
    \item Popup d'information au clic sur un marqueur
    \item Zoom et pan fluides
    \item Clustering automatique des marqueurs à faible zoom
    \item Filtres par région, date, niveau de confiance
    \item Lien vers Google Maps pour navigation
\end{itemize}

\textbf{Critère d'acceptation :} La carte doit afficher jusqu'à 500 marqueurs sans dégradation notable des performances.
\end{objectifbox}

\vspace{0.3cm}

\begin{objectifbox}{RF16 : Calcul du Rayon de Propagation}
\textbf{Description :} Pour chaque hotspot, le système doit calculer une estimation du rayon de propagation probable basée sur les données disponibles.

\textbf{Spécifications détaillées :}
\begin{itemize}[leftmargin=1cm, itemsep=5pt]
    \item Algorithme de calcul intégrant plusieurs facteurs :
    \begin{itemize}
        \item \textbf{Luminosité (brightness\_ti4/ti5)} : indicateur d'intensité
        \item \textbf{Confiance NASA} : fiabilité de la détection
        \item \textbf{FRP (Fire Radiative Power)} : puissance du feu en MW
    \end{itemize}
    \item Affichage d'un cercle de propagation sur la carte
    \item Rayon borné entre 1 km et 15 km
    \item Code couleur du cercle selon le niveau de risque
    \item Tooltip avec estimation de la surface menacée
\end{itemize}

\textbf{Critère d'acceptation :} Le rayon de propagation doit être calculé et affiché en moins de 100ms par hotspot.
\end{objectifbox}

\subsection{Module de Prédiction de Propagation}

Le module de prédiction de propagation constitue l'élément d'anticipation du système AI Sentinel. Il permet d'estimer, à horizon de 6 heures, la zone potentiellement affectée par la propagation d'un incendie en tenant compte de multiples facteurs environnementaux.

\subsubsection{Contexte et Justification}

La prédiction de la propagation des feux de forêt est un problème complexe qui dépend de nombreuses variables : topographie, type de végétation, conditions météorologiques, humidité du sol, etc. Notre approche empirique, basée sur les données réelles de NASA FIRMS et les données météorologiques, offre une estimation utilisable opérationnellement tout en restant computationnellement légère.

\begin{alertbox}{Importance de la Prédiction}
Une prédiction, même approximative, de la propagation d'un feu permet aux services d'intervention de :
\begin{itemize}[leftmargin=0.5cm, itemsep=3pt]
    \item \textbf{Anticiper} les zones à évacuer
    \item \textbf{Positionner} les moyens de lutte de manière optimale
    \item \textbf{Protéger} les infrastructures critiques en priorité
    \item \textbf{Informer} les populations à risque
\end{itemize}
\end{alertbox}

\subsubsection{Besoins Fonctionnels Détaillés}

\begin{objectifbox}{RF17 : Saisie des Paramètres de Simulation}
\textbf{Description :} L'utilisateur doit pouvoir saisir ou modifier les paramètres de simulation de propagation.

\textbf{Spécifications détaillées :}
\begin{itemize}[leftmargin=1cm, itemsep=5pt]
    \item Interface de saisie intuitive avec valeurs par défaut
    \item Paramètres configurables :
    \begin{itemize}
        \item \textbf{Luminosité (Kelvin)} : 300K - 500K (défaut : 350K)
        \item \textbf{Confiance NASA} : high, nominal, low
        \item \textbf{Vitesse du vent} : 0 - 100 km/h
        \item \textbf{Direction du vent} : 0° - 360°
    \end{itemize}
    \item Sliders et sélecteurs pour faciliter la saisie
    \item Validation des valeurs en temps réel
    \item Pré-remplissage automatique depuis les données réelles
\end{itemize}

\textbf{Critère d'acceptation :} Tous les paramètres doivent être modifiables avec feedback visuel immédiat.
\end{objectifbox}

\vspace{0.3cm}

\begin{objectifbox}{RF18 : Calcul Prédictif de Propagation}
\textbf{Description :} Le système doit calculer le rayon de propagation estimé à 6 heures basé sur l'algorithme empirique.

\textbf{Algorithme de calcul :}

\begin{center}
\begin{tcolorbox}[
    enhanced,
    colback=white,
    colframe=accentTeal,
    boxrule=2pt,
    rounded corners,
    width=12cm
]
\begin{align*}
\text{Rayon} &= \text{Base} + f(\text{Brightness}) + f(\text{Confidence}) + f(\text{Wind})
\end{align*}

Où :
\begin{itemize}[leftmargin=0.5cm]
    \item $\text{Base} = 3.0$ km (rayon de base)
    \item $f(\text{Brightness})$ :
    \begin{itemize}
        \item $> 350K \rightarrow +3.0$ km
        \item $> 320K \rightarrow +1.5$ km
        \item sinon $\rightarrow +0.0$ km
    \end{itemize}
    \item $f(\text{Confidence})$ :
    \begin{itemize}
        \item high $\rightarrow +1.0$ km
        \item low $\rightarrow -1.0$ km
    \end{itemize}
    \item $f(\text{Wind}) = (\text{vitesse} / 30) \times 2.0$ km
\end{itemize}

Résultat final : $\max(1.0, \min(15.0, \text{Rayon}))$ km
\end{tcolorbox}
\end{center}

\textbf{Critère d'acceptation :} Le calcul doit s'exécuter en moins de 50ms avec une précision de 2 décimales.
\end{objectifbox}

\vspace{0.3cm}

\begin{objectifbox}{RF19 : Visualisation de la Zone de Danger}
\textbf{Description :} La zone de danger prédite doit être visualisée sur une carte interactive avec des indicateurs clairs.

\textbf{Spécifications détaillées :}
\begin{itemize}[leftmargin=1cm, itemsep=5pt]
    \item Cercle de propagation centré sur le hotspot
    \item Dégradé de couleur du centre (rouge intense) vers l'extérieur (jaune)
    \item Flèche indiquant la direction du vent
    \item Affichage des coordonnées du centre
    \item Estimation de la surface en km²
    \item Animation optionnelle de l'expansion
    \item Légende explicative des zones de risque
\end{itemize}

\textbf{Critère d'acceptation :} La visualisation doit se mettre à jour en temps réel lors de la modification des paramètres.
\end{objectifbox}

\vspace{0.5cm}

\begin{table}[H]
\centering
\caption{Exemples de calcul de propagation}
\label{tab:propagation}
\rowcolors{2}{mintGreen!30}{white}
\begin{tabular}{c c c c c}
\toprule
\rowcolor{primaryGreen}
\textcolor{white}{\textbf{Brightness}} & \textcolor{white}{\textbf{Confiance}} & \textcolor{white}{\textbf{Vent (km/h)}} & \textcolor{white}{\textbf{Rayon (km)}} & \textcolor{white}{\textbf{Surface (km²)}} \\
\midrule
320 K & Nominal & 20 & 5.83 & 106.9 \\
355 K & High & 40 & 9.67 & 293.6 \\
380 K & High & 60 & 11.00 & 380.1 \\
310 K & Low & 10 & 2.67 & 22.4 \\
400 K & High & 80 & 12.33 & 477.7 \\
\bottomrule
\end{tabular}
\end{table}

\subsection{Module de Notifications}

Le module de notifications assure la diffusion rapide et fiable des alertes vers les parties prenantes via différents canaux de communication. Cette composante est critique pour garantir une réponse rapide aux détections.

\subsubsection{Contexte et Justification}

La détection n'a de valeur que si elle est suivie d'une action. Le module de notifications transforme une détection technique en information actionnable pour les équipes terrain. L'utilisation de canaux multiples (\textbf{Email} et \textbf{Telegram}) garantit que l'alerte atteindra ses destinataires même en cas de défaillance d'un canal.

\begin{greenbox}[\faBell\ Philosophie du Système d'Alertes]
Notre système d'alertes repose sur trois principes fondamentaux :

\begin{enumerate}[leftmargin=1cm, itemsep=5pt]
    \item \textbf{Rapidité :} L'alerte doit être envoyée dans les secondes suivant la détection
    \item \textbf{Fiabilité :} Utilisation de canaux redondants (Email + Telegram)
    \item \textbf{Non-intrusivité :} Mécanisme anti-spam pour éviter la fatigue d'alerte
\end{enumerate}
\end{greenbox}

\subsubsection{Besoins Fonctionnels Détaillés}

\begin{objectifbox}{RF20 : Envoi d'Email avec Détails}
\textbf{Description :} Le système doit envoyer des alertes par email contenant toutes les informations pertinentes sur la détection.

\textbf{Spécifications détaillées :}
\begin{itemize}[leftmargin=1cm, itemsep=5pt]
    \item Envoi via SMTP (Gmail ou serveur configuré)
    \item Contenu HTML richement formaté :
    \begin{itemize}
        \item Titre d'alerte avec niveau de sévérité
        \item Zone géographique concernée
        \item Coordonnées GPS du point de détection
        \item Image satellite ou capture d'écran intégrée
        \item Prédiction de propagation
        \item Lien Google Maps pour localisation rapide
        \item Horodatage précis
    \end{itemize}
    \item Gestion des pièces jointes (images)
    \item Configuration des destinataires multiples
    \item Retry automatique en cas d'échec (3 tentatives)
\end{itemize}

\textbf{Critère d'acceptation :} L'email doit être envoyé dans les 10 secondes suivant la détection.
\end{objectifbox}

\vspace{0.3cm}

\begin{objectifbox}{RF21 : Envoi de Message Telegram}
\textbf{Description :} Le système doit envoyer des alertes instantanées via un bot Telegram.

\textbf{Spécifications détaillées :}
\begin{itemize}[leftmargin=1cm, itemsep=5pt]
    \item Intégration avec l'API Telegram Bot
    \item Message formaté avec emojis pour lisibilité rapide :
    \begin{itemize}
        \item 🔥 \textbf{ALERTE FEU DÉTECTÉ}
        \item 📍 Zone : [nom de la région]
        \item 🌡️ Confiance : [score]\%
        \item 📐 Propagation estimée : [rayon] km
        \item 🗺️ [Lien Google Maps]
    \end{itemize}
    \item Envoi d'image satellite avec le message
    \item Boutons inline pour actions rapides
    \item Notification push instantanée sur mobile
    \item Support des groupes et channels
\end{itemize}

\textbf{Critère d'acceptation :} Le message Telegram doit être reçu dans les 5 secondes suivant la détection.
\end{objectifbox}

\vspace{0.3cm}

\begin{objectifbox}{RF22 : Gestion du Cooldown Anti-Spam}
\textbf{Description :} Le système doit implémenter un mécanisme de cooldown pour éviter l'envoi d'alertes répétées pour la même détection.

\textbf{Spécifications détaillées :}
\begin{itemize}[leftmargin=1cm, itemsep=5pt]
    \item Cooldown par défaut : 30 secondes
    \item Cooldown configurable par zone (5s - 300s)
    \item Tracking par zone géographique + type de détection
    \item Reset du cooldown lors d'un changement significatif
    \item Logs des alertes ignorées pour audit
    \item Interface de monitoring des cooldowns actifs
\end{itemize}

\textbf{Critère d'acceptation :} Aucune alerte dupliquée ne doit être envoyée pendant la période de cooldown.
\end{objectifbox}

\subsection{Module FWI (Fire Weather Index)}

Le module FWI intègre l'indice météorologique de risque d'incendie, un standard international développé par le Service canadien des forêts. Cet indice combine plusieurs paramètres météorologiques pour évaluer le potentiel de départ et de propagation des feux.

\subsubsection{Besoins Fonctionnels Détaillés}

\begin{objectifbox}{RF23 : Affichage des Indices Météo}
\textbf{Description :} Le système doit afficher les indices météorologiques pertinents pour l'évaluation du risque d'incendie.

\textbf{Spécifications détaillées :}
\begin{itemize}[leftmargin=1cm, itemsep=5pt]
    \item Récupération des données via Open-Meteo API
    \item Paramètres affichés :
    \begin{itemize}
        \item Température (°C)
        \item Humidité relative (\%)
        \item Vitesse et direction du vent
        \item Précipitations (mm)
        \item Indice FWI calculé
    \end{itemize}
    \item Mise à jour automatique toutes les heures
    \item Historique sur 24h avec graphiques
    \item Prévisions à 48h
\end{itemize}

\textbf{Critère d'acceptation :} Les données météo doivent être affichées avec une latence maximale de 2 secondes.
\end{objectifbox}

\vspace{0.3cm}

\begin{objectifbox}{RF24 : Carte de Risque}
\textbf{Description :} Le système doit afficher une carte de risque d'incendie basée sur les indices FWI.

\textbf{Spécifications détaillées :}
\begin{itemize}[leftmargin=1cm, itemsep=5pt]
    \item Carte choroplèthe avec zones colorées par niveau de risque
    \item Échelle de risque standardisée :
    \begin{itemize}
        \item \textcolor{primaryGreen}{\textbf{Vert}} : Faible (FWI 0-5)
        \item \textcolor{sunYellow}{\textbf{Jaune}} : Modéré (FWI 5-10)
        \item \textcolor{alertOrange}{\textbf{Orange}} : Élevé (FWI 10-20)
        \item \textcolor{moroccanRed}{\textbf{Rouge}} : Très élevé (FWI 20-30)
        \item \textbf{Pourpre} : Extrême (FWI > 30)
    \end{itemize}
    \item Superposition optionnelle avec les hotspots actifs
    \item Export de la carte en image
    \item Légende interactive
\end{itemize}

\textbf{Critère d'acceptation :} La carte de risque doit se charger en moins de 3 secondes.
\end{objectifbox}

% ============================================
% CHAPITRE III - PARTIE 3 : BESOINS NON FONCTIONNELS ET MÉTHODOLOGIE
% ============================================

\newpage
\section{Analyse des Besoins Non Fonctionnels}

Les besoins non fonctionnels définissent \textbf{comment} le système doit fonctionner, en termes de qualité, de performance et de contraintes techniques. Ces exigences sont tout aussi importantes que les besoins fonctionnels car elles déterminent l'acceptabilité du système par les utilisateurs et sa viabilité à long terme.

Dans le contexte critique de la détection d'incendies, où chaque seconde compte, les exigences de performance, de fiabilité et de disponibilité revêtent une importance capitale.

\subsection{Performance}

La performance du système AI Sentinel est un facteur clé de succès. Un système de détection d'incendies doit être capable de traiter les données rapidement pour permettre une intervention précoce.

\begin{techbox}{Exigences de Performance}

\textbf{NF01 : Temps de Réponse API}

\begin{itemize}[leftmargin=1cm, itemsep=5pt]
    \item Temps de réponse moyen : < \textbf{2 secondes}
    \item Temps de réponse au 95ème percentile : < \textbf{5 secondes}
    \item Endpoints critiques (détection) : < \textbf{500 ms}
\end{itemize}

\vspace{0.3cm}

\textbf{NF02 : Traitement Vidéo en Temps Réel}

\begin{itemize}[leftmargin=1cm, itemsep=5pt]
    \item Fréquence de traitement : $\geq$ \textbf{25 FPS} sur GPU
    \item Fréquence minimale acceptable : $\geq$ \textbf{15 FPS} sur CPU
    \item Latence bout en bout : < \textbf{500 ms}
    \item Résolution supportée : jusqu'à \textbf{1080p}
\end{itemize}

\vspace{0.3cm}

\textbf{NF03 : Capacité de Traitement}

\begin{itemize}[leftmargin=1cm, itemsep=5pt]
    \item Support de \textbf{10 utilisateurs} simultanés minimum
    \item Traitement de \textbf{100 images/heure} pour classification
    \item Scan satellite de \textbf{8 zones} en moins de 10 minutes
\end{itemize}
\end{techbox}

\vspace{0.5cm}

\begin{center}
\begin{tikzpicture}
    % Graphique de performance
    \begin{axis}[
        width=12cm,
        height=6cm,
        xlabel={Composant},
        ylabel={Temps de réponse (ms)},
        ymin=0, ymax=3000,
        symbolic x coords={YOLOv8, MobileNet, CAM, API REST, Satellite, Notification},
        xtick=data,
        xticklabel style={rotate=45, anchor=east, font=\footnotesize},
        ylabel style={font=\small},
        xlabel style={font=\small},
        bar width=15pt,
        nodes near coords,
        nodes near coords style={font=\tiny},
        ybar,
        enlarge x limits=0.15,
    ]
    \addplot[fill=primaryGreen!70, draw=primaryGreen] coordinates {
        (YOLOv8, 33)
        (MobileNet, 150)
        (CAM, 200)
        (API REST, 100)
        (Satellite, 2000)
        (Notification, 500)
    };
    \end{axis}
    \node[font=\bfseries\small, text=darkGreen] at (6, -1) {Temps de Réponse Cibles par Composant};
\end{tikzpicture}
\end{center}

\subsection{Fiabilité}

La fiabilité du système est essentielle dans un contexte où des vies et des biens sont en jeu. Le système doit fonctionner de manière continue et prévisible.

\begin{greenbox}[\faShieldAlt\ Exigences de Fiabilité]

\textbf{NF04 : Disponibilité}

Le système doit garantir une disponibilité de \textbf{99\%} minimum, soit moins de \textbf{87.6 heures} d'indisponibilité par an. Cette disponibilité couvre :
\begin{itemize}[leftmargin=0.5cm, itemsep=3pt]
    \item Le backend API (FastAPI)
    \item Le frontend web (React)
    \item Les services de notification (Telegram, Email)
\end{itemize}

\vspace{0.3cm}

\textbf{NF05 : Tolérance aux Pannes}

\begin{itemize}[leftmargin=0.5cm, itemsep=5pt]
    \item \textbf{Dégradation gracieuse :} En cas d'indisponibilité d'un service externe (NASA, Sentinel), le reste du système continue de fonctionner
    \item \textbf{Retry automatique :} Les appels échoués sont automatiquement relancés (3 tentatives avec backoff exponentiel)
    \item \textbf{Circuit breaker :} Protection contre les cascades de pannes
    \item \textbf{Logs complets :} Enregistrement de toutes les erreurs pour diagnostic
\end{itemize}

\vspace{0.3cm}

\textbf{NF06 : Précision des Modèles IA}

\begin{table}[H]
\centering
\rowcolors{2}{mintGreen!30}{white}
\begin{tabular}{l c c}
\toprule
\rowcolor{primaryGreen}
\textcolor{white}{\textbf{Modèle}} & \textcolor{white}{\textbf{Précision Cible}} & \textcolor{white}{\textbf{Faux Positifs}} \\
\midrule
YOLOv8 (Détection) & $\geq$ 85\% & < 5\% \\
MobileNetV2 (Classification) & $\geq$ 97\% & < 3\% \\
CAM (Satellite) & $\geq$ 90\% & < 10\% \\
\bottomrule
\end{tabular}
\end{table}
\end{greenbox}

\subsection{Sécurité}

Bien que le système AI Sentinel soit principalement orienté vers la surveillance environnementale, les aspects de sécurité informatique restent importants pour protéger les données et garantir l'intégrité du système.

\begin{alertbox}{Exigences de Sécurité}

\textbf{NF07 : Authentification API}

\begin{itemize}[leftmargin=0.5cm, itemsep=5pt]
    \item Protection des endpoints API par clés d'authentification
    \item Rotation régulière des clés API externes (NASA, Sentinel Hub)
    \item Stockage sécurisé des credentials dans des variables d'environnement
    \item Rate limiting pour prévenir les abus
\end{itemize}

\vspace{0.3cm}

\textbf{NF08 : Protection des Données}

\begin{itemize}[leftmargin=0.5cm, itemsep=5pt]
    \item Chiffrement HTTPS pour toutes les communications
    \item Pas de stockage de données personnelles sensibles
    \item Anonymisation des logs si nécessaire
    \item Conformité RGPD pour les données utilisateur
\end{itemize}

\vspace{0.3cm}

\textbf{NF09 : Intégrité du Système}

\begin{itemize}[leftmargin=0.5cm, itemsep=5pt]
    \item Validation de toutes les entrées utilisateur (images, paramètres)
    \item Protection contre les injections (SQL, command injection)
    \item Limitation de la taille des fichiers uploadés
    \item Sandboxing du traitement des fichiers
\end{itemize}
\end{alertbox}

\subsection{Maintenabilité}

La maintenabilité garantit que le système peut évoluer et être corrigé efficacement au fil du temps.

\begin{infobox}{Exigences de Maintenabilité}

\textbf{NF10 : Code Modulaire}

\begin{itemize}[leftmargin=0.5cm, itemsep=5pt]
    \item Architecture en microservices logiques (YoloService, FirmsService, PredictionService, etc.)
    \item Séparation claire des responsabilités (Single Responsibility Principle)
    \item Injection de dépendances pour faciliter les tests
    \item Interfaces bien définies entre les modules
\end{itemize}

\vspace{0.3cm}

\textbf{NF11 : Documentation}

\begin{itemize}[leftmargin=0.5cm, itemsep=5pt]
    \item Documentation API Swagger/OpenAPI automatique
    \item Commentaires de code pour les sections complexes
    \item README avec instructions d'installation et d'utilisation
    \item Documentation technique d'architecture
\end{itemize}

\vspace{0.3cm}

\textbf{NF12 : Testabilité}

\begin{itemize}[leftmargin=0.5cm, itemsep=5pt]
    \item Tests unitaires pour les fonctions critiques
    \item Tests d'intégration pour les flux principaux
    \item Couverture de code cible : > 70\%
    \item Environnement de test isolé
\end{itemize}
\end{infobox}

\subsection{Ergonomie}

L'ergonomie et l'expérience utilisateur sont essentielles pour garantir l'adoption du système par les opérateurs terrain qui doivent pouvoir réagir rapidement.

\begin{greenbox}[\faDesktop\ Exigences d'Ergonomie]

\textbf{NF13 : Interface Intuitive}

\begin{itemize}[leftmargin=0.5cm, itemsep=5pt]
    \item Navigation claire avec menu principal accessible
    \item Actions principales accessibles en \textbf{moins de 3 clics}
    \item Feedback visuel immédiat pour toutes les actions
    \item Messages d'erreur explicites et actionables
    \item Thème visuel cohérent (palette verte nature)
\end{itemize}

\vspace{0.3cm}

\textbf{NF14 : Design Responsive}

\begin{itemize}[leftmargin=0.5cm, itemsep=5pt]
    \item Compatibilité : Desktop (1920×1080), Tablette (768px), Mobile (375px)
    \item Adaptation automatique de la mise en page
    \item Carte interactive utilisable sur écran tactile
    \item Performances optimisées pour les appareils mobiles
\end{itemize}

\vspace{0.3cm}

\textbf{NF15 : Accessibilité}

\begin{itemize}[leftmargin=0.5cm, itemsep=5pt]
    \item Contraste suffisant pour lisibilité
    \item Labels pour tous les éléments de formulaire
    \item Navigation au clavier possible
    \item Compatibilité avec les lecteurs d'écran (ARIA)
\end{itemize}
\end{greenbox}

\vspace{0.5cm}

% Tableau récapitulatif des besoins non fonctionnels
\begin{table}[H]
\centering
\caption{Récapitulatif des Besoins Non Fonctionnels}
\label{tab:bnf}
\rowcolors{2}{mintGreen!30}{white}
\begin{tabular}{l l p{6cm} l}
\toprule
\rowcolor{primaryGreen}
\textcolor{white}{\textbf{ID}} & \textcolor{white}{\textbf{Catégorie}} & \textcolor{white}{\textbf{Description}} & \textcolor{white}{\textbf{Priorité}} \\
\midrule
NF01 & Performance & Temps de réponse API < 2s & Critique \\
NF02 & Performance & Traitement vidéo $\geq$ 25 FPS & Critique \\
NF03 & Performance & 10 utilisateurs simultanés & Haute \\
NF04 & Fiabilité & Disponibilité 99\% & Critique \\
NF05 & Fiabilité & Tolérance aux pannes & Haute \\
NF06 & Fiabilité & Précision modèles $\geq$ 85\% & Critique \\
NF07 & Sécurité & Authentification API & Haute \\
NF08 & Sécurité & Protection des données & Haute \\
NF09 & Sécurité & Validation des entrées & Haute \\
NF10 & Maintenabilité & Code modulaire & Moyenne \\
NF11 & Maintenabilité & Documentation complète & Moyenne \\
NF12 & Maintenabilité & Couverture tests > 70\% & Moyenne \\
NF13 & Ergonomie & Interface intuitive & Haute \\
NF14 & Ergonomie & Design responsive & Haute \\
NF15 & Ergonomie & Accessibilité & Moyenne \\
\bottomrule
\end{tabular}
\end{table}

% ============================================
\newpage
\section{Méthodologie de Développement}
% ============================================

Le choix d'une méthodologie de développement adaptée est crucial pour la réussite d'un projet logiciel. Dans le cadre d'AI Sentinel, nous avons opté pour une approche \textbf{Agile} avec le framework \textbf{Scrum}, particulièrement adapté aux projets innovants nécessitant flexibilité et itérations rapides.

\subsection{Choix de la Méthodologie : Agile/Scrum}

\subsubsection{Justification du Choix}

Le développement d'un système de détection d'incendies par IA présente plusieurs caractéristiques qui oriententnaturellement vers une méthodologie agile :

\begin{greenbox}[\faList\ Pourquoi Agile/Scrum ?]

\textbf{Complexité Technique}

Le projet intègre des technologies diverses (IA, satellite, temps réel) dont l'intégration peut révéler des défis imprévus. Scrum permet d'adapter le plan au fil des découvertes.

\vspace{0.3cm}

\textbf{Besoins Évolutifs}

Les exigences des utilisateurs (autorités, gestionnaires forestiers) peuvent évoluer à mesure qu'ils découvrent les possibilités du système. Les sprints courts permettent d'intégrer leurs retours.

\vspace{0.3cm}

\textbf{Livraisons Incrémentales}

Chaque module (détection temps réel, satellite, prédiction) peut être développé et livré indépendamment, permettant des tests terrain précoces.

\vspace{0.3cm}

\textbf{Gestion des Risques}

Les revues de sprint régulières permettent d'identifier et de traiter rapidement les problèmes techniques ou fonctionnels.
\end{greenbox}

\subsubsection{Principes Scrum Appliqués}

\begin{center}
\begin{tikzpicture}[
    role/.style={ellipse, draw=primaryGreen, line width=2pt, fill=mintGreen, text width=2cm, align=center, font=\small, minimum height=1.5cm},
    artifact/.style={rectangle, rounded corners=8pt, draw=accentTeal, line width=1.5pt, fill=skyBlue!20, text width=2.5cm, align=center, font=\small, minimum height=1.5cm},
    event/.style={rectangle, rounded corners=8pt, draw=leafGreen, line width=1.5pt, fill=leafGreen!20, text width=2cm, align=center, font=\small, minimum height=1.2cm}
]
    % Titre
    \node[font=\Large\bfseries, text=darkGreen] at (6, 4) {Framework Scrum appliqué à AI Sentinel};
    
    % Rôles
    \node[role] (po) at (0, 2) {Product\\Owner};
    \node[role] (sm) at (0, 0) {Scrum\\Master};
    \node[role] (team) at (0, -2) {Dev\\Team};
    
    % Artefacts
    \node[artifact] (pb) at (4, 2) {Product\\Backlog};
    \node[artifact] (sb) at (4, 0) {Sprint\\Backlog};
    \node[artifact] (inc) at (4, -2) {Incrément\\Produit};
    
    % Événements
    \node[event] (sprint) at (8, 2) {Sprint\\(2 sem.)};
    \node[event] (plan) at (8, 0.5) {Sprint\\Planning};
    \node[event] (daily) at (8, -0.8) {Daily\\Standup};
    \node[event] (review) at (8, -2) {Sprint\\Review};
    
    % Connexions
    \draw[->, >=stealth, primaryGreen, line width=1pt] (po) -- (pb);
    \draw[->, >=stealth, primaryGreen, line width=1pt] (pb) -- (sb);
    \draw[->, >=stealth, primaryGreen, line width=1pt] (sb) -- (inc);
    \draw[->, >=stealth, leafGreen, line width=1pt] (sb) -- (sprint);
    
\end{tikzpicture}
\end{center}

\subsection{Planning des Sprints}

Le projet AI Sentinel a été planifié sur \textbf{6 sprints} de 2 semaines chacun, soit une durée totale de 12 semaines.

\begin{table}[H]
\centering
\caption{Planning des Sprints AI Sentinel}
\label{tab:sprints}
\rowcolors{2}{mintGreen!30}{white}
\begin{tabular}{c p{4cm} p{5cm} c}
\toprule
\rowcolor{primaryGreen}
\textcolor{white}{\textbf{Sprint}} & \textcolor{white}{\textbf{Objectif}} & \textcolor{white}{\textbf{Livrables}} & \textcolor{white}{\textbf{Durée}} \\
\midrule
1 & Setup \& Architecture & Environnement dev, structure projet, API base & 2 sem. \\
2 & Module Détection & YOLOv8 intégré, détection temps réel & 2 sem. \\
3 & Module Classification & MobileNetV2, upload images & 2 sem. \\
4 & Module Satellite & Sentinel Hub, NASA FIRMS & 2 sem. \\
5 & Module Prédiction & Algorithme propagation, carte & 2 sem. \\
6 & Notifications \& Polish & Email, Telegram, UI final & 2 sem. \\
\bottomrule
\end{tabular}
\end{table}

\vspace{0.3cm}

\begin{center}
\begin{tikzpicture}
    % Timeline
    \draw[line width=3pt, primaryGreen] (0,0) -- (14,0);
    
    % Sprints
    \foreach \x/\s/\label in {0/1/Setup, 2.33/2/Détection, 4.66/3/Classification, 7/4/Satellite, 9.33/5/Prédiction, 11.66/6/Notif.} {
        \fill[primaryGreen] (\x, 0) circle (6pt);
        \node[above, font=\footnotesize\bfseries] at (\x, 0.3) {S\s};
        \node[below, font=\tiny, text width=1.5cm, align=center] at (\x, -0.3) {\label};
    }
    \fill[leafGreen] (14, 0) circle (8pt);
    \node[above, font=\footnotesize\bfseries] at (14, 0.3) {Release};
    
    % Durées
    \node[font=\tiny] at (1.16, -1) {2 sem.};
    \node[font=\tiny] at (3.5, -1) {2 sem.};
    \node[font=\tiny] at (5.83, -1) {2 sem.};
    \node[font=\tiny] at (8.16, -1) {2 sem.};
    \node[font=\tiny] at (10.5, -1) {2 sem.};
    \node[font=\tiny] at (12.83, -1) {2 sem.};
\end{tikzpicture}
\end{center}

\subsection{Outils de Gestion}

La gestion efficace du projet repose sur un ensemble d'outils complémentaires permettant le suivi des tâches, la gestion du code source et la collaboration d'équipe.

\begin{techbox}{Stack d'Outils de Gestion de Projet}

\textbf{\faTrello\ Trello --- Gestion des Tâches}

\begin{itemize}[leftmargin=0.5cm, itemsep=3pt]
    \item Tableau Kanban avec colonnes : Backlog, To Do, In Progress, Review, Done
    \item Cartes pour chaque User Story avec checklist de sous-tâches
    \item Labels par module (Détection, Satellite, UI, etc.)
    \item Power-ups pour estimation en points et burndown
\end{itemize}

\vspace{0.3cm}

\textbf{\faGitAlt\ Git/GitHub --- Gestion du Code}

\begin{itemize}[leftmargin=0.5cm, itemsep=3pt]
    \item Repository centralisé avec branches par feature
    \item Convention de nommage : \texttt{feature/module-description}
    \item Pull Requests obligatoires pour merge sur \texttt{main}
    \item Code review par au moins 1 développeur
    \item GitHub Actions pour CI/CD automatisé
\end{itemize}

\vspace{0.3cm}

\textbf{\faSlack\ Communication}

\begin{itemize}[leftmargin=0.5cm, itemsep=3pt]
    \item Channels dédiés par sujet (dev, bugs, releases)
    \item Intégrations avec GitHub et Trello
    \item Daily standups asynchrones
\end{itemize}

\vspace{0.3cm}

\textbf{\faFileAlt\ Documentation}

\begin{itemize}[leftmargin=0.5cm, itemsep=3pt]
    \item Confluence/Notion pour documentation technique
    \item Swagger pour documentation API
    \item README.md dans chaque module
\end{itemize}
\end{techbox}

\vspace{0.5cm}

% Workflow Git
\begin{center}
\begin{tikzpicture}[
    branch/.style={rectangle, rounded corners=5pt, draw=accentTeal, line width=1.5pt, fill=skyBlue!20, text width=2cm, minimum height=1cm, align=center, font=\small},
    commit/.style={circle, draw=primaryGreen, line width=1.5pt, fill=mintGreen, minimum size=0.6cm},
    arrow/.style={->, >=stealth, line width=1.5pt}
]
    % Branches
    \node[font=\bfseries, text=darkGreen] at (0, 2) {main};
    \node[font=\bfseries, text=accentTeal] at (0, 0.5) {develop};
    \node[font=\bfseries, text=leafGreen] at (0, -1) {feature/x};
    
    % Commits main
    \foreach \x in {2, 6, 10, 14} {
        \node[commit] at (\x, 2) {};
    }
    \draw[darkGreen, line width=2pt] (1, 2) -- (15, 2);
    
    % Commits develop
    \foreach \x in {2, 4, 6, 8, 10, 12, 14} {
        \node[commit, fill=skyBlue!50] at (\x, 0.5) {};
    }
    \draw[accentTeal, line width=2pt] (1, 0.5) -- (15, 0.5);
    
    % Feature branch
    \foreach \x in {4, 5, 6} {
        \node[commit, fill=leafGreen!50] at (\x, -1) {};
    }
    \draw[leafGreen, line width=1.5pt] (3, 0.5) -- (4, -1) -- (6, -1) -- (7, 0.5);
    
    % Merge arrows
    \draw[arrow, accentTeal] (6, 0.5) -- (6, 1.8);
    \draw[arrow, accentTeal] (14, 0.5) -- (14, 1.8);
    
    \node[font=\bfseries\small, text=darkGreen] at (8, -2.5) {Workflow Git Flow};
\end{tikzpicture}
\end{center}

% ============================================
% CHAPITRE III - PARTIE 4 : MODÉLISATION UML
% ============================================

\newpage
\section{Modélisation UML}

La modélisation UML (Unified Modeling Language) permet de visualiser l'architecture et le comportement du système de manière standardisée. Cette section présente les différents diagrammes qui décrivent AI Sentinel sous ses multiples facettes : fonctionnelle, structurelle et comportementale.

\subsection{Diagramme de Cas d'Utilisation Global}

Le diagramme de cas d'utilisation global offre une vue d'ensemble des fonctionnalités du système et des interactions entre les acteurs et le système.

\vspace{0.5cm}
\begin{center}
\begin{tikzpicture}[
    actor/.style={font=\small},
    usecase/.style={ellipse, draw=primaryGreen, line width=1.5pt, fill=mintGreen!50, text width=2.5cm, align=center, font=\footnotesize, minimum height=1cm},
    system/.style={rectangle, draw=darkGreen, line width=2pt, rounded corners=10pt, minimum width=11cm, minimum height=10cm}
]
    % Système
    \node[system, label={[font=\Large\bfseries, text=darkGreen]above:Système AI Sentinel}] (sys) at (5, 0) {};
    
    % Acteurs
    \node[actor] (user) at (-2, 1) {\includegraphics[width=1cm]{example-image}};
    \node[font=\small\bfseries, below] at (-2, 0.3) {Utilisateur};
    
    \node[actor] (admin) at (12, 1) {\includegraphics[width=1cm]{example-image}};
    \node[font=\small\bfseries, below] at (12, 0.3) {Admin};
    
    \node[actor] (ext) at (5, -6.5) {};
    \node[font=\small\bfseries, text=accentTeal] at (5, -6.5) {\faCloud\ Services Externes};
    
    % Cas d'utilisation - Colonne gauche
    \node[usecase] (uc1) at (2, 3) {Détection\\temps réel};
    \node[usecase] (uc2) at (2, 1.5) {Upload\\média};
    \node[usecase] (uc3) at (2, 0) {Surveillance\\satellite};
    \node[usecase] (uc4) at (2, -1.5) {Visualiser\\carte};
    
    % Cas d'utilisation - Colonne droite
    \node[usecase] (uc5) at (8, 3) {Prédiction\\propagation};
    \node[usecase] (uc6) at (8, 1.5) {Configurer\\alertes};
    \node[usecase] (uc7) at (8, 0) {Consulter\\historique};
    \node[usecase] (uc8) at (8, -1.5) {Exporter\\rapports};
    
    % Cas Admin
    \node[usecase, fill=skyBlue!30] (uca1) at (5, -3) {Gérer\\monitoring};
    \node[usecase, fill=skyBlue!30] (uca2) at (5, -4.5) {Configurer\\système};
    
    % Connexions Utilisateur
    \draw[primaryGreen, line width=1pt] (user) -- (uc1);
    \draw[primaryGreen, line width=1pt] (user) -- (uc2);
    \draw[primaryGreen, line width=1pt] (user) -- (uc3);
    \draw[primaryGreen, line width=1pt] (user) -- (uc4);
    \draw[primaryGreen, line width=1pt] (-0.5, 1) -- (uc5);
    \draw[primaryGreen, line width=1pt] (-0.5, 1) -- (uc7);
    
    % Connexions Admin
    \draw[accentTeal, line width=1pt] (admin) -- (uc6);
    \draw[accentTeal, line width=1pt] (admin) -- (uc8);
    \draw[accentTeal, line width=1pt] (admin) -- (uca1);
    \draw[accentTeal, line width=1pt] (admin) -- (uca2);
    
    % Connexions Services Externes
    \draw[textGray, line width=1pt, dashed] (uc3) -- (ext);
    \draw[textGray, line width=1pt, dashed] (uc5) -- (ext);
    \draw[textGray, line width=1pt, dashed] (uca1) -- (ext);
    
\end{tikzpicture}
\end{center}

\vspace{0.5cm}

\begin{greenbox}[\faListOl\ Légende des Cas d'Utilisation]
\begin{minipage}[t]{0.48\textwidth}
\textbf{Cas Utilisateur :}
\begin{itemize}[leftmargin=0.5cm, itemsep=3pt]
    \item Détection temps réel (vidéo/webcam)
    \item Upload média (images/vidéos)
    \item Surveillance satellite
    \item Visualiser carte interactive
    \item Prédiction de propagation
    \item Consulter historique
\end{itemize}
\end{minipage}
\hfill
\begin{minipage}[t]{0.48\textwidth}
\textbf{Cas Administrateur :}
\begin{itemize}[leftmargin=0.5cm, itemsep=3pt]
    \item Configurer alertes
    \item Exporter rapports
    \item Gérer monitoring satellite
    \item Configurer système
\end{itemize}
\end{minipage}
\end{greenbox}

\subsection{Diagrammes de Cas d'Utilisation Détaillés}

\subsubsection{UC01 : Détection en Temps Réel}

\begin{center}
\begin{tikzpicture}[
    usecase/.style={ellipse, draw=primaryGreen, line width=1.5pt, fill=mintGreen!50, text width=2.2cm, align=center, font=\footnotesize, minimum height=0.9cm},
    extends/.style={->, >=stealth, dashed, line width=1pt, textGray},
    includes/.style={->, >=stealth, dashed, line width=1pt, accentTeal}
]
    % Acteur
    \node (user) at (-3, 0) {\faUser};
    \node[font=\small\bfseries, below] at (-3, -0.5) {Utilisateur};
    
    % Cas principal
    \node[usecase, fill=leafGreen!30, text width=3cm] (main) at (3, 0) {Détection\\Temps Réel};
    
    % Sous-cas
    \node[usecase] (uc1) at (7, 2) {Activer\\webcam};
    \node[usecase] (uc2) at (7, 0) {Visualiser\\flux};
    \node[usecase] (uc3) at (7, -2) {Afficher\\détections};
    \node[usecase] (uc4) at (11, 1) {Envoyer\\alerte};
    \node[usecase] (uc5) at (11, -1) {Sauvegarder\\capture};
    
    % Connexions
    \draw[primaryGreen, line width=1.5pt] (user) -- (main);
    \draw[includes] (main) -- node[above, font=\tiny] {<<include>>} (uc1);
    \draw[includes] (main) -- node[above, font=\tiny] {<<include>>} (uc2);
    \draw[includes] (main) -- node[below, font=\tiny] {<<include>>} (uc3);
    \draw[extends] (uc4) -- node[above, font=\tiny] {<<extends>>} (uc3);
    \draw[extends] (uc5) -- node[below, font=\tiny] {<<extends>>} (uc3);
    
\end{tikzpicture}
\end{center}

\subsubsection{UC02 : Surveillance Satellite}

\begin{center}
\begin{tikzpicture}[
    usecase/.style={ellipse, draw=accentTeal, line width=1.5pt, fill=skyBlue!30, text width=2.2cm, align=center, font=\footnotesize, minimum height=0.9cm},
    extends/.style={->, >=stealth, dashed, line width=1pt, textGray},
    includes/.style={->, >=stealth, dashed, line width=1pt, primaryGreen}
]
    % Acteurs
    \node (user) at (-3, 1) {\faUser};
    \node[font=\small\bfseries, below] at (-3, 0.5) {Utilisateur};
    
    \node (admin) at (-3, -2) {\faUserCog};
    \node[font=\small\bfseries, below] at (-3, -2.5) {Admin};
    
    % Cas principal
    \node[usecase, fill=accentTeal!30, text width=3cm] (main) at (3, 0) {Surveillance\\Satellite};
    
    % Sous-cas
    \node[usecase] (uc1) at (7, 2.5) {Sélectionner\\zone};
    \node[usecase] (uc2) at (7, 1) {Récupérer\\image Sentinel};
    \node[usecase] (uc3) at (7, -0.5) {Analyser\\par CAM};
    \node[usecase] (uc4) at (7, -2) {Afficher\\résultat};
    \node[usecase] (uc5) at (11, 0) {Démarrer\\scan auto};
    
    % Connexions
    \draw[primaryGreen, line width=1.5pt] (user) -- (main);
    \draw[primaryGreen, line width=1.5pt] (admin) -- (main);
    \draw[includes] (main) -- node[above, font=\tiny] {<<include>>} (uc1);
    \draw[includes] (main) -- node[above, font=\tiny] {<<include>>} (uc2);
    \draw[includes] (main) -- node[above, font=\tiny] {<<include>>} (uc3);
    \draw[includes] (main) -- node[below, font=\tiny] {<<include>>} (uc4);
    \draw[extends] (uc5) -- node[above, font=\tiny] {<<extends>>} (main);
    
\end{tikzpicture}
\end{center}

\subsection{Description Textuelle des Cas d'Utilisation}

\begin{table}[H]
\centering
\caption{UC01 : Détection en Temps Réel}
\label{tab:uc01}
\rowcolors{2}{mintGreen!20}{white}
\begin{tabular}{>{\bfseries}p{3cm} p{10cm}}
\toprule
\rowcolor{primaryGreen}
\multicolumn{2}{l}{\textcolor{white}{\textbf{UC01 : Détection en Temps Réel}}} \\
\midrule
Acteur principal & Utilisateur (Opérateur de surveillance) \\
Acteurs secondaires & Système de notification, Modèle YOLOv8 \\
Description & Permet à l'utilisateur de détecter en temps réel la présence de feu ou de fumée via une webcam ou caméra connectée \\
Préconditions & \begin{itemize}[leftmargin=0.3cm, topsep=0pt, itemsep=0pt]
    \item Caméra/webcam connectée et fonctionnelle
    \item Modèle YOLOv8 chargé en mémoire
    \item Connexion au backend établie
\end{itemize} \\
\midrule
\multicolumn{2}{l}{\textbf{\textcolor{primaryGreen}{Scénario nominal :}}} \\
\multicolumn{2}{p{13cm}}{
\begin{enumerate}[leftmargin=0.5cm, topsep=0pt, itemsep=2pt]
    \item L'utilisateur accède à la page /realtime
    \item L'utilisateur clique sur "Démarrer la détection"
    \item Le système active le flux vidéo de la webcam
    \item Le système affiche le flux vidéo en temps réel
    \item Le système analyse chaque frame avec YOLOv8
    \item Si feu/fumée détecté : affichage des bounding boxes
    \item Le système génère une alerte visuelle
    \item [Optionnel] Le système envoie une notification
\end{enumerate}
} \\
\midrule
Postconditions & \begin{itemize}[leftmargin=0.3cm, topsep=0pt, itemsep=0pt]
    \item Flux vidéo affiché avec annotations
    \item Alertes générées si détection positive
    \item Logs de détection enregistrés
\end{itemize} \\
Exceptions & \begin{itemize}[leftmargin=0.3cm, topsep=0pt, itemsep=0pt]
    \item E1 : Caméra non accessible → Message d'erreur
    \item E2 : Modèle non chargé → Tentative de rechargement
\end{itemize} \\
\bottomrule
\end{tabular}
\end{table}

\vspace{0.5cm}

\begin{table}[H]
\centering
\caption{UC02 : Upload et Analyse d'Image}
\label{tab:uc02}
\rowcolors{2}{mintGreen!20}{white}
\begin{tabular}{>{\bfseries}p{3cm} p{10cm}}
\toprule
\rowcolor{primaryGreen}
\multicolumn{2}{l}{\textcolor{white}{\textbf{UC02 : Upload et Analyse d'Image}}} \\
\midrule
Acteur principal & Utilisateur \\
Acteurs secondaires & Modèle MobileNetV2 \\
Description & Permet d'uploader une image et d'obtenir sa classification (Fire/Smoke/Non-Fire) \\
Préconditions & \begin{itemize}[leftmargin=0.3cm, topsep=0pt, itemsep=0pt]
    \item Image au format valide (JPEG, PNG)
    \item Taille < 10 MB
\end{itemize} \\
\midrule
\multicolumn{2}{l}{\textbf{\textcolor{primaryGreen}{Scénario nominal :}}} \\
\multicolumn{2}{p{13cm}}{
\begin{enumerate}[leftmargin=0.5cm, topsep=0pt, itemsep=2pt]
    \item L'utilisateur accède à la page /upload
    \item L'utilisateur sélectionne une image (drag-and-drop ou bouton)
    \item L'utilisateur clique sur "Analyser"
    \item Le système uploade l'image vers le backend
    \item Le backend prétraite l'image (224×224, normalisation)
    \item Le modèle MobileNetV2 effectue la classification
    \item Le système affiche le résultat : classe + score de confiance
\end{enumerate}
} \\
\midrule
Postconditions & Résultat de classification affiché avec probabilités \\
Exceptions & \begin{itemize}[leftmargin=0.3cm, topsep=0pt, itemsep=0pt]
    \item E1 : Format invalide → Message d'erreur avec formats acceptés
    \item E2 : Image trop grande → Message avec taille maximale
\end{itemize} \\
\bottomrule
\end{tabular}
\end{table}

\subsection{Diagramme de Séquence - Détection Temps Réel}

Ce diagramme illustre les interactions entre les composants lors d'une session de détection en temps réel.

\vspace{0.5cm}
\begin{center}
\begin{tikzpicture}[
    lifeline/.style={dashed, line width=1pt, textGray},
    message/.style={->, >=stealth, line width=1pt, primaryGreen},
    return/.style={->, >=stealth, dashed, line width=1pt, accentTeal},
    actor/.style={rectangle, draw=primaryGreen, line width=1.5pt, fill=mintGreen, minimum width=1.8cm, minimum height=0.8cm, align=center, font=\footnotesize}
]
    % Acteurs
    \node[actor] (user) at (0, 0) {Utilisateur};
    \node[actor] (front) at (3.5, 0) {Frontend\\React};
    \node[actor] (back) at (7, 0) {Backend\\FastAPI};
    \node[actor] (yolo) at (10.5, 0) {YOLOv8\\Model};
    \node[actor] (notif) at (14, 0) {Telegram\\Bot};
    
    % Lifelines
    \draw[lifeline] (0, -0.5) -- (0, -11);
    \draw[lifeline] (3.5, -0.5) -- (3.5, -11);
    \draw[lifeline] (7, -0.5) -- (7, -11);
    \draw[lifeline] (10.5, -0.5) -- (10.5, -11);
    \draw[lifeline] (14, -0.5) -- (14, -11);
    
    % Messages
    \draw[message] (0, -1.5) -- node[above, font=\tiny] {1: Accéder /realtime} (3.5, -1.5);
    \draw[return] (3.5, -2) -- node[above, font=\tiny] {Page chargée} (0, -2);
    
    \draw[message] (0, -2.8) -- node[above, font=\tiny] {2: Clic "Démarrer"} (3.5, -2.8);
    \draw[message] (3.5, -3.3) -- node[above, font=\tiny] {3: GET /video\_feed} (7, -3.3);
    
    % Boucle de détection
    \draw[darkGreen, line width=1pt] (6.2, -4) rectangle (11.3, -7.5);
    \node[font=\tiny\bfseries, text=darkGreen] at (7.5, -3.8) {loop [pour chaque frame]};
    
    \draw[message] (7, -4.5) -- node[above, font=\tiny] {4: capture\_frame()} (7, -4.5);
    \draw[message] (7, -5) -- node[above, font=\tiny] {5: predict(frame)} (10.5, -5);
    \draw[return] (10.5, -5.5) -- node[above, font=\tiny] {boxes, scores} (7, -5.5);
    \draw[message] (7, -6) -- node[above, font=\tiny] {6: draw\_boxes()} (7, -6);
    \draw[message] (7, -7) -- node[above, font=\tiny] {7: encode MJPEG} (7, -7);
    
    % Condition alerte
    \draw[alertOrange, line width=1pt] (6.2, -8) rectangle (14.8, -9.5);
    \node[font=\tiny\bfseries, text=alertOrange] at (8, -7.8) {alt [si Fire détecté]};
    
    \draw[message] (7, -8.5) -- node[above, font=\tiny] {8: send\_alert()} (14, -8.5);
    \draw[return] (14, -9) -- node[above, font=\tiny] {OK} (7, -9);
    
    % Retour stream
    \draw[return] (7, -10) -- node[above, font=\tiny] {MJPEG stream} (3.5, -10);
    \draw[return] (3.5, -10.5) -- node[above, font=\tiny] {Affichage vidéo} (0, -10.5);
    
\end{tikzpicture}
\end{center}

\subsection{Diagramme de Séquence - Surveillance Satellite}

\vspace{0.5cm}
\begin{center}
\begin{tikzpicture}[
    lifeline/.style={dashed, line width=1pt, textGray},
    message/.style={->, >=stealth, line width=1pt, accentTeal},
    return/.style={->, >=stealth, dashed, line width=1pt, primaryGreen},
    actor/.style={rectangle, draw=accentTeal, line width=1.5pt, fill=skyBlue!20, minimum width=1.5cm, minimum height=0.8cm, align=center, font=\footnotesize}
]
    % Acteurs
    \node[actor] (user) at (0, 0) {Admin};
    \node[actor] (front) at (2.8, 0) {Frontend};
    \node[actor] (back) at (5.6, 0) {Backend};
    \node[actor] (sent) at (8.4, 0) {Sentinel\\Hub};
    \node[actor] (cam) at (11.2, 0) {CAM\\Model};
    \node[actor] (notif) at (14, 0) {Email\\Service};
    
    % Lifelines
    \draw[lifeline] (0, -0.5) -- (0, -10);
    \draw[lifeline] (2.8, -0.5) -- (2.8, -10);
    \draw[lifeline] (5.6, -0.5) -- (5.6, -10);
    \draw[lifeline] (8.4, -0.5) -- (8.4, -10);
    \draw[lifeline] (11.2, -0.5) -- (11.2, -10);
    \draw[lifeline] (14, -0.5) -- (14, -10);
    
    % Messages
    \draw[message] (0, -1.2) -- node[above, font=\tiny] {1: Démarrer scan} (2.8, -1.2);
    \draw[message] (2.8, -1.7) -- node[above, font=\tiny] {2: POST /start\_monitoring} (5.6, -1.7);
    
    % Boucle zones
    \draw[darkGreen, line width=1pt] (4.8, -2.3) rectangle (14.8, -8);
    \node[font=\tiny\bfseries, text=darkGreen] at (6.5, -2.1) {loop [pour chaque zone]};
    
    \draw[message] (5.6, -3) -- node[above, font=\tiny] {3: get\_image(zone)} (8.4, -3);
    \draw[return] (8.4, -3.5) -- node[above, font=\tiny] {image satellite} (5.6, -3.5);
    
    \draw[message] (5.6, -4.2) -- node[above, font=\tiny] {4: analyze(image)} (11.2, -4.2);
    \draw[return] (11.2, -4.7) -- node[above, font=\tiny] {prediction, heatmap} (5.6, -4.7);
    
    % Condition feu
    \draw[moroccanRed, line width=1pt] (4.8, -5.3) rectangle (14.8, -7.5);
    \node[font=\tiny\bfseries, text=moroccanRed] at (6.8, -5.1) {alt [si Fire détecté]};
    
    \draw[message] (5.6, -6) -- node[above, font=\tiny] {5: send\_email(details)} (14, -6);
    \draw[return] (14, -6.5) -- node[above, font=\tiny] {sent OK} (5.6, -6.5);
    \draw[message] (5.6, -7) -- node[above, font=\tiny] {6: log\_detection()} (5.6, -7);
    
    % Retour
    \draw[return] (5.6, -8.5) -- node[above, font=\tiny] {scan results} (2.8, -8.5);
    \draw[return] (2.8, -9) -- node[above, font=\tiny] {Affichage rapport} (0, -9);
    
\end{tikzpicture}
\end{center}

\newpage
\subsection{Diagramme de Classes}

Le diagramme de classes présente la structure statique du système, avec les principales classes et leurs relations.

\vspace{0.5cm}
\begin{center}
\begin{tikzpicture}[
    class/.style={rectangle, draw=primaryGreen, line width=1.5pt, fill=mintGreen!30, rounded corners=3pt, minimum width=4cm, align=left, font=\footnotesize},
    inherit/.style={->, >=open triangle 60, line width=1pt, primaryGreen},
    compose/.style={-*, line width=1pt, darkGreen},
    assoc/.style={-, line width=1pt, textGray}
]
    % Classes Services
    \node[class] (yolo) at (0, 6) {
        \textbf{\textcolor{darkGreen}{YoloService}}\\
        \rule{3.8cm}{0.5pt}\\
        - model: YOLO\\
        - class\_names: list\\
        - confidence: float\\
        \rule{3.8cm}{0.5pt}\\
        + load\_model()\\
        + detect(frame)\\
        + draw\_boxes()
    };
    
    \node[class] (mobile) at (5, 6) {
        \textbf{\textcolor{darkGreen}{ClassificationService}}\\
        \rule{3.8cm}{0.5pt}\\
        - model: tf.Model\\
        - img\_size: tuple\\
        \rule{3.8cm}{0.5pt}\\
        + load\_model()\\
        + classify(image)\\
        + preprocess()
    };
    
    \node[class] (firms) at (10, 6) {
        \textbf{\textcolor{darkGreen}{FirmsService}}\\
        \rule{3.8cm}{0.5pt}\\
        - api\_key: str\\
        - regions: dict\\
        \rule{3.8cm}{0.5pt}\\
        + fetch\_data(region)\\
        + get\_wind\_data()\\
        + parse\_hotspots()
    };
    
    \node[class] (sentinel) at (0, 1) {
        \textbf{\textcolor{darkGreen}{SentinelService}}\\
        \rule{3.8cm}{0.5pt}\\
        - client\_id: str\\
        - client\_secret: str\\
        - zones: list\\
        \rule{3.8cm}{0.5pt}\\
        + authenticate()\\
        + get\_image(zone)\\
        + get\_evalscript()
    };
    
    \node[class] (prediction) at (5, 1) {
        \textbf{\textcolor{darkGreen}{PredictionService}}\\
        \rule{3.8cm}{0.5pt}\\
        - base\_radius: float\\
        \rule{3.8cm}{0.5pt}\\
        + calculate(params)\\
        + get\_brightness\_factor()\\
        + get\_wind\_factor()
    };
    
    \node[class] (monitoring) at (10, 1) {
        \textbf{\textcolor{darkGreen}{MonitoringService}}\\
        \rule{3.8cm}{0.5pt}\\
        - scheduler: APScheduler\\
        - is\_running: bool\\
        - interval: int\\
        \rule{3.8cm}{0.5pt}\\
        + start()\\
        + stop()\\
        + scan\_all\_zones()
    };
    
    \node[class] (notif) at (5, -4) {
        \textbf{\textcolor{darkGreen}{NotificationService}}\\
        \rule{3.8cm}{0.5pt}\\
        - telegram\_token: str\\
        - email\_config: dict\\
        - cooldown: int\\
        \rule{3.8cm}{0.5pt}\\
        + send\_telegram()\\
        + send\_email()\\
        + check\_cooldown()
    };
    
    % Relations
    \draw[compose] (monitoring) -- (sentinel);
    \draw[compose] (monitoring) -- (notif);
    \draw[assoc] (monitoring) -- (firms);
    \draw[assoc] (prediction) -- (firms);
    
\end{tikzpicture}
\end{center}

\subsection{Diagramme d'Activité - Processus de Détection}

\vspace{0.5cm}
\begin{center}
\begin{tikzpicture}[
    start/.style={circle, fill=darkGreen, minimum size=0.5cm},
    stop/.style={circle, draw=darkGreen, line width=2pt, fill=darkGreen, minimum size=0.5cm},
    activity/.style={rectangle, rounded corners=8pt, draw=primaryGreen, line width=1.5pt, fill=mintGreen!50, text width=2.5cm, minimum height=1cm, align=center, font=\small},
    decision/.style={diamond, draw=alertOrange, line width=1.5pt, fill=sunYellow!30, aspect=2, font=\small},
    arrow/.style={->, >=stealth, line width=1.5pt, primaryGreen}
]
    % Start
    \node[start] (start) at (0, 0) {};
    
    % Activities
    \node[activity] (a1) at (3, 0) {Initialiser\\caméra};
    \node[activity] (a2) at (6.5, 0) {Capturer\\frame};
    \node[activity] (a3) at (10, 0) {Prétraiter\\image};
    \node[activity] (a4) at (13.5, 0) {Inférence\\YOLO};
    
    \node[decision] (d1) at (13.5, -2.5) {Feu?};
    
    \node[activity] (a5) at (10, -2.5) {Dessiner\\boxes};
    \node[activity] (a6) at (16.5, -2.5) {Envoyer\\alerte};
    
    \node[activity] (a7) at (6.5, -2.5) {Encoder\\MJPEG};
    \node[activity] (a8) at (3, -2.5) {Streamer\\vers client};
    
    \node[decision] (d2) at (3, -5) {Stop?};
    
    \node[stop] (stop) at (0, -5) {};
    
    % Arrows
    \draw[arrow] (start) -- (a1);
    \draw[arrow] (a1) -- (a2);
    \draw[arrow] (a2) -- (a3);
    \draw[arrow] (a3) -- (a4);
    \draw[arrow] (a4) -- (d1);
    \draw[arrow] (d1) -- node[left, font=\tiny] {Non} (a5);
    \draw[arrow] (d1) -- node[above, font=\tiny] {Oui} (a6);
    \draw[arrow] (a6) |- (a5);
    \draw[arrow] (a5) -- (a7);
    \draw[arrow] (a7) -- (a8);
    \draw[arrow] (a8) -- (d2);
    \draw[arrow] (d2) -- node[above, font=\tiny] {Oui} (stop);
    \draw[arrow] (d2.east) -- ++(1,0) |- node[right, font=\tiny, pos=0.25] {Non} (a2.south);
    
\end{tikzpicture}
\end{center}

\subsection{Diagramme de Déploiement}

Le diagramme de déploiement illustre l'architecture physique du système et la répartition des composants sur les différents nœuds.

\vspace{0.5cm}
\begin{center}
\begin{tikzpicture}[
    node/.style={rectangle, draw=darkGreen, line width=2pt, fill=mintGreen!20, rounded corners=5pt, minimum width=4.5cm, minimum height=3cm},
    component/.style={rectangle, draw=accentTeal, line width=1pt, fill=skyBlue!20, rounded corners=3pt, minimum width=3.5cm, minimum height=0.7cm, font=\footnotesize},
    external/.style={rectangle, draw=textGray, line width=1.5pt, fill=softGray, rounded corners=5pt, minimum width=3cm, minimum height=1.5cm},
    arrow/.style={<->, >=stealth, line width=1.5pt, primaryGreen}
]
    % Nœud Client
    \node[node, label={[font=\small\bfseries, text=darkGreen]above:Client (Browser)}] (client) at (0, 0) {};
    \node[component] at (0, 0.5) {React Application};
    \node[component] at (0, -0.4) {Leaflet Maps};
    \node[component] at (0, -1.3) {TailwindCSS};
    
    % Nœud Serveur
    \node[node, minimum height=5cm, label={[font=\small\bfseries, text=darkGreen]above:Serveur Backend}] (server) at (7, 0) {};
    \node[component] at (7, 1.5) {FastAPI};
    \node[component] at (7, 0.6) {TensorFlow};
    \node[component] at (7, -0.3) {YOLOv8/Ultralytics};
    \node[component] at (7, -1.2) {APScheduler};
    \node[component] at (7, -2.1) {SMTP/Telegram};
    
    % Services Externes
    \node[external] (nasa) at (13, 2) {NASA FIRMS};
    \node[external] (sentinel) at (13, 0) {Sentinel Hub};
    \node[external] (meteo) at (13, -2) {Open-Meteo};
    
    % Connexions
    \draw[arrow] (client) -- node[above, font=\footnotesize] {HTTP/REST} node[below, font=\footnotesize] {MJPEG} (server);
    \draw[arrow] (server) -- node[above, font=\tiny] {API} (nasa);
    \draw[arrow] (server) -- node[above, font=\tiny] {OAuth2} (sentinel);
    \draw[arrow] (server) -- node[above, font=\tiny] {API} (meteo);
    
\end{tikzpicture}
\end{center}

\vspace{1cm}

% Transition vers le chapitre suivant
\begin{center}
\begin{tikzpicture}
    \node[
        fill=primaryGreen!10,
        draw=primaryGreen,
        line width=1.5pt,
        rounded corners=12pt,
        inner sep=20pt,
        text width=13cm,
        align=center
    ] {
        \textcolor{primaryGreen}{\fontsize{24}{28}\selectfont\faArrowCircleRight}\\[15pt]
        \large\textbf{Chapitre Suivant}\\[10pt]
        \normalsize Le prochain chapitre détaille la conception et l'architecture technique\\
        du système AI Sentinel, incluant l'architecture logicielle,\\
        la conception des modèles IA et l'interface utilisateur.\\[10pt]
        \textit{\textcolor{textGray}{Chapitre IV --- Conception et Architecture}}
    };
\end{tikzpicture}
\end{center}

