% ============================================
% CONCLUSION GÉNÉRALE ET PERSPECTIVES
% ============================================
\chapter*{Conclusion Générale et Perspectives}
\addcontentsline{toc}{chapter}{Conclusion Générale et Perspectives}
\thispagestyle{fancy}

\section*{Synthèse des Travaux Réalisés}

\lettrine[lines=3, lhang=0.15, loversize=0.1, findent=3pt]{\textcolor{primaryGreen}{A}}{u terme} de ce projet de fin d'études, nous pouvons dresser un bilan positif des travaux réalisés. Le système \textbf{AI Sentinel} répond aux objectifs initialement fixés : développer une plateforme complète de détection et de prévention des incendies de forêt exploitant les technologies d'intelligence artificielle les plus avancées.

\begin{greenbox}[\faBullseye\ Rappel des Objectifs]
\begin{itemize}[leftmargin=0.5cm, itemsep=5pt]
    \item[\textcolor{primaryGreen}{\faCheckCircle}] Développer un système de détection en temps réel via analyse vidéo
    \item[\textcolor{primaryGreen}{\faCheckCircle}] Intégrer des données satellites pour une surveillance à grande échelle
    \item[\textcolor{primaryGreen}{\faCheckCircle}] Implémenter un algorithme de prédiction de propagation
    \item[\textcolor{primaryGreen}{\faCheckCircle}] Créer un système d'alertes multi-canaux réactif
    \item[\textcolor{primaryGreen}{\faCheckCircle}] Concevoir une interface utilisateur moderne et intuitive
\end{itemize}
\end{greenbox}

\subsection*{Récapitulatif des Fonctionnalités Développées}

Le système AI Sentinel offre aujourd'hui un ensemble complet de fonctionnalités opérationnelles :

\begin{table}[H]
\centering
\rowcolors{2}{mintGreen!30}{white}
\begin{tabular}{l p{8cm} c}
\toprule
\rowcolor{primaryGreen}
\textcolor{white}{\textbf{Module}} & \textcolor{white}{\textbf{Fonctionnalité}} & \textcolor{white}{\textbf{Statut}} \\
\midrule
Détection Temps Réel & Analyse vidéo YOLOv8 avec bounding boxes & \textcolor{primaryGreen}{\faCheckCircle} \\
Classification Image & Upload et analyse MobileNetV2 (3 classes) & \textcolor{primaryGreen}{\faCheckCircle} \\
Analyse Vidéo & Traitement de fichiers vidéo avec export annoté & \textcolor{primaryGreen}{\faCheckCircle} \\
Surveillance Satellite & Intégration Sentinel Hub avec modèle CAM & \textcolor{primaryGreen}{\faCheckCircle} \\
Prévention Hotspots & Carte interactive NASA FIRMS & \textcolor{primaryGreen}{\faCheckCircle} \\
Prédiction Propagation & Algorithme empirique avec visualisation & \textcolor{primaryGreen}{\faCheckCircle} \\
Notifications & Alertes Email et Telegram automatiques & \textcolor{primaryGreen}{\faCheckCircle} \\
Interface Web & Application React moderne et responsive & \textcolor{primaryGreen}{\faCheckCircle} \\
\bottomrule
\end{tabular}
\end{table}

\subsection*{Technologies Maîtrisées}

Ce projet nous a permis d'acquérir une expertise pratique sur un large spectre technologique :

\begin{center}
\begin{tikzpicture}[
    tech/.style={rectangle, rounded corners=5pt, draw=primaryGreen, line width=1pt, fill=mintGreen!50, minimum width=2.5cm, minimum height=0.8cm, align=center, font=\small}
]
    % Backend
    \node[font=\bfseries, text=darkGreen] at (0, 3) {Backend};
    \node[tech] at (0, 2.2) {Python};
    \node[tech] at (0, 1.3) {FastAPI};
    \node[tech] at (0, 0.4) {TensorFlow};
    
    % IA/ML
    \node[font=\bfseries, text=darkGreen] at (3.5, 3) {IA/ML};
    \node[tech] at (3.5, 2.2) {YOLOv8};
    \node[tech] at (3.5, 1.3) {MobileNetV2};
    \node[tech] at (3.5, 0.4) {OpenCV};
    
    % Frontend
    \node[font=\bfseries, text=darkGreen] at (7, 3) {Frontend};
    \node[tech] at (7, 2.2) {React/TS};
    \node[tech] at (7, 1.3) {TailwindCSS};
    \node[tech] at (7, 0.4) {Leaflet};
    
    % APIs
    \node[font=\bfseries, text=darkGreen] at (10.5, 3) {APIs};
    \node[tech] at (10.5, 2.2) {NASA FIRMS};
    \node[tech] at (10.5, 1.3) {Sentinel Hub};
    \node[tech] at (10.5, 0.4) {Telegram};
\end{tikzpicture}
\end{center}

% ============================================
\section*{Contributions et Points Forts}
% ============================================

Le projet AI Sentinel se distingue par plusieurs contributions significatives dans le domaine de la détection d'incendies :

\begin{objectifbox}{Points Forts du Projet}

\textbf{\faLayerGroup\ Solution Complète de Bout en Bout}

Contrairement aux solutions partielles existantes, AI Sentinel offre une plateforme intégrée couvrant l'ensemble du cycle de détection : surveillance préventive, détection en temps réel, prédiction de propagation, et alertes automatisées.

\vspace{0.3cm}

\textbf{\faSatelliteDish\ Approche Multi-Modale}

La combinaison de sources de données complémentaires (caméras au sol + satellites) permet une couverture optimale. Les caméras offrent la réactivité tandis que les satellites assurent la couverture géographique.

\vspace{0.3cm}

\textbf{\faBolt\ Alertes Temps Réel}

Le système de notifications multi-canaux (Email + Telegram) garantit que les alertes atteignent les opérateurs rapidement, où qu'ils se trouvent.

\vspace{0.3cm}

\textbf{\faDesktop\ Interface Moderne et Intuitive}

L'interface utilisateur, développée avec React et TailwindCSS, offre une expérience utilisateur fluide avec un design thématique cohérent (thème vert nature).

\vspace{0.3cm}

\textbf{\faCode\ Architecture Modulaire et Extensible}

La conception en services indépendants permet une maintenance aisée et l'ajout futur de nouvelles fonctionnalités sans refonte majeure.
\end{objectifbox}

% ============================================
\section*{Difficultés Rencontrées}
% ============================================

Le développement d'AI Sentinel n'a pas été sans défis. Cette section présente les principales difficultés rencontrées et les solutions apportées.

\begin{alertbox}{Défis Techniques Surmontés}

\textbf{\faDatabase\ Collecte et Annotation de Données}

La constitution d'un dataset de qualité pour l'entraînement des modèles a nécessité la combinaison de plusieurs sources publiques et un travail de curation pour éliminer les images de mauvaise qualité.

\textbf{Solution :} Utilisation de datasets préexistants (Kaggle) combinés avec des techniques d'augmentation de données.

\vspace{0.3cm}

\textbf{\faTachometerAlt\ Optimisation des Performances Temps Réel}

Atteindre 30 FPS pour la détection vidéo tout en maintenant une précision acceptable a requis un travail d'optimisation considérable.

\textbf{Solution :} Utilisation de YOLOv8n (nano), optimisation du pipeline de traitement, exploitation du GPU lorsque disponible.

\vspace{0.3cm}

\textbf{\faPlug\ Intégration des APIs Externes}

Les APIs NASA FIRMS et Sentinel Hub présentent des particularités (authentification OAuth2, quotas, formats de données) qui ont nécessité une attention particulière.

\textbf{Solution :} Création de services dédiés avec gestion robuste des erreurs, retry automatique et cache.

\vspace{0.3cm}

\textbf{\faExclamationTriangle\ Gestion des Faux Positifs}

La distinction entre un véritable incendie et des phénomènes visuellement similaires (coucher de soleil, reflets) a représenté un défi pour les modèles.

\textbf{Solution :} Enrichissement du dataset avec des contre-exemples, ajustement des seuils de confiance, validation manuelle optionnelle.
\end{alertbox}

% ============================================
\newpage
\section*{Perspectives d'Amélioration}
% ============================================

Le système AI Sentinel, bien que fonctionnel, offre de nombreuses possibilités d'évolution. Nous proposons une feuille de route en trois horizons temporels.

\subsection*{Court Terme (3-6 mois)}

\begin{infobox}{Améliorations Immédiates}
\begin{itemize}[leftmargin=0.5cm, itemsep=5pt]
    \item \textbf{Enrichissement du Dataset :} Collecte de données supplémentaires spécifiques au contexte marocain pour améliorer la précision des modèles
    \item \textbf{Authentification Utilisateur :} Ajout d'un système de login pour sécuriser l'accès et personnaliser l'expérience
    \item \textbf{Historique des Détections :} Base de données pour conserver et consulter les détections passées
    \item \textbf{Rapports Automatisés :} Génération de rapports PDF quotidiens/hebdomadaires
    \item \textbf{Configuration des Seuils :} Interface permettant aux utilisateurs d'ajuster les seuils d'alerte
\end{itemize}
\end{infobox}

\subsection*{Moyen Terme (6-12 mois)}

\begin{infobox}{Évolutions Structurantes}
\begin{itemize}[leftmargin=0.5cm, itemsep=5pt]
    \item \textbf{Intégration de Drones :} Support de flux vidéo provenant de drones pour surveillance active des zones difficiles d'accès
    \item \textbf{Modèle de Prédiction ML :} Remplacement de l'algorithme heuristique par un modèle de machine learning entraîné sur des données historiques de propagation
    \item \textbf{Application Mobile Native :} Développement d'applications iOS et Android pour les équipes terrain
    \item \textbf{Multi-Caméras :} Support de plusieurs flux vidéo simultanés avec gestion centralisée
    \item \textbf{Intégration Météo Avancée :} Incorporation des prévisions météo pour l'anticipation des risques
\end{itemize}
\end{infobox}

\subsection*{Long Terme (1-3 ans)}

\begin{greenbox}[\faRocket\ Vision à Long Terme]
\begin{itemize}[leftmargin=0.5cm, itemsep=5pt]
    \item \textbf{Déploiement à l'Échelle Nationale :} Couverture complète des zones forestières marocaines avec infrastructure cloud
    \item \textbf{Partenariat Institutionnel :} Collaboration avec l'ANEF, la Protection Civile et les autorités régionales
    \item \textbf{Système d'Aide à la Décision :} Outil complet pour les pompiers intégrant optimisation des ressources et routage
    \item \textbf{Edge Computing :} Déploiement de modèles légers sur dispositifs embarqués pour détection locale sans latence réseau
    \item \textbf{Contribution Open Source :} Publication du code et des modèles pour favoriser l'adoption et les contributions communautaires
\end{itemize}
\end{greenbox}

% ============================================
\section*{Compétences Acquises}
% ============================================

Ce projet de fin d'études a été l'occasion d'acquérir et de consolider un ensemble de compétences transversales :

\begin{center}
\begin{tikzpicture}[
    skill/.style={rectangle, rounded corners=8pt, draw=primaryGreen, line width=2pt, fill=mintGreen!30, text width=5.5cm, minimum height=2cm, align=center}
]
    \node[skill] (s1) at (0, 2) {
        \textcolor{darkGreen}{\faCode\ \textbf{Développement Full-Stack}}\\[5pt]
        \footnotesize Backend Python/FastAPI\\
        Frontend React/TypeScript
    };
    
    \node[skill] (s2) at (7, 2) {
        \textcolor{darkGreen}{\faBrain\ \textbf{Deep Learning}}\\[5pt]
        \footnotesize Transfer Learning, CNN\\
        YOLO, Classification
    };
    
    \node[skill] (s3) at (0, -0.5) {
        \textcolor{darkGreen}{\faSatellite\ \textbf{Télédétection}}\\[5pt]
        \footnotesize Imagerie satellite\\
        APIs spatiales (Sentinel, FIRMS)
    };
    
    \node[skill] (s4) at (7, -0.5) {
        \textcolor{darkGreen}{\faTasks\ \textbf{Gestion de Projet}}\\[5pt]
        \footnotesize Méthodologie Agile\\
        Git, Documentation
    };
\end{tikzpicture}
\end{center}

\vspace{0.5cm}

\begin{quotebox}
\textit{``Ce projet représente une contribution modeste mais concrète à la protection de notre patrimoine forestier. Face à l'urgence climatique et à l'augmentation des risques d'incendies, chaque initiative technologique compte. AI Sentinel démontre que l'intelligence artificielle peut être mise au service de causes environnementales vitales.''}

\hfill --- \textbf{L'équipe du projet}
\end{quotebox}

% ============================================
% ANNEXES
% ============================================
\appendix
\chapter*{Annexes}
\addcontentsline{toc}{chapter}{Annexes}

\section*{Annexe A : Guide d'Installation}
\addcontentsline{toc}{section}{Annexe A : Guide d'Installation}

\begin{techbox}{Installation du Projet}
\begin{verbatim}
# Cloner le repository
git clone https://github.com/username/WildFireDetection.git
cd WildFireDetection

# ========== BACKEND ==========
cd backend

# Créer un environnement virtuel
python -m venv venv
source venv/bin/activate  # Linux/Mac
.\venv\Scripts\activate   # Windows

# Installer les dépendances
pip install -r requirements.txt

# Configurer les variables d'environnement
cp .env.example .env
# Éditer .env avec vos clés API

# Lancer le serveur
uvicorn main:app --reload --host 0.0.0.0 --port 8000

# ========== FRONTEND ==========
cd ../frontend

# Installer les dépendances
npm install

# Lancer le serveur de développement
npm run dev

# L'application est accessible sur http://localhost:5173
\end{verbatim}
\end{techbox}

\section*{Annexe B : Variables d'Environnement}
\addcontentsline{toc}{section}{Annexe B : Variables d'Environnement}

\begin{techbox}{Fichier .env}
\begin{verbatim}
# ========== SENTINEL HUB ==========
SENTINEL_CLIENT_ID=your_sentinel_client_id
SENTINEL_CLIENT_SECRET=your_sentinel_client_secret

# ========== NASA FIRMS ==========
NASA_FIRMS_MAP_KEY=your_firms_map_key

# ========== EMAIL NOTIFICATIONS ==========
SMTP_SERVER=smtp.gmail.com
SMTP_PORT=587
SMTP_EMAIL=your_email@gmail.com
SMTP_PASSWORD=your_app_password
ALERT_RECIPIENT=recipient@example.com

# ========== TELEGRAM BOT ==========
TELEGRAM_BOT_TOKEN=123456789:ABCdefGHIjklMNOpqrsTUVwxyz
TELEGRAM_CHAT_ID=-1001234567890

# ========== APPLICATION ==========
ENVIRONMENT=development
DEBUG=true
\end{verbatim}
\end{techbox}

\section*{Annexe C : Endpoints API}
\addcontentsline{toc}{section}{Annexe C : Endpoints API}

\begin{table}[H]
\centering
\caption{Liste complète des endpoints API}
\rowcolors{2}{mintGreen!20}{white}
\begin{tabular}{l l p{6cm}}
\toprule
\rowcolor{primaryGreen}
\textcolor{white}{\textbf{Méthode}} & \textcolor{white}{\textbf{Endpoint}} & \textcolor{white}{\textbf{Description}} \\
\midrule
GET & /video\_feed & Flux vidéo MJPEG temps réel \\
POST & /predict & Classification d'image (MobileNetV2) \\
POST & /detect/image & Détection sur image (YOLOv8) \\
POST & /detect/video & Détection sur vidéo \\
GET & /api/wildfire/realtime & Hotspots NASA FIRMS \\
POST & /api/satellite/scan & Scan zone satellite \\
POST & /api/prediction & Calcul propagation \\
POST & /start\_monitoring & Démarrer scheduler \\
POST & /stop\_monitoring & Arrêter scheduler \\
GET & /api/fwi/\{region\} & Indice FWI météo \\
\bottomrule
\end{tabular}
\end{table}

\section*{Annexe D : Bibliographie}
\addcontentsline{toc}{section}{Annexe D : Bibliographie}

\subsection*{Articles Scientifiques}

\begin{enumerate}
    \item Howard, A. G., et al. (2017). \textit{``MobileNets: Efficient Convolutional Neural Networks for Mobile Vision Applications''}. arXiv:1704.04861.
    
    \item Redmon, J., \& Farhadi, A. (2018). \textit{``YOLOv3: An Incremental Improvement''}. arXiv:1804.02767.
    
    \item Zhou, B., et al. (2016). \textit{``Learning Deep Features for Discriminative Localization''}. CVPR 2016.
    
    \item Jocher, G., et al. (2023). \textit{``Ultralytics YOLOv8''}. GitHub repository.
    
    \item Van Wagner, C. E. (1987). \textit{``Development and structure of the Canadian Forest Fire Weather Index System''}. Canadian Forestry Service.
\end{enumerate}

\subsection*{Documentation Technique}

\begin{itemize}[leftmargin=0.5cm, itemsep=3pt]
    \item FastAPI Documentation --- \url{https://fastapi.tiangolo.com/}
    \item TensorFlow Documentation --- \url{https://www.tensorflow.org/}
    \item Ultralytics YOLOv8 --- \url{https://docs.ultralytics.com/}
    \item Sentinel Hub Documentation --- \url{https://docs.sentinel-hub.com/}
    \item NASA FIRMS --- \url{https://firms.modaps.eosdis.nasa.gov/}
    \item React Documentation --- \url{https://react.dev/}
    \item Leaflet Documentation --- \url{https://leafletjs.com/}
    \item TailwindCSS --- \url{https://tailwindcss.com/}
\end{itemize}

\subsection*{Ressources et Datasets}

\begin{itemize}[leftmargin=0.5cm, itemsep=3pt]
    \item Copernicus Open Access Hub --- \url{https://scihub.copernicus.eu/}
    \item Kaggle Fire/Smoke Dataset --- \url{https://www.kaggle.com/datasets/}
    \item Open-Meteo API --- \url{https://open-meteo.com/}
\end{itemize}

\vspace{2cm}

% Fin du document
\begin{center}
\begin{tikzpicture}
    \node[
        fill=mintGreen,
        draw=primaryGreen,
        line width=2pt,
        rounded corners=15pt,
        inner sep=25pt,
        text width=10cm,
        align=center
    ] {
        \textcolor{darkGreen}{\fontsize{40}{48}\selectfont\faLeaf}\\[20pt]
        \Large\textbf{\textcolor{darkGreen}{AI Sentinel}}\\[10pt]
        \normalsize\textcolor{textGray}{Système de Détection des Incendies de Forêt}\\[5pt]
        \small\textcolor{textGray}{Projet de Fin d'Études --- 2024-2025}\\[15pt]
        \textcolor{primaryGreen}{\faHeart\ Protégeons nos forêts}
    };
\end{tikzpicture}
\end{center}
